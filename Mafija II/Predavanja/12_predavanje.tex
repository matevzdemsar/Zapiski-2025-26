\documentclass[a4paper]{article}
\usepackage{amsmath, amssymb, amsfonts}
\usepackage[margin=1in]{geometry}
\usepackage{graphicx}
\usepackage{tikz}
\usepackage{esint}
\setlength{\parindent}{0em}
\setlength{\parskip}{1ex}

\newcommand{\vct}[1]{\overrightarrow{#1}}
\newcommand{\dif}{\,\mathrm{d}}
\newcommand{\pd}[2]{\frac{\partial {#1}}{\partial {#2}}}
\newcommand{\dd}[2]{\frac{\mathrm{d} {#1}}{\mathrm{d} {#2}}}
\newcommand{\C}{\mathbb{C}}
\newcommand{\R}{\mathbb{R}}
\newcommand{\Q}{\mathbb{Q}}
\newcommand{\Z}{\mathbb{Z}}
\newcommand{\N}{\mathbb{N}}
\newcommand{\fn}[3]{{#1}\colon {#2} \rightarrow {#3}}
\newcommand{\avg}[1]{\left\langle {#1} \right\rangle}
\newcommand{\Sum}[2][0]{\sum_{{#2} = {#1}}^{\infty}}
\newcommand{\Lim}[1]{\lim_{{#1} \rightarrow \infty}}
\newcommand{\Binom}[2]{\begin{pmatrix} {#1} \cr {#2} \end{pmatrix}}
\newcommand{\duline}[1]{\underline{\underline{#1}}}
\newcommand{\bra}[1]{\left\langle {#1} \right|}
\newcommand{\ket}[1]{\left| {#1} \right\rangle}
\newcommand{\rot}{\vct{\nabla}\times}
\newcommand{\dvg}{\vct{\nabla}\cdot}
\renewcommand{\figurename}{Slika}

\begin{document}
\section{Integralska formulacija Greenove funkcije}
Od prej imamo Fredholmovo integralsko ena\v cbo:
\[u(\vct{r}) = h(\vct{r}) + \lambda\int_{\mathcal{D}}G_0(\vct{r}, \vct{r_0})V(\vct{r_0})u(\vct{r_0})\dif^3\vct{r_0}\]
\paragraph{Primer.} Sskanje vezanih lastnih stanj v 1D (\(E < 0\)):
\[-\frac{\hbar^2}{2m}\dd{^2}{x^2} + V(x)u = Eu\]
\[\left(\dd{^2}{x^2} - \kappa^2\right)u = -\frac{2m}{\hbar}V(x)u,\qquad \kappa^2 = -\frac{2mE}{\hbar}\]
Dobili smo operator \(\mathcal{L} = \dd{^2}{x^2} - \kappa^2\). Gre za Helmholtzov operator, za katerega smo napisali 2D in 3D Greenove funkcije. V 1D se zadeve lotimo tako, da
poi\v s\v cemo funkcijo, ki ima v nekem \(x_0\) singularnost, drugje pa re\v si homogeno ena\v cbo za deni operator. V tem primeru so re\v sitve homogene ena\v cbe eksponentne funkcije,
torej je
\[G_0(x, x_0) = -\frac{1}{2\kappa}\,e^{-\kappa|x - x_0|}\]
Tako najdemo tudi funkcijo \(h\):
\[h = c_1e^{-\kappa x} + c_2 e^{\kappa x}\]
Zaradi lastnosti re\v sitve \(u(|x| \to \infty) = 0\) je \(c_1 = c_2 = 0\).
Tako dobimo pogoj, ki mu morajo zado\v s\v cati lastne funkcije \(u\):
\[u(x) = -\int_{-\infty}^{\infty}\frac{h}{\hbar^2\kappa}\,e^{-\kappa|x - x_0|}V(x_0)u(x_0)\dif x_0\]
\paragraph{Opomba.} \v Ce je \(V(x) \sim \delta(x)\), je lastna funcija ravno \(\exp(-\kappa|x - x_0|)\) in imamo samo eno vezano stanje.
\subsection{Sipalni problem}
\paragraph{Primer.} Sipanje v kvantni mehaniki, 3D. Imamo vpadni val \(\exp(i\vct{k_0}\cdot\vct{r})\) na nek potencial \(V(\vct{r})\).
\[(\nabla^2 + k_0^2)u = \frac{2m}{\hbar^2}V(\vct{r})u(\vct{r})\]
Pri tem je \(k_0^2 = 2mE/\hbar^2\). Za Helmholtzov problem v 3D imamo Greenovo funkcijo
\[G_0(\vct{r}, \vct{r_0}) = -\frac{e^{ikr_0}}{4\pi r}\]
Poi\v s\v cimo \v se funkcijo \(h\) kot re\v sitev ena\v cbe \[\mathcal{L}h = 0\]
Ker moramo poskrbeti za robni pogoj v \(r \to \infty\), izberemo kar vpadni val \(\exp(i\vct{k}\cdot\vct{r})\).
Re\v sitev mora torej izpolnjevati pogoj
\[u(\vct{r}) = e^{-i\vct{k}\cdot\vct{r}} - \frac{2m}{\hbar^2}\frac{e^{i\vct{k}\cdot(\vct{r} - \vct{r_0})}}{4\pi|\vct{r} - \vct{r_0}|}V(\vct{r_0})u(\vct{r_0})\dif^3\vct{r}\]
Tej ena\v cbi se re\v ce tudi Lippmann-Schwingerjeva ena\v cba. \v Ce je \(V\) \v sibek (oziroma \(\lambda \ll 1\)), lahko napi\v semo preturbacijsko vrsto za Fredholovo ena\v cbo: \\[2mm]
Ni\v cti red:
\[u_0(\vct{r}) = h(\vct{r})\]
Prvi red:
\[u_1(\vct{r}) = \lambda\int G_0(\vct{r}, \vct{r_1})V(\vct{r_1})h(\vct{r_1})\dif^3\vct{r_1}\]
Drugi red:
\[u_2(\vct{r}) = u_1(\vct{r}) + \lambda^2\int G_0(\vct{r}, \vct{r_0}) G_0(\vct{r_1}, \vct{r_2})V(\vct{r_1})V(\vct{r_2})\dif^3\vct{r_1}\dif^3\vct{r_2}\]
Tretji red: (namesto \(h\) vstavimo \(u_1\) in o\v cedimo, kar se o\v cediti da). Dobljeni vrsti pravimo tudi Neumannova ali Bornova vrsta. \\[2mm]
Definiramo linearni operator \[K_0(u) = \int_{\mathcal{D}}G_0(\vct{r}, \vct{r_0})V(\vct{r_0})u(\vct{r_0})\dif^3\vct{r_0}\]
Tako lahko Friedholmovo ena\v cbo predstavimo kot
\[u = h + \lambda K_0 u\]
\[\left(I - \lambda K_0\right)u = h\]
\[u = \left(1 - \lambda K_0\right)^{-1}h\]
Neumannova (Bornova) vrsta je nekak\v cen Taylorjev razvoj operatorja \((I - \lambda K_0)^{-1}\).
\paragraph{Opomba.} Kak\v sen je konvergen\v cni radij take vrste? Je kon\v cen, obstaja pa tudi bolj\v sa vrsta, ki vedno konvergira, imenovana Firedholmova vrsta. Mi je ne bomo potrebovali.
\end{document}