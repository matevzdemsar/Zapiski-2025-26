\documentclass[a4paper]{article}
\usepackage{amsmath, amssymb, amsfonts}
\usepackage[margin=1in]{geometry}
\usepackage{graphicx}
\usepackage{tikz}
\usepackage{esint}
\setlength{\parindent}{0em}
\setlength{\parskip}{1ex}

\newcommand{\vct}[1]{\overrightarrow{#1}}
\newcommand{\dif}{\,\mathrm{d}}
\newcommand{\pd}[2]{\frac{\partial {#1}}{\partial {#2}}}
\newcommand{\dd}[2]{\frac{\mathrm{d} {#1}}{\mathrm{d} {#2}}}
\newcommand{\C}{\mathbb{C}}
\newcommand{\R}{\mathbb{R}}
\newcommand{\Q}{\mathbb{Q}}
\newcommand{\Z}{\mathbb{Z}}
\newcommand{\N}{\mathbb{N}}
\newcommand{\fn}[3]{{#1}\colon {#2} \rightarrow {#3}}
\newcommand{\avg}[1]{\left\langle {#1} \right\rangle}
\newcommand{\Sum}[2][0]{\sum_{{#2} = {#1}}^{\infty}}
\newcommand{\Lim}[1]{\lim_{{#1} \rightarrow \infty}}
\newcommand{\Binom}[2]{\begin{pmatrix} {#1} \cr {#2} \end{pmatrix}}
\newcommand{\duline}[1]{\underline{\underline{#1}}}
\newcommand{\bra}[1]{\left\langle {#1} \right|}
\newcommand{\ket}[1]{\left| {#1} \right\rangle}
\newcommand{\rot}{\vct{\nabla}\times}
\newcommand{\dvg}{\vct{\nabla}\cdot}
\renewcommand{\figurename}{Slika}

\begin{document}
\section{Cilindri\v cne koordinate - nadaljevanje}
\paragraph{Primer.} Besselobve funkcije z necelimi indeksi. Imejmo na primer odsek cilindra, ki bodi v \(z\) smeri neskon\v cen (translacijska invariantnost pomeni, da obravnavamo le odvisnost od \(r\) in \(\varphi\)). Robni pogoj je Dirichletov, torej \(u(\partial D) = 0\).
Spet re\v sujemo s separacijo, za\v cnimo s kotno smerjo.
\[\phi(\varphi) \sim \sin\left(\frac{m\pi}{\varphi_0}\phi\right),~m = 1, 2, ...\]
Ozna\v cimo \(\mu = m\pi/\varphi_0 \in \R\). Za \(r\) torej dobimo:
\[r^2R'' + rR' + [r^2\lambda - \mu^2]R\]
Ta ena\v cba ni v Sturm-Liouvillovi obliki, jo pa lahko na obeh straneh delimo z \(r\) in dobimo Sturm-Liouvillov problem z ute\v zjo \(w = r\). Pri re\v sitvi imamo slede\v ce primere:
\begin{itemize}
    \item \(\lambda = 0\): Nimamo nobene re\v sitve, ki zado\v s\v cajo Dirichletovemu robnemu pogoju.
    \item \(\lambda < 0\): Na\v se re\v sitve bi bile modoficirane Besselove funkcije, ki ne zado\v s\v cajo robnim pogojem.
    \item \(\lambda > 0\): Re\v sitve so Besselove funkcije \(J_\mu(\xi_{\mu, n} r/R_0)\), kjer je \(\mu \in \R\), \(n = 1, 2, ...\). Lastne vrednosti pa so \(\lambda = (\xi_{\mu, n}/R_0)^2\).
    Funkcije \(\phi_m\) so ortogonalne za razli\v cne \(m\), 
\end{itemize}
\subsection{Laplaceova ena\v cba}
\[\nabla^2 u = 0\]
Imejmo valj in robna pogoja \(u = f(r, \varphi)\) na zgornji ploskvi, \(u = h(r, \varphi)\) na spodnji ploskvi in \(u = 0\) na pla\v s\v cu valja.
Re\v sujemo s separacijo, v smeri \(\varphi\) dobimo \(\phi(\varphi) = \exp(im\varphi),~m = 0, 1, 2, ...\). V radialni smeri imamo spet Besselove funkcije \(J_m(\xi_{m, n}r/R_0)\). \\[2mm]
V \(z\) smeri smo imeli pri Helmholtzovi ena\v cbi mo\v znosti kotnih ali hiperboli\v cnih funkcij. Upo\v stevamo tudi lastnost
\[\nabla^2_rJ_m = -\frac{\xi}{R_0}J_m\]
Tako je v smeri \(z\) re\v sitev linearna kombinacija funkcij
\[\left\{\begin{matrix}
    \sinh\left(\frac{\xi_{m, n}}{R_0}z\right) \\ \cosh\left(\frac{\xi_{m, n}}{R_0}z\right)
\end{matrix}\right\}\]
Dolo\v citi moramo koeficiente za lastne funkcije \(Z(z)\). Definiramo naslednji funkciji:
\[v(z) = \sinh\left(\xi_{m, n}\frac{z}{R_0}\right)\cosh\left(\xi_{m, n}\frac{0}{R_0}\right) - \sinh\left(\xi_{m, n}\frac{0}{R_0}\right)\cosh\left(\xi_{m, n}\frac{z}{R_0}\right)\]
\[w(z) = \sinh\left(\xi_{m, n}\frac{z}{R_0}\right)\cosh\left(\xi_{m, n}\frac{H}{R_0}\right) - \sinh\left(\xi_{m, n}\frac{H}{R_0}\right)\cosh\left(\xi_{m, n}\frac{z}{R_0}\right)\]
Funkciji sta sestavljeni tako, da je \(v(0) = 0\) in \(w(H) = 0\). Tako je kon\v cna re\v sitev
\[u = \sum_{m, n}\left(\alpha_{m,n}v(z) + \beta_{m,n}w(z)\right)\,J_m\left(\xi_{m, n}\frac{r}{R_0}\right)\,e^{im\varphi}\]
koeficiente \(\alpha_{m,n}\) dobimo z razvojem funkcije \(f\), koeficiente \(\beta_{m,n}\) pa z razvojem funkcije \(h\).
\section{Sferi\v cne koordinate}
Obravnavali bomo Helmholtzovo ena\v cbo, pri Laplaceovi je stvar podobna.
\[\nabla^2u + \lambda u = 0\]
Zapi\v simo Laplaceov operator v sferi\v cnih koordinatah:
\[\nabla^2 u = \frac{1}{r^2}\pd{}{r}\left(r^2\pd{u}{r}\right) - \frac{L^2}{r^2}r\]
Definirali smo operator \(L\), in sicer tako, da je
\[\vct{L} = \vct{r} \times \vct{p},~~~ \vct{p} = -i \nabla\]
Ali druga\v ce:
\[L^2 = \vct{L} \cdot \vct{L} = -\frac{1}{\sin\vartheta}\pd{}{\vartheta}\left(\sin\vartheta\pd{}{\vartheta}\right) - \frac{1}{\sin\vartheta}\pd{^2}{\varphi^2}\]
Re\v sujemo kar s separacijo, najprej kotni del, ki ima lastne funkcije \(Y(\varphi, vartheta)\). po vzoru kvantne mehanike pi\v semo problem lastnih vrednosti kot
\[L^2Y(\vartheta, \varphi) = l(l+1)Y(\vartheta, \varphi)\]
Spet uporabimo separacijo:
\[Y(\vartheta, \varphi) = \phi(\varphi)\cdot\theta(\vartheta)\]
V smeri \(\varphi\):
\[\frac{\phi''}{\phi} = -m^2\]
\[\phi = \left\{\begin{matrix}
    \sin(m\varphi) \\ \cos(m\varphi)
\end{matrix}\right\}~,\qquad \begin{matrix}
    L_z = -i\pd{}{\varphi} \\ L_z^2\phi = m^2\phi
\end{matrix}\]
V smeri \(\vartheta\): Vzamemo spremenljivko \(t = \cos\vartheta\) in jo vstavimo v definicijo \(L^2\).
\[\dd{}{t}\left[(1-t^2)\dd{}{t}\right]\theta + \left[l(l+1) - \frac{m^2}{1-t^2}\right]\theta = 0\]
Gre za diferencialno ena\v cbo, imenovano Legendrova diferencialna ena\v cba z znanimi re\v sitvami
\[\theta = \left\{\begin{matrix}
    P_l^m \\ Q_l^m
\end{matrix}\right\}\]
Funkcijam \(P^l_m\) in \(Q_l^m\) pravimo pridru\v zene Legendrove funkcije. Parametra \(m\) in \(l\) sta odvisna od oblike problema.
\v Ce imamo opravka s celo sfero, se izka\v ze, da so dovoljene vrednosti \(m\) in \(l\) ravno cela \v stevila. Sama ena\v cba o tem ni\v cesar ne pove.
Pri kvantni mehaniki zahteve o \(m\) in \(l\) izpeljemo z algebrajskim postopkom: definiramo operatorja \(L_\pm = L_x \pm L_y\)
in opazimo, da nam operator zve\v ca ali zmanj\v sa \(m\) za \(\pm 1\). Nato zahtevamo, da imamo pri neki vrednosti \(m\) kon\v cno vrednost mo\v znosti za \(l\),
kar je mogo\v ce le, \v ce sta \(m\) in \(l\) celi ali polceli \v stevili. Vzrok za to je v zahtevi po konvergenci lastnih funkcij, ki pa je mogo\v ca, \v ce je \(L\) sebi adjungiran operator.
\subsection{Sferi\v cni harmoniki}
Za kotni del Helmholtzove ena\v cbe v sferi\v cnih koordinatah smo dobili lastne funkcije
\[Y_{lm}(\vartheta, \varphi) \sim P_l^m(\cos\vartheta)\,e^{im\varphi}\]
ki so ortonormirane:
\[\int Y_{lm}^* Y_{l'm'}\dif\varphi\sin\vartheta\dif\vartheta = \delta_{m, m'}\delta_{l, l'}\]
\paragraph{Opomba.} Funkcije \(Q_l^m\) imajo za cele indekse logaritemske singularnosti, zato obi\v cajno niso primerne kot re\v sitve. Kot primer:
\[Q_0 = \frac{1}{2}\ln\frac{1 + t}{1 - t},\qquad Q_1 = \frac{t}{2}\ln\frac{1 + t}{1 - t} - 1\]
Raidalni del:
\[\frac{1}{r}\dd{}{r}\left(r^2 R'\right) + \lambda R - \frac{l(l+1)}{r^2}R = 0\]
Naredimo substitucijo \(R(r) = Z(r)/\sqrt{r}\)
\[r^2Z'' + rZ' + \left[\lambda r^2 - \left(l + \frac{1}{2}\right)^2\right]Z = 0\]
To je Besselova diferencialna ena\v cba, katere re\v sitve so Bessli s polcelimi indeksi (\v ce je \(l\) celo \v stevilo, o \v cemer smo \v ze govorili).
Spet lo\v cimo primere:
\begin{itemize}
    \item \(\lambda = 0\): V resnici gre za Laplaceovo ena\v cbo. \[R = \left\{\begin{matrix}
        r^l \\ r^{-(l+1)}
    \end{matrix}\right\}\]
    Za celotno Laplaceovo ena\v cbo je re\v sitev oblike
    \[u(\vct{r}) = \sum_{l, m} c_{l, m} Y_{l, m}(\vartheta, \varphi)\left\{\begin{matrix}
        r^l \\ r^{-(l+1)}
    \end{matrix}\right\}\]
    \item \(\lambda < 0\): Re\v sitev zapi\v semo z modificirani sferi\v cnimi Besselovimi funukcijami (ozna\v cimo \(-\lambda = k^2\)): \[R = \left\{\begin{matrix}
        i_l(kr) \\ k_l(kr)
    \end{matrix}\right\}\]
    Funkcije \(i_l\) in \(k_l\) so eksponentne narave: \(i_l\) padajo, \(k_l\) nara\v s\v cajo, ni\v cel pa nimajo ne ene, ne druge.
    \item \(\lambda > 0\): Re\v sitev zapi\v semo s sferi\v cnimi Besselovimi funkcijami:
    \[R = \left\{\begin{matrix}
        j_l(kr) \\ y_l(kr)
    \end{matrix}\right\}\]
    Sferi\v cne Besseflove funkcije so definirane kot
    \[j_l(x) = \sqrt{\frac{\pi}{2x}}\,J_{l + 1/2}(x),\qquad j_0 = \frac{\sin x}{x}\]
    \[y_l(x) = \sqrt{\frac{\pi}{2x}}\,Y_{l + 1/2}(x),\qquad y_0 = -\frac{\cos x}{x}\]
    V limiti \(x \to 0\) spet velja \(y_l \to x^{-(l+1)}\).
\end{itemize}
\subsection{Lastnosti Legendrovih funkcij}
Sferni harmoniki so navedeni kot
\[Y_{lm}(t = \cos\vartheta, \varphi) = \sqrt{\frac{2l+1}{4\pi}\frac{(l-m)!}{(l+m)!}}P_l^m(t)\,e^{im\varphi}\]
Za sferne harmonike velja, da so ortogonalni in normirani:
\[\int_{-1}^{1}\int_{0}^{2\pi} Y_{lm}^* Y_{l'm'}^*\dif \varphi\dif t = \delta_{ll'}\delta_{mm'}\]
Gre v bistvu za to, da so \(P_l^m \perp P_{l'}^m\).
\paragraph{Definicija.} Legendrove funkcije so definirane kot
\[P_l^m = (-1)^m (1 - t^2)^{m/2} dd{^m}{t^m}P_l(t)\]
\(P_l(t)\) so Legendrovi polinomi, npr. \(P_0 = 1\), \(P_1 = t\), \dots
\paragraph{Legendrovi polinomi.}
Takoj vidimo, da velja \(P_l^m = 0\), \v ce je \(m > l\). Legendrovi polinomi so definirani kot
\[P_l = \frac{1}{2^l l!}\dd{^l}{t^l}\left[(t^2 - 1)^l\right]\]
So ortogonalni, velja:
\[\int_{-1}^{1}P_lP_{l'}\dif t = \frac{2}{2l + 1}\delta_{ll'}\]
Imamo tudi rekurzivno zvezo
\[(2l+1) t P_l = (l+1)P_{l+1} + lP_{l-1}\]
\paragraph{Generatrisa.} Legendrove funkcije so koeficienti razvoja funkcije
\[\frac{1}{\sqrt{1 - 2xt + x^2}} = \Sum{l}P_l(t)x^l\]
To lahko s pridom uporabimo, npr.
\[\frac{1}{|\vct{r_1} - \vct{r_2}|} = \frac{1}{\max(r_1, r_2)}\sum_{l=0}^{\infty}\left(\frac{\min(r_1, r_2)}{\max(r_1, r_2)}\right)^lP_l(\cos\alpha)\]
\subsection{Legendrove funkcije z necelimi indeksi}
Spomnimo se, da so Legendrove funkcije lastne funkcije kotnega dela ena\v cbe \[\nabla^2u + \lambda u = 0\]
Za \(m = 0\) zaradi rotacijske simatrije velja:
\begin{align*}
    P_\nu(1) & = 1\qquad\forall \nu \\
    P_\nu(-1) & = \infty\qquad\forall\nu\notin\Z \\
    Q_\nu(1) & = \infty\qquad\forall\nu \\
    Q_\nu(-1) & = \infty\qquad\forall\nu\notin\frac{1}{2}\Z
\end{align*}
Da na\v se re\v sitve ustrezajo Dirichletovim robim pogojem, smemo torej uporabljali le cele \(\nu\) in funkcije \(P\).
To pa velja le, \v ce re\v sujemo ena\v cbo na celi krogli. \v Ce gre za odsek krogle, lahko dobimo poljubne vrednosti \(\nu\), malo tako kot pri odsekih valja.
\end{document}