\documentclass[a4paper]{article}
\usepackage{amsmath, amssymb, amsfonts}
\usepackage[margin=1in]{geometry}
\usepackage{graphicx}
\usepackage{tikz}
\usepackage{esint}
\setlength{\parindent}{0em}
\setlength{\parskip}{1ex}

\newcommand{\vct}[1]{\overrightarrow{#1}}
\newcommand{\dif}{\,\mathrm{d}}
\newcommand{\pd}[2]{\frac{\partial {#1}}{\partial {#2}}}
\newcommand{\dd}[2]{\frac{\mathrm{d} {#1}}{\mathrm{d} {#2}}}
\newcommand{\C}{\mathbb{C}}
\newcommand{\R}{\mathbb{R}}
\newcommand{\Q}{\mathbb{Q}}
\newcommand{\Z}{\mathbb{Z}}
\newcommand{\N}{\mathbb{N}}
\newcommand{\fn}[3]{{#1}\colon {#2} \rightarrow {#3}}
\newcommand{\avg}[1]{\left\langle {#1} \right\rangle}
\newcommand{\Sum}[2][0]{\sum_{{#2} = {#1}}^{\infty}}
\newcommand{\Lim}[1]{\lim_{{#1} \rightarrow \infty}}
\newcommand{\Binom}[2]{\begin{pmatrix} {#1} \cr {#2} \end{pmatrix}}
\newcommand{\duline}[1]{\underline{\underline{#1}}}
\newcommand{\bra}[1]{\left\langle {#1} \right|}
\newcommand{\ket}[1]{\left| {#1} \right\rangle}
\newcommand{\rot}{\vct{\nabla}\times}
\newcommand{\dvg}{\vct{\nabla}\cdot}
\renewcommand{\figurename}{Slika}

\begin{document}
\section{Sturm-Liouviellov problem - nadaljevanje}
\subsection{Liouville-Greenov (WKB) pribli\v zek}
Od prej\v snjega predavanje nam ostane, da imamo za lastne funkcije v Sturm-Liouvilleovem problemu pribli\v zek
\[u_n(x) \sim \frac{1}{\sqrt[4]{wp}}\,e^{\pm i\int_0^x\sqrt{\lambda_n w/p}\dif y},\]
\[\lambda_n = \frac{n^2\pi^2}{(b-a)^2}D_{eff},\qquad D_{eff} = \left(\frac{1}{b-a}\int_a^b\sqrt{\frac{w}{p}}\dif y\right)^{-2}\]
Kon\v cna re\v sitev pri danih robnih pogojih je torej linearna kombinacija teh funkcij.
Poglejmo si \v se, kaj se zgodi, \v ce imamo Liouviellov problem oblike
\[\mathcal{L}u = fu'' + gu' + hu,\qquad f' \neq g\]
\v Ce pomno\v zimo s funkcijo, definirano pri prej\v snjem predavanju, torej
\[\frac{1}{f}\exp\left(\int\frac{g}{f}\dif y\right),\]
dobimo standardni Sturm-Liouviellov problem.
\paragraph{Primer.} Prevajanje toplote v nehomogeni kon\v cni palici. Imamo Dirichletov robni pogoj \(T_{rob} = 0\), podana je toplotna prevodnost \(\lambda(x) = p(x)\), in podatka \(\rho(x)\) in \(c_p(x)\), za katera velja \(\rho(x)c_p(x) = w(x)\) - \(w\) pa je na\v sa ute\v z.
Recimo, da imamo \[\lambda(x) = 1 - \frac{1}{1 + \cosh(A\left[x - \frac{1}{2}\right])}\]
\[\rho c_p = w(x) = 1 + \beta\left(x - \frac{1}{2}\right)\]
Tako dobimo Sturm-Liouvillov problem, ki ga lahko re\v sujemo npr. numeri\v cno ali z WKB pribli\v zkom in preverimo, da so lastne funkcije pri ute\v zenem skalarnem produktu tudi pri nehomogenem problemu ortogonalne (vsaj v okviru natan\v cnosti pribli\v zka in numeri\v cne natan\v cnosti).
\subsection{Splo\v sni povzetek}
Delamo na primeru difuzijske ena\v cbe \[D\nabla^2u = \pd{u}{t}\]
Naredimo nastavek za lastne funkcije
\[u(\vct{r}, t) = \sum_n c_n u_n(\vct{r}) T_n(t)\]
Za difuzijsko ena\v cbo dobimo na primer nastavek \[T_n(t) = e^{-k_n^2 Dt}\]
Za valovno ena\v cbo imamo nastavek \[T_n(t) = e^{i\omega_nt},\qquad \omega^2 = k_n^2c^2\]
Drugi korak je obravnava valovnega dela, pri katerem smo dobili Sturm-Liouviellov problem
\[\nabla^2 u_n + k^2_nu = 0\]
Poi\v s\v cemo lastne funkcije \(u_n(\vct{r})\) za homogene robne pogoje - zahtevamo \(\mathcal{L^\dag} = \mathcal{L}\).
Nato z razvojem po lastnih funkcijah re\v simo problem za poljubne za\v cetne pogoje \(f(\vct{r})\).
\[u(\vct{r}, t) = \sum_n c_nu_n(\vct{r})e^{-k_n^2Dt}\]
\[c_n = \frac{\avg{u_n, f}}{\avg{u_n, u_n}}\]
\v Ce imamo nehomogeno ena\v cbo \[D\nabla^2u = \pd{u}{t},\]
lahko razvijemo tudi nehomogenost \(g\):
\[u(\vct{r}, t) = \sum_n c_n(t)u_n(\vct{r})\]
\[g(\vct{r}, t) = \sum_n c_n(t)g_n(\vct{r})\]
Dobimo navadne diferencialne ena\v cbo za \(c_n\):
\[\dd{}{t}c_n = -k_n^2Dt c_n + g_n\]
ki imajo znane re\v sitve
\[c_n(t) = c_n(0)e^{-k_n^2 Dt} + \int_0^t e^{-k_n^2 Dt\tau}g_n(\tau)\dif\tau\]
\v Ce imamo nehomogene robne pogoje, problem prevedemo na re\v sevanje nehomogene diferencialne ena\v cbe s homogenimi robnimi pogoji (\(u = v + g\)).
\section{Cilindri\v cni koordinatni sistem}
\subsection{Besselove funkcije}
Iz Keplerjevega problema imamo ena\v cbo 
\[\theta = \psi - \varepsilon\sin\psi\]
ki opisuj obliko obrite. \v Zeleli bi poiskati izraz za \(\psi\), vendar gre za transcendentno ena\v cbo, zato tega ne moremo storiti na analiti\v cen na\v cin.
Besselove funkcije imajo ime po Priedrichu Besselu, ki je z njmi izrazil \(\psi\), in sicer:
\[\psi = \theta + \sum_{n=1}^{\infty}\frac{2}{n}J_n(n\varepsilon)\sin(n\theta)\]
V osnovi pa gre za re\v sitve diferencialne ena\v cbe \[x^2y'' + xy' + (x^2 - \nu^2)y = 0\]
\subsection{ Helmholtzova ena\v cba}
\[\nabla^2 u + \lambda u = 0\]
Namesto v kartezi\v cnih koordinatah re\v sujemo ena\v cbo v cilindri\v cnih:
\[u = R(r)Z(z)\phi(\varphi)\]
Najprej re\v sujemo kotni del:
\[\frac{\phi''}{\phi} = -m^2 \rightarrow \phi = \left\{\begin{matrix}
    \cos\varphi \\ \sin\varphi
\end{matrix}\right\}\]
ali v posebnem primeru \(m = 0\):
\[\phi = \left\{\begin{matrix}
    1 \\ \varphi
\end{matrix}\right\}\]
Vzdol\v z osi \(z\) imamo dve mo\v znosti, odvisni od robnih pogojev:
\[\frac{Z''}{Z} = \begin{cases}
    + \beta^2, & \text{ re\v sitev je } \left\{\begin{matrix}
        \cosh(\beta z) \\ \sinh(\beta z)
    \end{matrix}\right\} \\
    - \beta^2, & \text{re\v sitev je } \left\{\begin{matrix}
        \cos(\beta z) \\ \sin(\beta z)
    \end{matrix}\right\}
\end{cases}\]
Ena\v cba za radij je
\[R'' + \frac{1}{r}R' + \left[\pm \beta^2 R - \frac{m^2}{r^2}R\right] + \lambda R\]
To prepi\v semo v
\[r^2R'' + rR' + \left[r^2\left(\lambda \pm \beta^2 - m^2\right)R = 0\right]\]
Dobili smo Besselovo diferencialno ena\v cbo. Re\v sitve so cilindri\v cne Besselove funkcije. Poglejmo si nekaj primerov:
\begin{itemize}
\item \(\lambda \pm \beta^2 = 0\): Re\v sitve so linearna kombinacija \(r^m\) in \(r^{-m}\), za \(m = 0\) pa \(1 + \ln(r)\).
\item \(\lambda \pm \beta^2 > 0\): definiramo \(k^2 = \lambda \pm \beta^2\) in dobimo re\v sitev kot linearno kombinacijo \(J_m(kr)\) in \(Y_m(kr)\) (\(Y_m\) v\v casih imenujemo Neumannova funkcija). Lahko omenimo tudi limite
\[J_\nu \sim x^\nu,\qquad x \ll 1\]
\[Y_\nu \sim x^{-\nu},\qquad x \ll 1\]
\[Y_0 \sim \ln x, \qquad x \ll 1\]
V limiti \(x \gg 1\) pa velja
\[J_\nu \asymp \sqrt{\frac{2}{\pi x}}\cos\left(x - \frac{1}{2}\nu\pi - \frac{\pi}{4}\right)\]
\[Y_\nu \asymp \sqrt{\frac{2}{\pi x}}\sin\left(x - \frac{1}{2}\nu\pi - \frac{\pi}{4}\right)\]
Gre v bistvu za cilindri\v cne stoje\v ce valove. Omenimo tudi Hanklovi funkciji:
\[H_\nu^{(1)} = J_\nu + iY_\nu\]
\[H_\nu^{(2)} = J_\nu - iY_\nu\]
\item \(\lambda \pm \beta^2 < 0\): Definiramo \(k^2 = - \left(\lambda + \beta^2\right)\), da je \(k\) lahko realno \v stevilo. Nato izrazimo re\v sitev z modificiranimi Besselovimi funkcijami \(I_\nu(kr)\) in \(K_\nu(kr)\). V limiti \(kr \ll 1\) se obna\v sata podobno kot navadni Besselovi funkciji (z izjemo \(K_0 \sim -\ln kr, ~ kr \ll 1\)), v limiti \(kr \gg 1\) pa velja
\[I_\nu \sim e^x/\sqrt{x}\]
\[K_\nu \sim e^{-x}/\sqrt{x}\]
Modificirane Besselove funkcije so torej eksponentne narave in nimajo ni\v cel (razen nekatere pri \(x = 0\)), kakor imajo navadne Besselove funkcije oscilirajo\v co naravo.
\end{itemize}
\paragraph{Primer.} Re\v sujemo ena\v cbo \(\nabla^2u + \lambda u = 0\) za valj z radijem \(a\), vi\v sino \(H\). Zahtevamo robni pogojem \(u = 0\).
V smeri \(\varphi\) dobimo \[\phi(\varphi) = e^{im\varphi},~~~m = 0, 1, 2, ...\]
V smeri \(z\) izberemo sinusne funkcije, saj imamo robni ogoj \(u = 0\). Med opisanimi funkcijami ima le \(\sin\) preiodi\v cne ni\v cle.
V smeri \(r\) vemo, da bo \v slo za Besselove funkcije, saj mora imeti re\v sitev periodi\v cne ni\v cle (to zahteva robni pogoj). Re\v sitev je torej
\[R(r) = J_m(kr),\]
kjer mora zaradi robnega pogoja veljati
\[k = \frac{\xi_{m,p}}{a},\]
kjer \(\xi_{m, p}\) pomeni \(p\)-to ni\v clo \(m\)-te Besselove funkcije (te os tabelirane).
Re\v sitev originalnega problema je linearna kombinacija lastnih funkcij
\[e^{im\varphi}\sin\left(\frac{m\pi z}{H}\right)\,J_m\left(\frac{\xi_{m, p}r}{a}\right)\]
z lastnimi vrednostmi
\[\lambda_{n, m, p} = \left(\frac{n\pi}{H}\right)^2 + \left(\frac{\xi_{m, p}}{a}\right)^2\]
\subsection{Lastnosti Besselovih funkcij}
\paragraph{Generatrisa.} Besselove funkcije uporabimo kot koeficiente razvoja
\[\exp\left[\frac{x}{2}\left(t - \frac{1}{t}\right)\right] = \sum_{n = -\infty}^{\infty} t^n J_n(x)\]
Primer:
\[e^{ix\sin\theta} = \sum_{n = -\infty}^{\infty} e^{in\theta}J_n(x)\]
\[e^{ix\cos\theta} = J_0(x) + 2\sum_{n = -\infty}^{\infty} i^n J_n(x) \cos(n\theta)\]
\paragraph{Integralska reprezentacija.}
\[J_n(x) = \frac{1}{2\pi}\int_0^{2\pi}e^{-i\left(n\theta - x\sin\theta\right)}\dif\theta\]
\paragraph{Rekurzivne zveze.}
\begin{align*}
    2J_n' & = J_{n-1} - J_{n+1}, \quad J_0' = -J_1 \\
    \frac{2n}{x}J_n & = J_{n-1} + J_{n+1} \\
    \left(x^nJ_n\right)' = x^n J_(n-1) \\[2mm]
    J_n(x) = (-1)^n J_n(-x) \\
    J_0^2(x) + 2\sum_{n = 1}^{\infty} J_n^2(x) = 1
\end{align*}
\paragraph{Adicijski izreki.}
\begin{align*}
    J_n(x + y) & = \sum_{n = -\infty}^{\infty} J_r(x) J_{n-r}(y) \\
    J_0(|\vct{r_1} - \vct{r_2}|) & = \sum_{n = -\infty}^{\infty} J_n(\vct{r_1})J_n(r_2)e^{in(\varphi_2 - \varphi_1)}
\end{align*}
Podobno zvezo imamo tudi za Hanklove funkcije:
\[H_0(|\vct{r_1} - \vct{r_2}|) = \sum_{n = -\infty}^{\infty} J_n(\vct{r_1})H_n(r_2)e^{in(\varphi_2 - \varphi_1)},\text{ za } r_1 > r_2\]
\paragraph{Normalizacija.} V\v casih nam pride prav Besselove funkcije normirati. Za \(J_\nu(\xi_{\nu, \mu}) = 0\) zapi\v semo normo
\[\int_{0}^{R} r\left[J_\nu\left(\xi_{\nu, m}\frac{r}{R}\right)\right]^2\dif r = \frac{R^2}{2}\left[J_\nu'(\xi_{\nu, m})\right]^2 = \frac{R^2}{2}\left[J_{\nu + 1}(\xi_{\nu, m})\right]^2\]
Za \(J_\nu'(\xi'_{\nu, m}) = 0\):
\[\int_0^R r\left[J_\nu\left(\xi_{\nu, m}\frac{r}{R}\right)\right]^2\dif r = \frac{R^2}{2}\left(1 - \frac{\nu^2}{{\xi'}_{\nu, m}^2}\right)\left[J_\nu(\xi'_{\nu, m})\right]^2\]
\end{document}