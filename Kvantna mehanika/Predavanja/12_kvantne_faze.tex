\documentclass[a4paper]{article}
\usepackage{amsmath, amssymb, amsfonts}
\usepackage[margin=1in]{geometry}
\usepackage{graphicx}
\usepackage{tikz}
\usepackage{esint}
\setlength{\parindent}{0em}
\setlength{\parskip}{1ex}

\newcommand{\vct}[1]{\overrightarrow{#1}}
\newcommand{\dif}{\,\mathrm{d}}
\newcommand{\pd}[2]{\frac{\partial {#1}}{\partial {#2}}}
\newcommand{\dd}[2]{\frac{\mathrm{d} {#1}}{\mathrm{d} {#2}}}
\newcommand{\C}{\mathbb{C}}
\newcommand{\R}{\mathbb{R}}
\newcommand{\Q}{\mathbb{Q}}
\newcommand{\Z}{\mathbb{Z}}
\newcommand{\N}{\mathbb{N}}
\newcommand{\fn}[3]{{#1}\colon {#2} \rightarrow {#3}}
\newcommand{\avg}[1]{\left\langle {#1} \right\rangle}
\newcommand{\Sum}[2][0]{\sum_{{#2} = {#1}}^{\infty}}
\newcommand{\Lim}[1]{\lim_{{#1} \rightarrow \infty}}
\newcommand{\Binom}[2]{\begin{pmatrix} {#1} \cr {#2} \end{pmatrix}}
\newcommand{\duline}[1]{\underline{\underline{#1}}}
\newcommand{\bra}[1]{\left\langle {#1} \right|}
\newcommand{\ket}[1]{\left| {#1} \right\rangle}
\newcommand{\rot}{\vct{\nabla}\times}
\newcommand{\dvg}{\vct{\nabla}\cdot}
\renewcommand{\figurename}{Slika}

\begin{document}
Verjetnost prehoda med dvema stanjema kot rezultat perturbacije opi\v semo z matriko:
\[P_{km}(t) = \frac{|V_{km}|^2}{\hbar^2} \pi\delta_t\left(\frac{1}{2\hbar}(E_k - E_m)\right)\,t\]
Kar bi lahko napisali kot \(\bra{k}V\ket{m}\). Ko gre \(t\) proti \(\infty\), gre razlika energij proti \(0\). \\
Verjetnost, da se zgodi kar koli, je vsota vseh teh verjetnosti, torej:
\[P = \sum_{k \neq m}\P_{km}\]
To aproksimiramo z integralom:
\[P = \sum_m \int_{-\infty}^{\infty} P_{km}(t)\rho(E_k)\dif E_k\]
Ko gre \(t \to \infty\):
\[P \to \sum_m \int_{-\infty}^{\infty} \frac{|V_{km}|^2}{\hbar^2} 2\pi\hbar\delta(E_k - E_m)\rho(E_k)\dif E_k\]
\paragraph{Fermijevo zlato pravilo.} Imamo kon\v cna stanja, gostoto energij aproksimiramo z odvodom:
\[\rho_k = \dd{N_k}{E_k}\frac{1}{N_k} = \dd{P}{E_k}\]
\[P = \frac{2\pi}{\hbar}|V_{km}|^2\rho(E_m)t\]
Ferijevo zlato pravilo pa je odvod tega:
\[w = \dd{P}{t} = \frac{2\pi}{\hbar}|V_{km}|^2\rho(E_m)\]
Nekaj lastnosti matrike \(P\):
\begin{itemize}
    \item \(P_{mm} = |c_m|^2 \approx 1 - \lambda^2\)
    \item \(P_{km} \ll 1\)
    \item \(V_{km}\) ni zelo odvisno od \(E_k\)
\end{itemize}
Primer: Radioaktivni razpad: Imamo \(N\) delcev; velja
\[w = \dd{P}{t} = \text{konst.}\]
\[\dif N = -N\dif P = -Nw\dif t\]
Ta izra\v cun nam omogo\v ca, da hitro izra\v cunamo \(N(t)\) kot
\[N(t) = N_0 e^{-wt}\]
\paragraph{Adiabatne spremembe in kvantne faze.} Imejmo potencialno jamo \v sirine \(L\), ki jo raztegnemo (\(L \to L(t)\)).
Ne raztegujemo prehitro - kajti hipne spremembe niso adiabatne - temve\v c dovolj po\v casi,
da se ima verjetnostna gostota \v cas razporediti po prostoru.
Mislimo si, da je Hamiltonian odvisen od nekih parametrov, ki so odvisni od \v casa:
\[\vct{Q} = (q_1, q_2, ... q_n) = (\text{na primer})~(L(t), V_0(t), ...)\]
Navino bi to vstavili v Schr\"odingerjevo ena\v cbo:
\[H(\vct{Q})\ket{\psi_n(\vct{Q})} = E_n(\vct{Q})\ket{\psi_n(\vct{Q})}\]
Kajti tedaj bi imeli \v casovni razvoj
\[\psi_n^0(\vct{r}, t) = \avg{\vct{r}|\psi_n(\vct{Q}(t))} \left(= A\sin(k(t)r)e^{-i\hbar^2k^2(t)t/\hbar}\right)\]
in posledico
\[i\hbar\pd{}{t}\ket{\psi_n^0(\vct{r}, t)} = H(\vct{Q})\ket{\psi_n^0(\vct{r}, t)}\]
To pa v splo\v snem ne velja. Poseben primer imamo, ko se \(\vct{Q}\) s \v casom dovolj po\v casi spreminja.
Kaj je mi\v sljeno z "dovolj po\v casi", ni \v cisto jasno definirano, v primeru \v sirjenja potencialne jame
imamo obi\v cajno zahtevo
\[\frac{1}{L}\dd{L}{t} \ll \frac{\Delta E_n}{\hbar}\]
Tedaj imamo nastavek
\[\psi_n(\vct{r}, t) = e^{i\phi_n(t)}\psi^0_n(\vct{r}, t) = e^{i\phi_n(t)}\avg{\vct{r}|\psi_n(\vct{Q}(t))}\]
\(\psi^0_n\) je lastna funkcija, ki je lastna funkcija v tistem trenutku. Lahko predpostavimo, da so normirane.
S tem nastavkom gremo v Schr\"odingerjevo ena\v cbo in i\v s\v cemo \(\phi_n\).
\[i\hbar\left(i\dd{\phi_n}{t}e^{i\phi_n}\ket{\psi_n^0} + e^{i\phi_n}\pd{}{t}\ket{\psi_n^0}\right) = E_n e^{i\phi_n} \ket{\psi_n^0}\]
Pokraj\v samo \(\ket{\psi_n^0}\) in na obeh straneh skalarno mno\v zimo s \(\bra{\psi_n^0}\):
\[i\hbar\left(i\dd{\phi_n}{t}\avg{\psi_n^0|\psi_n^0} + \avg{\psi_n^0|\pd{}{t}\psi_n^0}\right) = E_n\avg{\psi_n^0|\psi_n^0}\]
Ostane:
\[i\hbar\left(i\dd{\phi_n}{t} + \avg{\psi_n^0|\pd{}{t}\psi_n^0}\right) = E_n\]
Nastavili bomo \(\phi_n =\gamma_n + \theta_n\). Lastnosti teh funkcij bomo zahtevali pozneje.
\[i\hbar\left(i\dd{\gamma}{t} + \avg{\psi_n^0|\pd{}{t}\psi_n^0}\right) = E_n + \hbar\dd{\theta_n}{t}\]
Lahko nastavimo tak \(\theta_n\), da bo izraz na desni enak \(0\):
\[\dd{\theta_n}{t} = -\frac{E_n(t)}{\hbar}\]
\[\theta_n = -\frac{1}{\hbar}\int_0^t E_n(t')\dif t'\]
Zdaj moramo posikati \v se \(\gamma_n\). Ker bomo ra\v cunali odvod \(\psi_n^0\),
bomo tako kot pri klasi\v cni mehaniki uporabili koordinate, ki nam najbolj ustrezajo: \(q_i\),
torej komponente vektorja \(\vct{Q}\).
\[\pd{}{t}\psi_n^0 = \sum_i\left(\pd{\psi_n^0}{q_i}\right)\dot{q_i} = \nabla_{\vct{Q}}\psi_n^0 \dot{\vct{Q}}\]
Od Schr\"odingerjeve ena\v cbe nam je ostalo
\[\dd{\gamma_n}{t} = i\avg{\psi_n^0|\pd{}{t}\psi_n^0}\]
Vstavimo odvod \(\psi_n^0\) in integriramo:
\[\gamma_n(t) = \int_{0}^{t} i\avg{\psi_n^0|\nabla_{\vct{Q}}\psi_n^0}\dot{\vct{Q}}(t')\dif t'\]
\v Ce si zamislimo, da vektor \(\vct{Q}\) prepotuje pot po faznem prostoru, lahko pi\v semo tudi:
\[\gamma_n(t) = \int_{\vct{Q}(0)}^{\vct{Q}(t)} i\avg{\psi_n^0|\nabla_{\vct{Q}}\psi_n^0}\dif\vct{Q}\]
Oglejmo si kon\v cni rezultat \(\phi_n\):
\[\phi_n = -\frac{1}{\hbar}\int_0^t E_n(t')\dif t' + \gamma_n(t) = \int_{\vct{Q}(0)}^{\vct{Q}(t)} i\avg{\psi_n^0|\nabla_{\vct{Q}}\psi_n^0}\dif\vct{Q}\]
Prvemu \v clenu (prej \(\theta_n\)) pravimo dinami\v cna faza, saj gre za integral po \v casu. To smo v bistvu imeli \v ze prej.
Drugi \v clen (prej \(\gamma_n\)) je geometrijska faza, saj integriramo po poti v faznem prostoru. \\
\paragraph{Semiklasi\v cni pribli\v zek.} Imenovan tudi metoda WKB (Entzel-Kramers-Brillouin). Metoda je uporabna, ko obravnavamo te\v zke delce, na primer ione.
Za\v cnemo z nastavkom
\[\psi = e^{iS(\vct{r}, t)/\hbar}\]
Ko to vstavimo v Schr\"odingerjevo ena\v cbo, dobimo:
\[-\pd{S}{t}\psi = \left(\frac{1}{2m}(\nabla S)^2 - i\frac{\hbar}{2m}\nabla^2 S\right)\psi\]
Prvi \v clen, \(-\pd{S}{t}\psi = \frac{1}{2m}(\nabla S)^2\), v klasi\v cni mehaniki velja za Hamilton-Jacobijevo ena\v cbo, re\v sitev katere je \v casovni integral Lagrangeove funkcije. Ima lastnost \(\nabla S = \vct{v}\).
Ker imamo trudi drugi \v clen, problem re\v sujemo druga\v ce. Pristop bo malo nenavaden: \(S\) razvijemo po potencah \(\hbar\):
\[S = S_0 + \hbar S_1 + \hbar^2 S_2 + ...\]
\[\hbar^0:~~~-\pd{S_0}{t} = \frac{1}{2m}\left((\nabla S_0)^2 + V(\vct{r}, t)\right)\]
\[\hbar^1:~~~-\pd{S_0}{t} = \frac{1}{2m}\left(2\nabla S_0 \cdot \nabla S_1 - i\nabla^2 S_0\right)\]
Z nadaljnjimi redi se ne bomo ukvarjali.
Za primer si vzemimo stacionarno stanje v eni dimenziji:
\[\Psi(x, t) = e^{-i\frac{Et}{\hbar}}e^{i\frac{S(t)}{\hbar}\hbar}\]
Velja: \[-i\frac{Et}{\hbar} = i\frac{S_0}{\hbar}\]
\[\hbar^0:~~~E = \frac{1}{2m}\left(\dd{S_0}{x}\right)^2 + V(x)\]
\[\dd{S_0}{x} = \pm \sqrt{(E-V(x))2m}\]
Koren je ravno enak gibalni koli\v cini v odvisnosti od kraja, se pravi:
\[S_0(x) = \pm\int_{x_0}^{x}p(x')\dif x'\]

Se pravi v semiklasi\v cnem prigli\v zku dobimo velovno funkcijo:
\[\Psi_{WKB}(x, t) = \frac{C}{\sqrt{p(x)}}e^{i\frac{Et}{\hbar}}e^{\pm\frac{i}{\hbar}\int_{x_0}^{x}p(x')\dif x'}\]
Mimogrede:
\[\hbar^1:~~~0 = \frac{1}{2m}\left(2\dd{S_1}{x}\dd{S_0}{x} -i\dd{^2S_0}{x}\right)\]
Vstavimo \(S_0\) in dobimo
\[S_1(x) = i\ln\left(\frac{p(x)}{p(x_0)}\right)\]
Ta pribli\v zek je dober, ko se potencial relativno po\v casi spreminja (npr. pri razpadu \(\alpha\)), ali pa ko imamo opravka s te\v zkimi ioni.
\end{document}