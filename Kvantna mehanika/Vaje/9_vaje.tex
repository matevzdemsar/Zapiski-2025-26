\documentclass[a4paper]{article}
\usepackage{amsmath, amssymb, amsfonts}
\usepackage[margin=1in]{geometry}
\usepackage{graphicx}
\usepackage{tikz}
\usepackage{esint}
\setlength{\parindent}{0em}
\setlength{\parskip}{1ex}

\newcommand{\vct}[1]{\overrightarrow{#1}}
\newcommand{\dif}{\,\mathrm{d}}
\newcommand{\pd}[2]{\frac{\partial {#1}}{\partial {#2}}}
\newcommand{\dd}[2]{\frac{\mathrm{d} {#1}}{\mathrm{d} {#2}}}
\newcommand{\C}{\mathbb{C}}
\newcommand{\R}{\mathbb{R}}
\newcommand{\Q}{\mathbb{Q}}
\newcommand{\Z}{\mathbb{Z}}
\newcommand{\N}{\mathbb{N}}
\newcommand{\fn}[3]{{#1}\colon {#2} \rightarrow {#3}}
\newcommand{\avg}[1]{\left\langle {#1} \right\rangle}
\newcommand{\Sum}[2][0]{\sum_{{#2} = {#1}}^{\infty}}
\newcommand{\Lim}[1]{\lim_{{#1} \rightarrow \infty}}
\newcommand{\Binom}[2]{\begin{pmatrix} {#1} \cr {#2} \end{pmatrix}}
\newcommand{\duline}[1]{\underline{\underline{#1}}}
\newcommand{\bra}[1]{\left\langle {#1} \right|}
\newcommand{\ket}[1]{\left| {#1} \right\rangle}
\renewcommand{\figurename}{Slika}

\begin{document}
\paragraph{Delec v magnetnem polju.} Imamo zunanje magnetno polje \(\vct{B}\), velja:
\[l = 1\]
\[H = -\lambda\vct{L}\cdot\vct{B}\]
\[(\vct{L}\cdot\widehat{n})\ket{\psi, 0} = \hbar\ket{\psi, 0}\]
Poznamo tudi \v casovni razvoj \(\psi\), in sicer:
\[\ket{\psi, t} = \sin^2\frac{\vartheta}{2}e^{-i\lambda Bt}\ket{1,\,\text{-1}} + \sqrt{2}\sin\frac{\vartheta}{2}\cos\frac{\vartheta}{2}\ket{1,\,0} + \cos^2\frac{\vartheta}{2}e^{i\lambda Bt}\ket{1,\,1}\]
Poglejmo razvoj \(\avg{L_z, t}\):
\[\bra{\psi, t}L_z\ket{\psi, t} = \bra{\psi, t}\left(-\hbar\sin^2\frac{\vartheta}{2}e^{-i\lambda Bt}\ket{1,\,\text{-1}} + \hbar\cos^2\frac{\vartheta}{2}\ket{1,\,1}\right)\]
\[= \sin^2\frac{\vartheta}{2}e^{-\lambda Bt}\bra{1,\,\text{-1}}-\hbar\sin^2\frac{\vartheta}{2} e^{-i\lambda Bt}\ket{1,\,\text{-1}} + \cos^2\frac{\vartheta}{2}e^{-i\lambda Bt}\bra{1,\,1} \cos^2\frac{\vartheta}{2}e^{-i\lambda Bt} \ket{1,\,1}\]
\[= - \hbar\sin^4\frac{\vartheta}{2} + \hbar\cos^4\frac{\vartheta}{2} = \hbar\left(\cos^2\frac{\vartheta}{2} + \sin^2\frac{\vartheta}{2}\right)\left(\cos^2\frac{\vartheta}{2} - \sin^2\frac{\vartheta}{2}\right) = \hbar\cos\vartheta\]
Zdaj izra\v cunajmo \v casovni razvoj \(L_x\) in \(L_y\):
\[\avg{L_x, t} = \bra{\psi, t} \frac{L_+ + L_-}{2} \ket{\psi, t} = \frac{1}{2} \left(\avg{L_+, t} + \avg{L_-, t}\right) = \frac{1}{2}\left(\avg{L_+, t} + \avg{L_+^\dag, t}\right) = \mathfrak{Re}\left(\avg{L_+, t}\right)\]
\[\avg{L_y, t} = \bra{\psi, t} \frac{L_+ - L_-}{2i} \ket{\psi, t} = \frac{1}{2i} \left(\avg{L_+, t} - \avg{L_-, t}\right) = \frac{1}{2i}\left(\avg{L_+, t} + \avg{L_+^\dag, t}\right) = \mathfrak{Im}\left(\avg{L_+, t}\right)\]
Tako moramo izra\v cunati samo \(\avg{L_+, t}\):
\[\bra{\psi, t}L_+\ket{\psi, t} = ?\]
Vemo:
\[L_+\ket{1,\,\text{-1}} = \sqrt{2}\hbar\ket{1,\,0}\]
\[L_+\ket{1,\,0} = \sqrt{2}\hbar\ket{1,\,1}\]
\[L_+\ket{1,\,1} = 0\]
To vstavimo v skalarni produkt \(\bra{\psi, t}L_+\ket{\psi_t}\) (vsega nimam \v casa prepisati) in dobimo
\[\avg{L_+, t} = \hbar\sin\vartheta\cos(\lambda Bt) - i\sin\vartheta\sin(\lambda Bt)\]
Sledi:
\begin{align*}
    \avg{L_z, t} & = \hbar\cos\vartheta \\
    \avg{L_x, t} & = \hbar\sin\vartheta\cos(\lambda Bt) \\
    \avg{L_y, t} & = \hbar\sin\vartheta\cos(\lambda Bt) \\
\end{align*}
Opisanemu gibanju re\v cemo Larmorjeva precesija.
\paragraph{Opomba.} Ko izmerimo \(L_z\), dobimo eno od mo\v znih lastnih vrednosti,
pri naslednjih meritvah pa dobimo le vrednost, v katero je kolapsiralo stanje pri prvotni meritvi.
Izra\v cunamo lahko tudi verjetnost za posamezno stanje.
\[\begin{array}{r|l|l}
    L_z & p = |c_m|^2 & \psi\text{ ob }t + \dif t \\[2mm]
    \hline
    \hbar & \cos^4\frac{\vartheta}{2} & \ket{1,\,1} \\
    0 & 2\cos^2\frac{\vartheta}{2} & \ket{1,\,0} \\
    - \hbar & \cos^4\frac{\vartheta}{2} & \ket{1,\,-1}
\end{array}\]

\end{document}