\documentclass[a4paper]{article}
\usepackage{amsmath, amssymb, amsfonts}
\usepackage[margin=1in]{geometry}
\usepackage{graphicx}
\usepackage{tikz}
\usepackage{esint}
\setlength{\parindent}{0em}
\setlength{\parskip}{1ex}

\newcommand{\vct}[1]{\overrightarrow{#1}}
\newcommand{\dif}{\,\mathrm{d}}
\newcommand{\pd}[2]{\frac{\partial {#1}}{\partial {#2}}}
\newcommand{\dd}[2]{\frac{\mathrm{d} {#1}}{\mathrm{d} {#2}}}
\newcommand{\C}{\mathbb{C}}
\newcommand{\R}{\mathbb{R}}
\newcommand{\Q}{\mathbb{Q}}
\newcommand{\Z}{\mathbb{Z}}
\newcommand{\N}{\mathbb{N}}
\newcommand{\fn}[3]{{#1}\colon {#2} \rightarrow {#3}}
\newcommand{\avg}[1]{\langle {#1} \rangle}
\newcommand{\Sum}[2][0]{\sum_{{#2} = {#1}}^{\infty}}
\newcommand{\Lim}[1]{\lim_{{#1} \rightarrow \infty}}
\newcommand{\Binom}[2]{\begin{pmatrix} {#1} \cr {#2} \end{pmatrix}}
\newcommand{\duline}[1]{\underline{\underline{#1}}}
\newcommand{\bra}[1]{\langle {#1} |}
\newcommand{\ket}[1]{| {#1} \rangle}
\renewcommand{\figurename}{Slika}

\begin{document}
\paragraph{Magnetna indukcija v kvadratnem okvirju - 2. del.}
Imamo dolga vodnika na medsebojni razdalji \(d\), ki sta nekje dale\v stran povezana - tvorita torej zunanjo zanko. Znotraj te zanke je \v se ena kvadratna zanka.
\begin{figure}[h!]
    \centering
    \begin{tikzpicture}[scale=0.8]
    \draw (-5, 2) -- (5, 2);
    \draw[dashed] (5, 2) -- (5, 0);
    \draw[dashed] (-5, 2) -- (-5, 0);
    \draw (-5, 0) -- (5, 0);
    \draw (0, 0) -- (1, 1) -- (0, 2) -- (-1, 1) -- (0, 0);
    \end{tikzpicture}
\end{figure}
\[\phi_1 = L_{11} I_1\]
\[\phi_2 = L_{21} I_1\]
Zadnji\v c smo izra\v cunali medsebojno induktivnost kot
\[L_{12} = L_{21} = \frac{2\ln 2}{\pi}\,\mu_0d\]
Zdaj imejmo spreminjajo\v c se tok:
\[I_1 = I_{1,z}\sin\omega t\]
Vemo, da za vsak tokovni krog velja:
\[\dot U = R \dot I + L \ddot I + \frac{I}{C}\]
Ker ra\v cunamo, da so \v zice idealno prevodne, kondenzatorjev pa ni, velja:
\[U_{2} = -\dot\phi_2 = -L_{21}\dot I_1 = L_{22} \dot I_2\]
Ker je \(I_1\) sinusna funkcija, pri\v cakujemo, da je tudi \(I_2\) sinusna funkcija. Sledi:
\[\omega L_{21}I_{10}\cos\omega t = -\omega L_{22} I_{20} \cos\omega t\]
\[\frac{I_{20}}{I_{10}} = \frac{L_{21}}{L_{22}}\]
Izra\v cun \(L_{22}\):
\[B_2 = \frac{\mu_0I_2}{2\pi y} + \frac{\mu_0I_2}{2\pi(d-y)}\]
\[\phi_2 = \int B \dif S = \frac{\mu_0 I_2 l}{2\pi}\int_{a}^{d-a}\left(\frac{1}{y} + \frac{1}{d-y}\right)\dif y\]
\[\phi_2 = \frac{\mu_0 I_2 l}{2\pi}\left(\ln\frac{d-a}{a} - \ln\frac{a}{d-a}\right) = \frac{\mu_0 l}{\pi}\ln\frac{d-a}{a}\,I_2 = L_{22}I_2\]
Pri predpostavki, da je \(d \gg a\), ocenimo:
\[L_{22} \approx \frac{\mu_0 l}{\pi}\ln\frac{d}{a}\]
\[\frac{I_{20}}{I_{10}} = -\frac{2\ln 2}{\frac{l}{d}\ln\frac{d}{a}}\]
\paragraph{Kro\v zni pojav v prevodnem traku.} Skozi prevoden trak po\v zenemo izmeni\v cen sinusen tok.
Pri\v cakujemo, da se bo tok po traku spreminjal po \v sirini traka (saj se naboj obi\v cajno nabere na robovih telesa).
Poznamo \(\omega\), debelino traku \(a\) in povr\v sinsko gostota naboja \(\sigma\).
Zzanima nas:
\begin{itemize}
    \item \(E_z(x)\), \(j(x)\)
    \item \(Z(\omega)\) (impedanca), \(Z/R\)
\end{itemize}
Ozna\v cimo \(U = U_0e^{i\omega t}\). Za re\v sevanje bomo uporabljali Maxwellovi ena\v cbi:
\[\nabla\times\vct{E} = -\pd{\vct{B}}{t}\]
\[\nabla\times\vct{B} = \mu_0\vct{j} + \mu_0\varepsilon_0\pd{\vct{E}}{t}\]
Uporabili bomo kvazistati\v cno aproksimacijo, ki pravi:
\[\mu_0\varepsilon_0\pd{\vct{E}}{e} \approx 0\]
Sledi:
\[\nabla\times(\nabla\times\vct{E}) = -\pd{}{t}\nabla\times\vct{B} = -\pd{}{t}\mu_0\sigma\vct{E}\]
Hkrati je to enako \[\nabla\cdot(\nabla\cdot\vct{E}) - \nabla^2\vct{E}\]
Pri \v cemer je \(\nabla\cdot(\nabla\cdot\vct{E}) = 0\)
\[\nabla^2\vct{E} - \mu_0\sigma\pd{\vct{E}}{t} = 0\]
Ker je zaradi sinusne napetosti \(\vct{E} \propto \exp(i\omega t)\), lahko zapi\v semo:
\[\nabla^2\vct{E} -i\omega\mu_0\sigma\vct{E} = 0\]
Ozna\v cimo \(\kappa^2 = i\omega\mu_0\sigma\) - tedaj je \(\kappa = \frac{1}{\sqrt{2}}(1 + i)\sqrt{\omega\mu_0\sigma}\).
\[\nabla^2\vct{E} -\kappa^2\vct{E} = 0\]
Obravnavamo samo \(z\) komponento, za katero vemo, da je konstantna v smeri \(z\) in neodvisna od \(y\). Sledi:
\[\pd{^2}{x^2}E_z - \kappa^2E_z = 0\]
Z robnim pogojem \(E_z(x = \pm a/2) = E_0\)
Tovrstno ena\v cbo re\v sijo kotne ali hiperboli\v cne funkcije. Vzemimo torej nastavek
\[E_z = A\cosh(\kappa x)\]
Funkcije \(\sinh\) ne uporabimo, saj je robni pogoj simetri\v cen. Ko vstavimo v robni pogoj, dobimo:
\[E_0 = A\cosh\frac{\kappa a}{2}\]
\[E_z(x) = \frac{E_0}{\cosh\frac{\kappa a}{2}}\cosh(\kappa x)\]
Ker je \(\kappa\) kompleksen, je to v resnici nekak\v sen produkt funkcij \(\cos\) in \(\cosh\).
\v Se vedno pa ta ena\v cba velja (le v hiperboli\v cni kosinus moramo vstavljati kompleksna \v stevila).
\[\j_0 = \sigma E_{z, 0}\]
Izra\v cun impedance:
\[U_0 = E_0l,\qquad I_0 = \int j_0 \dif S\]
\[I_0 = \sigma \frac{E_0}{\cosh\frac{\kappa a}{2}}\int_{-a/2}^{a/2} \cosh \kappa x b \dif x,\]
kjer smo z \(b\) ozna\v cili vi\v sino traku (\(y\) smer).
\[I_0 = \frac{2\sigma E_0 b}{\kappa}\tanh\frac{\kappa a}{2}\]
\[Z = \frac{U_0}{I_0} \frac{l\kappa a}{a 2\sigma b \tanh\frac{\kappa a}{2}}\]
Vemo, da je \[\frac{l}{\sigma a b} = \frac{\zeta l}{S} = R\]
\[Z = R\,\frac{(\kappa a)/2}{\tanh(\kappa a/2)}\]
Kotno hitrost \(\omega\) smo uporabili v spremenljivki \(\kappa\): \(\kappa \propto \sqrt{\omega}\). \\[2mm]
Poglejmo si robna pogoja majhnih in velikih \(\omega\):
\[\omega \ll 1: \tanh\frac{\kappa a}{2} \approx \frac{\kappa a}{2} \Rightarrow z \approx R\]
\[\omega \gg 1: \tanh\frac{\kappa a}{2} \approx 1 \Rightarrow Z = R\frac{\kappa a}{2}\]
\end{document}