\documentclass[a4paper]{article}
\usepackage{amsmath, amssymb, amsfonts}
\usepackage[margin=1in]{geometry}
\usepackage{graphicx}
\usepackage{tikz}
\usepackage{esint}
\setlength{\parindent}{0em}
\setlength{\parskip}{1ex}

\newcommand{\vct}[1]{\overrightarrow{#1}}
\newcommand{\dif}{\,\mathrm{d}}
\newcommand{\pd}[2]{\frac{\partial {#1}}{\partial {#2}}}
\newcommand{\dd}[2]{\frac{\mathrm{d} {#1}}{\mathrm{d} {#2}}}
\newcommand{\C}{\mathbb{C}}
\newcommand{\R}{\mathbb{R}}
\newcommand{\Q}{\mathbb{Q}}
\newcommand{\Z}{\mathbb{Z}}
\newcommand{\N}{\mathbb{N}}
\newcommand{\fn}[3]{{#1}\colon {#2} \rightarrow {#3}}
\newcommand{\avg}[1]{\langle {#1} \rangle}
\newcommand{\Sum}[2][0]{\sum_{{#2} = {#1}}^{\infty}}
\newcommand{\Lim}[1]{\lim_{{#1} \rightarrow \infty}}
\newcommand{\Binom}[2]{\begin{pmatrix} {#1} \cr {#2} \end{pmatrix}}
\newcommand{\duline}[1]{\underline{\underline{#1}}}
\newcommand{\bra}[1]{\langle {#1} |}
\newcommand{\ket}[1]{| {#1} \rangle}
\renewcommand{\figurename}{Slika}

\begin{document}
\paragraph{Hermitsko adjungiran operator} Operator je hermitsko adjungiran (ozna\v cimo \(A^\dag\)), \v ce velja:
\[\bra{\varphi}A\ket{\psi} = \avg{B\varphi|\psi},~~B = A^\dag\]
Lastnosti:
\begin{align*}
    A &= zB,~~z\in\C \\
    A^\dag &= z^*B^\dag \\
    A &= \ket{m}\bra{n} \\
    A^\dag &= \ket{n}\bra{n} \\
    \bra{\varphi}A\ket{\psi} &= \avg{\varphi|m}\avg{n|\psi} =\left(\avg{\psi|n}\avg{m|\varphi}\right)^* \\
    &= \left(\bra{\psi}A^\dag\ket{\varphi}\right)^* = \avg{A^\dag\varphi|\psi} \\
    (\mu A + \lambda B)^\dag & = \mu^*A^\dag + \lambda^*A^\dag \\
    (AB)^\dag & = B^\dag A^\dag \\
    (A^\dag)^\dag & = A
\end{align*}
Kako najdemo hermitsko adjungiran operator nekega operatorja?
\[A = IAI = \sum_{mn}\ket{m}A_{nm}\bra{n}\]
\[A^\dag = IAI = \sum_{mn}\ket{n}A^*_{mn}\bra{m} = \sum_{mn}\ket{m}A^*_{nm}\bra{n}\]
Se pravi: \(A_{mn} \to A^*_{nm}\)
\paragraph{Sebi adjungirani operatorji} Operator je sebi adjungiran, \v ce velja:
\[A = A^\dag, ~~~ A_{mn} = A^*_{nm}, ~~~ D(A) = D(A^\dag)\]
\v Ce je operator sebi adjungiran, velja spektralni teorem (tega ne bomo dokazovali):
\v Ce poi\v s\v cemo vse lastne funkcije \(\ket{n}\) operatorja \(A\), se pravi:
\[A\ket{n} = a_n\ket{n},\]
tvorijo funkcije \(\ket{n}\) bazo v \(L^2\).
\paragraph{Unitarni operatorji} Operator je unitaren, \v ce je njegov inverz hermitsko adjungiran, torej:
\[U^{-1} = U^\dag,~~~ U^\dag U = UU^\dag = I\]
Nekaj lastnosti: Imamo operatorja \(\ket{\varphi}\) in \(\ket{\psi}\). Ozna\v cimo \(\ket{\tilde{\varphi}} = U\ket{\varphi}\) in \(\ket{\tilde{\psi}} = U\ket{\psi}\)
\[\avg{\varphi|\psi} = \avg{U^\dag\tilde{\varphi}|U^\dag\tilde{\psi}} = \bra{\tilde{\varphi}}UU^\dag\ket{\tilde{\psi}} = \avg{\tilde{\varphi}|\tilde{\psi}}\]
Se pravi unitarni operatorji ohranjajo skalarni produkt. \\[2mm]
Naj bo \(A\) operator. Velja:
\[\bra{\varphi}A\ket{\psi} = \bra{U^\dag\tilde{\varphi}}A\ket{U^\dag\tilde{\psi}}=\bra{\tilde{\varphi}}\tilde{A}\ket{\tilde{\psi}}\]
Ozna\v cili smo \(\tilde{A} = UAU^\dag\) \\[2mm]
Naj bo \(a\) lastna vrednost operatorja \(A\) in \(\ket{a}\) pripadajo\v ci lastni vektor.
\[UAU^\dag U\ket{a} = aU\ket{a}\]
Sledi \(\tilde{A}\ket{\tilde{a}} = \tilde{a}\ket{\tilde{a}}\). \\[2mm]
\v Ce je \(K=K^\dag\) unitaren, je tudi operator \(U\), definiran kot \[U = e^{iK}\] unitaren, in sicer je
\[U^\dag = e^{-iK}\]
Enoparametri\v cni unitarni operatorji imajo ravno tako obliko, in sicer:
\[U(s) = e^{isK},\]
kjer je \(K^\dag = K\) - temu operatorju pravimo tudi generator.
\paragraph{\v Casovni razvoj stanja z unitarnim operatorjem} Operator lahko razvijemo v vrsto. Funkcijske vrste imajo obliko:
\[f(x) = \sum_n c_nx^n\]
Tako lahko re\v cemo \[\widehat{B} = f(\widehat{A}) = \sum_n c_n\widehat{A}^n\]
Primer: \(f(x) = e^x\), \(\widehat{A} = \pd{}{x}\)
\[e^{\pd{}{x}} = 1 + \pd{}{x} + \frac{1}{2}\pd{^2}{x^2} + \frac{1}{6}\pd{^3}{x^3} + ... + \frac{1}{n!}\pd{^n}{x^n} + ...\]
Kaj se pri tak\v snem po\v cetju dogaja z lastnimi vrednostmi?
\[f(\widehat{A})\ket{a_n} = \sum_m c_m\widehat{A}^m\ket{a_n} = \sum_m c_m(a_n)^m\ket{a_n} = f(a_n)\ket{a_n}\]
\v Ce ima operator \(\widehat{A}\) lastno vrednost \(a_n\), bo imel operator \(f(\widehat{A})\) lastno vrednost \(f(a_n)\). \\[2mm]
Primer: Imamo stacionarno Schr\" odingerjevo ena\v cbo \(H\ket{\varphi_n} = E_n\ket{\varphi_n}\)
\[\ket{\psi(t)} = \sum_n \avg{\varphi_n | \psi(0)}e^{-\frac{iE_nt}{\hbar}}\ket{\varphi_n}\]
\(E_n\) zamenjamo s \(\widehat{H}\):
\[\ket{\psi(t)} = \sum_n \avg{\varphi_n | \psi(0)}e^{-\frac{iHt}{\hbar}}\ket{\varphi_n}\]
Opomba: \(\avg{\varphi_n | \psi(0)}\) je konstanta, zato lahko zamenjamo vrstni red in izpostavimo operator.
\[= e^{-\frac{iHt}{\hbar}} \sum_n \avg{\varphi_n | \psi(0)} \ket{\varphi_n}\]
\[= e^{-\frac{iHt}{\hbar}} \sum_n \ket{\varphi_n}\avg{\varphi_n|\psi(0)}\]
Vemo: \(\sum_n \ket{\varphi_n}\bra{\varphi_n} = I\), torej velja:
\[\ket{\psi(t)} = e^{-\frac{iHt}{\hbar}} \ket{\psi(0)}\]
\paragraph{Reprezentacoja \(p\) in \(x\)} Imamo fourierovo transformacijo:
\[f(x) = \int_{-\infty}^{\infty} \tilde{f}(k)\,e^{ikx}\dif k\]
Funcija \(\tilde{f}\) j eseveda Fourierova transformiranka funkcije \(f\):
\[\tilde{f}(k) = \frac{1}{2\pi}\int_{-\infty}^{\infty} f(x')\,e^{-ikx'}\dif x'\]
Sledi:
\[f(x) = \int_{-\infty}^{\infty}\left(\frac{1}{2\pi}\int_{-\infty}^{\infty}f(x')e^{-ikx'}\dif x'\right)e^{ikx}\dif k\]
\[= \int_{-\infty}^{\infty}\left(\frac{1}{2\pi}\int_{-\infty}^{\infty}e^{-ik(x'-x)}\dif k\right)f(x')\dif x\]
S tem smo implicitno definirali Diracovo delta funkcijo:
\[\delta(x'-x) = \frac{1}{2\pi} \int_{-\infty}^{\infty}e^{-ik(x'-x)}\dif k\]
Ta integral seveda ne konvergira, vendar v kvantni mehaniki tak\v sno funkcijo vedno mno\v zimo s kako funkcijo, ki jo dovolj omeji. \\
Primer: Prost delec v potencialu \(V(x) = 0\):
\[H = \frac{p^2}{2m}\]
\[p\ket{\varphi_0} = p_0\ket{\varphi_0}\]
\[-i\hbar\pd{}{x}\varphi_{p0}(x) = p_0\varphi|{p0}(x)\]
\[\varphi_{p0} = Ce^{i\frac{p_0}{\hbar}x}\]
To pomeni, da je \(|\varphi_{p0}|^2 = |C|^2\), torej funkcije ne bomo mogli normirati. Ima pa slede\v co zanimivo lastnost:
\v Ce izberemo \(C = 1/\sqrt{2\pi\hbar}\), velja:
\[\int_{-\infty}^{\infty}\varphi^*_{p0}(x)\varphi_{p0}(x)\dif x = \delta(p_0 - p)\]
Velikost gibalne koli\v cine \(p\) tu prevzame vlogo spremenljivke \(k\) (v eksponentu smo nenazadnje imeli \(\pm i (p_0/\hbar) x\), torej \(k = p_0/\hbar\)).
Sledi: \[\psi(x) = \frac{1}{\sqrt{2\pi\hbar}}\int_{-\infty}^{\infty}\tilde{\psi}(p)e^{i\frac{px}{\hbar}}\dif p\]
in
\[\tilde{\psi}(p) = \frac{1}{\sqrt{2\pi\hbar}}\int_{-\infty}^{\infty}\psi(x)e^{-i\frac{px}{\hbar}}\dif x\]
Posledica:
\[\widehat{p}\psi(x) = -i\hbar\pd{}{x}\psi(x) = \int_{-\infty}^{\infty}\tilde{\psi}(p)\left(-i\hbar\pd{}{x}\right)\varphi_p(x)\dif p = \int_{-\infty}^{\infty}p\tilde{\psi}(p)\varphi_p(x)\dif p\]
Tu je \(\varphi_p\) lastna funkcija operatorja \(\widehat{p}\). Sledi:
\[\widehat{p}\psi \leftrightarrow p\tilde{\psi}\]
Sledi: \v ce odvajamo funkcijo, je to enako, kot \v ce bi njeno transformiranko pomno\v zili z nekim \v stevilom. V splo\v snem velja:
\[\left(-ih\pd{}{x}\right)^n\psi(x) = \widehat{p}^n\psi(x) \leftrightarrow p^n\tilde{\psi}(p)\]
Podobno velja za operator \(x\) (s podobno izpeljavo):
\[x\psi(x) \leftrightarrow \left(+i\hbar\pd{}{p}\right)\tilde{\psi}(p)\]
Opomba: Te funkcije \(\psi\) niso nujno normirane in jih tudi ne moremo normirati, kot bi to lahko po\v celi s funkcijami v \(L^2\). Najve\v c, kar lahko naredimo, je da jih nekako normiramo z uporabo \(\delta\) funkcije. \\[2mm]
Velja tudi: \[\int |\psi(x)|^2\dif x = \int |\tilde{\psi}(p)|^2\dif p\]
\paragraph{Verjetnostna amplituda} \(\left(\avg{x|\psi}~\text{in}~\avg{p|\psi}\right)\)
Od prej imamo \[\ket{\psi} = \int \tilde{\psi}\ket{p}\dif p\]
\[\avg{p_1|p}=\int\tilde{\psi}(p)\avg{p_1|p}\dif p = \int\tilde{\psi}(p)\delta(p_1-p) = \tilde{\psi}(p_1)\]
Kot smo \v ze prej pokazali, je \(\delta(p_0 - p) = \avg{p_0|p}\). \(\delta\) funkcijo, ki smo jo uporabljali lani, lahko opi\v semo kot \(\delta_{nm} = \avg{\varphi_n|\varphi_m}\). \\
Velja torej \(\avg{p|\psi} = \tilde{\psi}(p)\). Podobno delamo za \(x\):
\[\tilde{x} \psi_0(x) = x_0\psi_0(x)\]
\[x\int \tilde{\psi}_0(p)\varphi_p(x)\dif p = \int \left(i\hbar\pd{}{p}\tilde{\psi}_0(p)\varphi_p(x)\dif p = x_0\int\tilde{\psi}_0(p)\varphi_p\dif p\right)\]
Tu je \(x_0\) lastna vrednost funkcije \(\psi_0\). Sledi:
\[i\hbar\pd{}{p}\tilde{\psi}_0(p) = x_0\tilde{\psi}_0(p) \Rightarrow \frac{1}{\sqrt{2\pi\hbar}}e^{-i\frac{px_0}{\hbar}}=\varphi_p^*(x_0)\]
\[\psi_0(x) = \int\varphi^*_p(x_0)\varphi_p(x)\dif p = \delta(x-x_0)\]
Povzetek:
\[\widehat{x}\psi_0 = x\psi_0(x) = x_0\psi_0(x) \Rightarrow \psi_0(x) = \delta(x_0-x)\]
Zdaj izra\v cunajmo \(\avg{x_0|\psi}\)
\[\avg{x_0|\psi} = \int \tilde{\psi}\avg{x_0|p}\dif p\]
Vemo:
\[\avg{x_0|p} = \int \delta(x_0-x)\frac{1}{\sqrt{2\pi\hbar}}e^{i\frac{px}{\hbar}}\dif x = \frac{1}{2\pi\hbar}e^{i\frac{px_0}{\hbar}}\]
Torej je
\[\avg{x_0|\psi} = \int \tilde{\psi}(p)\frac{1}{\sqrt{2\pi\hbar}}e^{i\frac{px_0}{\hbar}}\dif p\]
V tem prepoznamo Fourierovo transformacijo funkcije \(\psi\) v to\v cki \(x_0\).
Zaklju\v cek:
\[\avg{x|\psi} = \psi(x)\]
\[\avg{p|\psi} = \tilde{\psi}(p)\]
\paragraph{Razvoj valovne funkcije}
\[\ket{\psi} = \int \psi(x)\ket{x}\dif x = \int \avg{x|\psi}\ket{x}\dif x = \left(\int \ket{x}\bra{x}\dif x\right)\ket{\psi}\]
Vemo, da je \(\int \ket{x}\bra{x}\dif x = 1\). Dobili pa smo razvoj valovne funkcije po \(x\). Podobno lahko valovno funkcijo razvijemo po \(p\):
\[\ket{\psi} = \left(\int \ket{p}\bra{p}\dif p\right)\ket{\psi}\]
V splo\v snem ne moremo trditi, da je \(V(x) = 0\). Lahko pa obravnavamo tudi primere, ko \(V(x)\) sicer ni enak \(0\), vendar gre proti 0, ko \(x\) nara\v ste preko nekega radija \(R\).
Tedaj je namre\v c \[I = \sum_n \ket{\varphi_n}\bra{\varphi_n} + \int\ket{p}\bra{p}\dif p\]
\end{document}