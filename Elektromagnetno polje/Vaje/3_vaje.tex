\documentclass[a4paper]{article}
\usepackage{amsmath, amssymb, amsfonts}
\usepackage[margin=1in]{geometry}
\usepackage{graphicx}
\usepackage{tikz}
\usepackage{esint}
\setlength{\parindent}{0em}
\setlength{\parskip}{1ex}

\newcommand{\vct}[1]{\overrightarrow{#1}}
\newcommand{\dif}{\,\mathrm{d}}
\newcommand{\pd}[2]{\frac{\partial {#1}}{\partial {#2}}}
\newcommand{\dd}[2]{\frac{\mathrm{d} {#1}}{\mathrm{d} {#2}}}
\newcommand{\C}{\mathbb{C}}
\newcommand{\R}{\mathbb{R}}
\newcommand{\Q}{\mathbb{Q}}
\newcommand{\Z}{\mathbb{Z}}
\newcommand{\N}{\mathbb{N}}
\newcommand{\fn}[3]{{#1}\colon {#2} \rightarrow {#3}}
\newcommand{\avg}[1]{\langle {#1} \rangle}
\newcommand{\Sum}[2][0]{\sum_{{#2} = {#1}}^{\infty}}
\newcommand{\Lim}[1]{\lim_{{#1} \rightarrow \infty}}
\newcommand{\Binom}[2]{\begin{pmatrix} {#1} \cr {#2} \end{pmatrix}}
\newcommand{\duline}[1]{\underline{\underline{#1}}}
\newcommand{\bra}[1]{\langle {#1} |}
\newcommand{\ket}[1]{| {#1} \rangle}
\renewcommand{\figurename}{Slika}

\begin{document}
Zadnji\v c smo re\v sevali Laplaceovo ena\v cbo z robnimi pogoji
\begin{enumerate}
    \item \(U(x, 0) = 0\)
    \item \(U(x, a) = 0\)
    \item \(U(0, 0<y<a) = U_0\)
    \item \(U(x\to\infty, y) < \infty\)
\end{enumerate}
Uporabili bomo separacijo spremenljivk:
\[X''Y + XY'' = 0\]
\[\frac{X''}{X} + \frac{Y''}{Y} = 0\]
Sledi:
\[\frac{X''}{X} = -\frac{Y''}{Y} = \kappa^2\]
Dobimo splo\v sno re\v sitev \[X = Ce^{\kappa} + De^{-\kappa x}\]
\[Y = A\sin(\kappa y) + B\cos(\kappa y)\]
Zaradi \v cetrtega robnega pogoja je \(C = 0\), zaradi prvega pa \(B = 0\). Sledi:
\[U(x, y) = K\sin(\kappa y) e^{-\kappa x}\]
Iz drugega robnega pogoja je \(a\kappa = n\pi\)
Iz tretjega pogoja dobimo, da je
\[\sum_n K_n \sin\left(\frac{n\pi}{a}y\right) = U_0\]
To nam pove nekaj o koeficientih \(K_n\). Izra\v cunajmo najprej skalarni produkt z neko drugo lastno funkcijo:
\[\int_{0}^{\infty} \Sum{n} K_n\sin\left(\frac{n\pi}{a}y\right)\sin\left(\frac{m\pi}{a}y\right)\dif y
= \Sum{n} K_n \int_{0}^{\infty}\sin\left(\frac{n\pi}{a}y\right)\sin\left(\frac{m\pi}{a}y\right)\dif y = \Sum{n} \frac{K_n a}{2}\delta_{mn} = \frac{K_ma}{2}\]
Zdaj izra\v cunajmo \v se koeficiente \(K_n\):
\[\int_{0}^{\infty} U_0\sin\left(\frac{m\pi}{a}y\right)y\dif y = ... = -\frac{U_0a}{m\pi}\left(\cos m\pi - 1\right) = \frac{U_0a}{m\pi}\left(1 - (-1)^m\right)\]
Se pravi je \[K_m = \frac{2\,U_0}{m\pi}\left(1 - (-1)^m\right)\]
Kon\v cni rezultat je tedaj:
\[U(x, y) = \Sum[1]{m} \frac{2U_0}{m\pi}\left(1 - (-1)^m\right)\sin\left(\frac{m\pi}{a}y\right)\,e^{-\frac{m\pi}{a}x}\]
Tega ne moremo izraziti analiti\v cno, lahko pa si ogledamo limito \(x \gg a\):
\[U(x, y) \approx \frac{4\,U_0}{\pi}\sin\left(\frac{\pi}{a}y\right)\,e^{-\frac{\pi}{a}x}\]
Izra\v cunamo lahko tudi \(\vct{E}(x, \frac{a}{2})\) in izrazimo le komponento \(x\):
\[E_x(x) = -\pd{U(x, a/2)}{x}\]
\[= \pd{}{x}\Sum[1]{n}\frac{2\,U_0}{n\pi}\left(1 - (-1)^n\right)\sin\left(\frac{n\pi}{a}\frac{a}{2}\right)\,e^{-\frac{n\pi}{a}x}\]
\[= \frac{2\,U_0}{a}\Sum[1]{n} \left(1 - (-1)^n\right) \sin\left(\frac{m\pi}{2}\right)e^{-\frac{m\pi}{2}x}\]
Vidimo, da velja:
\[\left(1 - (-1)^n\right)\sin\left(\frac{n\pi}{2}\right) = \begin{cases}
    2, & \mathrm{mod}_4~n = 1 \\
    -2, & \mathrm{mod}_4~n = 3 \\
    0, & \text{sicer}
\end{cases}\]
Se pravi je \[E_x = \frac{4\,U_0}{a}\left[e^{-\frac{\pi}{a}x} - e^{-\frac{3\pi}{a}x} + e^{-\frac{5\pi}{a}x} + ...\right]\]
Ozna\v cimo \(\alpha = \exp\left(-\frac{2\pi}{a}x\right)\)
\[E_x(x) = \frac{4\,U_0}{a}e^{-\frac{\pi x}{a}}\left[1 - \alpha + \alpha^2 - \alpha^3 + ...\right]\]
Prepoznamo geometrijsko vrsto in iz dobljenega izrazimo hiperboli\v cno funkcijo:
\[E_x(x) = \frac{2\,U_0/a}{\mathrm{cosh}\left(\pi x / a\right)}\]
\paragraph{Naloga.} Prepolovljena prevodna cev s polmerom \(a\). Uporabimo jo kot kondenzator in ga napojimo z napetostjo \(U_0\).
V cilindri\v cnih koordinatah \v zelimo izraziti \(U\), kar seveda naredimo z Laplaceovo ena\v cbo \(\nabla^2U(r, \varphi) = 0\).
\[\nabla^2 = \frac{1}{r}\pd{}{r}\left(r\pd{}{r}\right) + \frac{1}{r^2}\pd{^2}{r^2}\]
Uporabimo separacijo: \(U(r, \varphi) = R(r)\Phi(\varphi)\)
Po nekaj ra\v cunanja dobimo ena\v cbo
\[\frac{\frac{1}{r}R' + r^2R''}{R} = -\frac{\Phi''}{\Phi} =: m^2\]
Dobimo dve diferencialni ena\v cbi:
\[\Phi'' + m^2\Phi = 0\]
\[r^2R'' + rR' -m^2R = 0\]
Re\v sitev prve so kotne funkcije:
\[\Phi(\varphi) = A_m\sin(m\varphi) + B_m\cos(m\varphi)\]
Iz periodi\v cnosti dobimo pogoj, da mora biti \(m\) celo \v stevilo, in sicer \(m = 1, 2, 3, ...\) (negativna ne pridejo v po\v stev zaradi antisimetri\v cnosti sinusa?) \\
Diferencialno ena\v cbo za \(R\) re\v sijo poten\v cne funkcije. \[R(r) = C_mr^m + D_mr^{-m}\]
Konstante \(A_m, B_m, C_m, D_m\) dolo\v cimo z robnimi pogoji:
\[U(a, \varphi) = \begin{cases}
    \frac{U_0}{2}, & 0 < \varphi < \pi \\
    -\frac{U_0}{2}, & \pi < \varphi < 2\pi \\
\end{cases}\]
\[U(0, \varphi) < \infty~\text{(sledi \(D_m = 0\))}\]
Opazimo: Robni pogoj pri \(r = a\) je liha funkcija \(\varphi\), torej je tudi funkcija \(U\) liha. Sledi \(B_m = 0\).
\[U_m(r, \varphi) = \Sum[1]{m} K_m\sin(m\varphi)r^m\]
Skalarno mno\v zimo dve lastni funkciji:
\[\Sum[1]{m} K_ma^m\sin(m\varphi) = \begin{cases}
    \frac{U_0}{2} & \\
    -\frac{U_0}{2} &
\end{cases}\]
Na obeh straneh skalarno pomno\v zimo z \(U_n\). Leva stran:
\[\Sum[1]{m} K_m a^m \int_{0}^{2\pi} \sin\left(m\varphi\right) \sin\left(n\varphi\right) \dif \varphi = \pi a^nK_n\]
Desna stran:
\[\int_{0}^{\pi} \frac{U_0}{2}\sin(n\varphi)\dif\varphi - \int_{\pi}^{2\pi} \frac{U_0}{2}\sin(n\varphi)\dif\varphi = \frac{U_0}{2n}\left[(1-(-1)^n) + (1 - (-1)^n)\right]\]
Sledi:
\[U(r, \varphi) = \Sum[1]{n} \frac{U_0}{n\pi}\left(1 - (-1)^n\right)\sin(n\varphi)\left(\frac{r}{a}\right)^n\]
\end{document}