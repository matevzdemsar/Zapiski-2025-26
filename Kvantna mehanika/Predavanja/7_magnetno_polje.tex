\documentclass[a4paper]{article}
\usepackage{amsmath, amssymb, amsfonts}
\usepackage[margin=1in]{geometry}
\usepackage{graphicx}
\usepackage{tikz}
\usepackage{esint}
\setlength{\parindent}{0em}
\setlength{\parskip}{1ex}

\newcommand{\vct}[1]{\overrightarrow{#1}}
\newcommand{\dif}{\,\mathrm{d}}
\newcommand{\pd}[2]{\frac{\partial {#1}}{\partial {#2}}}
\newcommand{\dd}[2]{\frac{\mathrm{d} {#1}}{\mathrm{d} {#2}}}
\newcommand{\C}{\mathbb{C}}
\newcommand{\R}{\mathbb{R}}
\newcommand{\Q}{\mathbb{Q}}
\newcommand{\Z}{\mathbb{Z}}
\newcommand{\N}{\mathbb{N}}
\newcommand{\fn}[3]{{#1}\colon {#2} \rightarrow {#3}}
\newcommand{\avg}[1]{\langle {#1} \rangle}
\newcommand{\Sum}[2][0]{\sum_{{#2} = {#1}}^{\infty}}
\newcommand{\Lim}[1]{\lim_{{#1} \rightarrow \infty}}
\newcommand{\Binom}[2]{\begin{pmatrix} {#1} \cr {#2} \end{pmatrix}}
\newcommand{\duline}[1]{\underline{\underline{#1}}}
\newcommand{\bra}[1]{\langle {#1} |}
\newcommand{\ket}[1]{| {#1} \rangle}
\renewcommand{\figurename}{Slika}
\newcommand{\rot}[1]{\nabla \times \vct{#1}}

\begin{document}
\paragraph{Lokalne umeritvene transformacije.} Zapi\v semo:
\[\vct{B} = \rot{A}\]
\[\vct{E} = -\nabla\phi - \pd{\vct{A}}{t}\]
\[\vct{F} =e\vct{E} + \vct{r}\times\vct{B}\]
Vemo, da je \(\vct{A}\) nedolo\v cen do gradienta neke funkcije. Lokalna umeritvena transformacija torej pomeni, da mu pri\v stejemo nek gradient.
\[\vct{A}' = \vct{A} + \nabla\Lambda(\vct{r}, t)\]
\[\phi' = \phi - \pd{\Lambda}{t}\]
Ko to vstavimo v Schr\" odingerjevo ena\v cbo, dobimo ena\v cbo za \(\psi'\):
\[i\hbar\pd{\psi'}{t} = \frac{(\vct{p} - e\vct{A'})^2}{2m}\psi' + e\phi'\psi'\]
\paragraph{Globalna umeritvena transformacija.} Uporabimo nastavek
\[\psi' = e^{i\delta(\vct{r}, t)}\]
Ko ga vstavimo v Schr\" odingerjevo ena\v cbo, dobimo:
\[i\hbar\pd{}e^{i\delta}\psi = \frac{1}{2m}\sum_\alpha\left(-i\hbar\pd{}{x_\alpha} + \left(eA'_\alpha - \hbar\pd{\delta}{x_\alpha}\right)\right)^2\psi + e\phi'\psi\]
Ko to re\v simo, ugotovimo, da mora biti \(\delta(\vct{r, t}) = \frac{e}{\hbar}\Lambda\). Tako imamo za \(\psi\) in \(\psi'\) enako ena\v cbo in velja:
\[\psi' = e^{i\frac{\hbar}{e}\Lambda}\psi\]
\[\vct{A}' = \vct{A} + \nabla\Lambda(\vct{r}, t)\]
\[\phi' = \phi - \pd{\Lambda}{t}\]
\paragraph{Aharonov-Bohmov pojav.} Zamislimo si elektron v ravnini, skozi katero te\v ce tuljava. Znotraj tuljave je polje enako \(\vct{B}\), zunaj pa \(0\), vendar ima \v se vedno magnetni potencial, ki lahko vpliva na elektron.
\[\vct{B} = \rot{A} = 0\]
\[\vct{A}  = \nabla\Lambda \neq 0\]
\[\Lambda(\vct{r}) = \lambda(\vct{r_0}) - \int_{\vct{r_0}}^{\vct{r}}\vct{A}(\vct{R})\dif\vct{R}\]
Mislimo si, da integral od \(\vct{r_0}\) do \(\vct{r}\) poteka po neki poti - ker je \(\vct{A}\) o\v citno potencialno polje, izbira poti ni pomembna.
Poglejmo dva primera: \(\vct{B} = 0, \vct{A} \neq 0\) in \(\vct{B} = 0, \vct{A} = 0\). \\
Prvi primer:
\[i\hbar\pd{\psi_A}{t} = \frac{(\vct{p} - e\vct{A})^2}{2m}\psi_A + V\psi_A\]
Tega ne znamo, zato si najprej oglejmo drugi primer: \(\vct{A} = 0\). To je sicer popolnoma nemogo\v ce, vendar nam bo pomagalo pri ra\v cunanju.
\[i\hbar \pd{\psi_0}{t} = \frac{p^2}{2m}\psi_0 + V\psi_0\]
Lahko si zamislimo, da smo naredili transformacijo \(\vct{A}' = \vct{A} + \nabla(-\Lambda) = 0\)
\[\psi_A(\vct{r}, t) \exp\left(i\frac{e}{\hbar}\int_{\vct{r_0}}^{\vct{r}}\vct{A}(\vct{r}')\cdot\dif\vct{r}'\right)\psi_0(\vct{r, t})\].
Mislimo si, da mora elektron priti okoli obmo\v cja z neni\v celnim \(\vct{B}\) - od to\v cke I do to\v cke II. V to ga "prisilimo" tako, da iz teh to\v ck napeljemo \v zici, skozi kateri te\v ce kot \(I\). Okoli magnetnega polja lahko gre elektron po levi ali po desni (pot \(1\) ali pot \(2\)). Zaradi kontinuitete velja:
\[\psi_{\text{II}, 0} = \psi_1 + \psi_2,\qquad \vct{B} = 0, \vct{A} = 0\]
\v Ce elektrona ne motimo, ga torej do to\v cke II pride toliko, kolikor ga je \v slo skupaj po levi in po desni.
Zdaj naj bo \(\vct{B} \neq 0\) in \(\vct{A} \neq 0\). Velja:
\[\psi_{\text{II}, B} = e^{i\delta_1}\psi_1 + e^{i\delta_2}\psi_2\]
\[\delta_{1, 2} = \frac{e}{\hbar}\int_{\gamma_1, \gamma_2}\vct{A}(\vct{r})\cdot\dif\vct{r}\]
Izpostavimo \(e^{i\delta_2}\):
\[\psi_{\text{II}, B} = e^{i\delta_2} \left(e^{i(\delta_1 - \delta_2)}\psi_1 + \psi_2\right) \approx\]
Predpostavimo \(\psi_1 \approx \psi_2\)
\[\approx e^{i\delta}\left(1 + e^{i(\delta_1 - \delta_2)}\right)\psi_2\]
\[\delta = \delta_1 - \delta_2 = \frac{e}{\hbar} \oint \vct{A}\cdot\dif\vct{r} = \frac{e}{\hbar}\iint\rot{A}\cdot\dif\vct{S} = \frac{e}{\hbar}\iint\vct{B}\cdot\dif\vct{S} = \frac{e}{\hbar}\Phi_B\]
Sledi:
\[\frac{I_B}{I_0} = \frac{|\psi_{\text{II}, B}|}{|\psi_{\text{II}, 0}|} = \frac{1}{4}\left|1 + e^{i\frac{e}{\hbar}\Phi_B}\right| = \cos^2\left(\frac{e}{2\hbar}\Phi_B\right)\]
Maksimume dobimo, ko je \[\frac{e}{\hbar}\Phi_B = 2\pi n,~n\in\N\]
\[\Phi_B = \frac{2\pi\hbar}{e}n = \frac{h}{e}n\]
Ozna\v cimo \(\Phi_0 = h/e\). V primeru superprevodnika je \(e=2e_0\), torej je najmanj\v si mo\v zni magnetni pretok skozi superprevodnik najmanj \(h/2e_0 \sim 10^{-19}\,\mathrm{T}\).
\paragraph{Superprevodnik.} Ima slede\v ce lastnosti:
\begin{enumerate}
    \item \(B = 0\). Temu pravimo Meissnerjev pojav, zato je superprevodnik idealen diamagnet. Na to lastnost superprevodnika se lahko vedno zana\v samo (jo je pa precej te\v zko dokazati).
    \item \(R = 0\). V njem ni opaznega upora, je idealen prevodnik.
    \item Energijska vrzel med valen\v cnim in prevodnim pasom ima neke posebne lastnosti.
    \item Levitacija. Ko je superprevodnik dovolj ohlajen (\(T < T_c\)), znotraj njega ni magnetnega polja, skozi dolo\v cene "luknje" pa \v se vedno te\v ce magnetni pretok, ki je celi ve\v ckratnik osnovnega magnetnega pretoka \(n\Phi_0\). Polje znotraj superprevodnika je enako \(0\), ker je superprevodnik ustvaril nasprotno polje, in sicer tako, da je okoli "lukenj" stekel tok. Tako v superprevodniku nastane nekak\v sna obratna slika magneta.
\end{enumerate}
\end{document}