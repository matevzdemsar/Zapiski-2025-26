\documentclass[a4paper]{article}
\usepackage{amsmath, amssymb, amsfonts}
\usepackage[margin=1in]{geometry}
\usepackage{graphicx}
\usepackage{tikz}
\usepackage{esint}
\setlength{\parindent}{0em}
\setlength{\parskip}{1ex}

\newcommand{\vct}[1]{\overrightarrow{#1}}
\newcommand{\dif}{\,\mathrm{d}}
\newcommand{\pd}[2]{\frac{\partial {#1}}{\partial {#2}}}
\newcommand{\dd}[2]{\frac{\mathrm{d} {#1}}{\mathrm{d} {#2}}}
\newcommand{\C}{\mathbb{C}}
\newcommand{\R}{\mathbb{R}}
\newcommand{\Q}{\mathbb{Q}}
\newcommand{\Z}{\mathbb{Z}}
\newcommand{\N}{\mathbb{N}}
\newcommand{\fn}[3]{{#1}\colon {#2} \rightarrow {#3}}
\newcommand{\avg}[1]{\left\langle {#1} \right\rangle}
\newcommand{\Sum}[2][0]{\sum_{{#2} = {#1}}^{\infty}}
\newcommand{\Lim}[1]{\lim_{{#1} \rightarrow \infty}}
\newcommand{\Binom}[2]{\begin{pmatrix} {#1} \cr {#2} \end{pmatrix}}
\newcommand{\duline}[1]{\underline{\underline{#1}}}
\newcommand{\bra}[1]{\left\langle {#1} \right|}
\newcommand{\ket}[1]{\left| {#1} \right\rangle}
\newcommand{\rot}{\vct{\nabla}\times}
\newcommand{\dvg}{\vct{\nabla}\cdot}
\renewcommand{\figurename}{Slika}

\begin{document}
\section{Greenove funkcije - nadaljevanje}
\subsection{Von Neumannov robni pogoj}
Za operator \(\mathcal{L} = \nabla^2 + q(\vct{r})\) i\v s\v cemo re\v sitev Neumannovega problema:
\[\nabla^2u = f, \qquad \pd{u(\vct{r_B})}{n} =  g(\vct{r_n}),\]
ko \(\vct{r_B}\) le\v zi na robu domene. Funkcij \(f\) in \(g\) ne moremo izbrati poljubno, temve\v c mora veljati konsisten\v cni pogoj:
\[\int_{\partial\mathcal{D}}g(\vct{r_B})\dif S_B = \int_{\mathcal{D}}f(\vct{r})\dif^3\vct{r}\]
Naivno bi lahko poskusili kot prej poiskati Neumannovo Greenovo funkcijo \(G_N\), za katero velja
\[\nabla^2G_N = \delta(\vct{r} - \vct{r_0}) \text{  in  } \pd{G_N}{n}(\partial\mathcal{D}) = 0\]
Te\v zava se pojavi, ker pri integralu levega pogoja dobimo \[\int\pd{G_N}{n}\dif S = 1,\]
kar pa zaradi desnega pogoja ni mogo\v ce. Enega od pogojev moramo relaksirati, in sicer recimo:
\[\nabla^2G_N = \delta(\vct{r} - \vct{r_0}) - \frac{1}{V},~V = \int_{\mathcal{D}}\dif^3\vct{r}\quad\text{  in  }\quad\pd{G_N}{n}(\partial\mathcal{D}) = 0\]
Zaradi dodatnega \v clena \(1/V\) si robna pogoja ne nasprotujeta ve\v c. Izrazimo \(u\):
\[u(\vct{r}) = \int_{\mathcal{D}}G_N(\vct{r}, \vct{r_0})f(\vct{r_0})\dif^3\vct{r_0} - \int_{\partial\mathcal{D}}g(\vct{r_B})G_N(\vct{r}, \vct{r_B})\dif S_B + \text{konst.}\]
\subsection{Difuzijski operator}
\[\mathcal{L} = \left(\pd{}{t} - D\nabla^2\right)\]
\subsubsection{Dirichletov robni pogoj}
Re\v sujemo problem \(\mathcal{L}u = f(\vct{r}, t)\)
z Dirichletovim robnim pogojem \(u(\vct{r_B}, t) = g(\vct{r_B}, t)\) in nekak\v snim za\v cetnim pogojem \(h(\vct{r})\)
Greenovo funkcijo i\v s\v cemo kot
\[\mathcal{L}G = \delta(\vct{r} - \vct{r_0})\delta(t - t_0),\qquad G(\vct{r_B}, \vct{r_0}) = 0\]
Ko jo enkrat imamo, deluje kot propagator in lahko izra\v cunamo
\[u(\vct{r}, t) = \int_0^\infty\int_{\mathcal{D}}f(\vct{r_0}, \tau)G(\vct{r}, \vct{R_0}; t - \tau)\dif^3\vct{r_0}\dif\tau\]
\[+ \int_{\mathcal{D}}h(\vct{r_0})G(\vct{r}, \vct{r_0};t)\dif^3\vct{r} - D\int_{0}^{\infty}\int_{\partial\mathcal{D}}g(\vct{r_B}, \tau)\pd{G}{n_B}(\vct{r}, \vct{r_B}, t - \tau)\dif^3\vct{r_B}\dif\tau\]
Prvi integral je \v casovni razvoj nehomogenosti, drugi integral je \v casovni razvij za\v cetnega pogoja, tretji integral pa \v casovni razvoj robnega pogoja.
\subsection{Greenova funkcija za krog}
\subsubsection{Vsota neskon\v cne funkcije in re\v sitve z robnim pogojem}
Imamo laplaceov operator \(\mathcal{L} = \nabla^2\) in Dirichletov robni pogoj za krog. I\v s\v cemo
\[\nabla^2G(r, \varphi, r_0, \varphi_0) = \frac{1}{r}\delta(r - r_0)\delta(\varphi - \varphi_0)\]
Predfaktor \(1/r\) potrebujemo, da se bo pri integriranju kraj\v sal z Jacobianom.
To bomo naredili na tri na\v cine, in sicer z nastavkon \(G = G_\infty + g\). Za tak\v sen problem je \[G_\infty = \frac{\ln r}{2\pi}\]
I\v s\v cemo \v se \(g\):
\[\nabla^2 g = 0\]
\[g\Big|_{\partial\mathcal{D}} = -\frac{1}{4\pi}\ln[R^2 + r_0^2 - 2Rr_0\cos(\varphi - \varphi_0)]\]
Re\v sitev Laplaceove ena\v cbe na krogu so funkcije
\[g = c_0 + \sum_{m=1}^{\infty} r^m\left(c_m\cos(m\varphi) + d_m\sin(m\varphi)\right)\]
Velja:
\[\ln\left[R^2\left(1 + \frac{r_0^2}{R^2} - \frac{2r_0}{R}\cos(\varphi - \varphi_0)\right)\right] = \ln R^2 + \sum_{m = 1}^{\infty} -\frac{r_0^m}{R^m}\frac{2}{m}\cos\left[m(\varphi - \varphi_0)\right]\]
To lahko doka\v zemo na primer s primerjavo vrednosti in odvodov v \(\varphi - \varphi_0 = 0\). Uporabimo \v se adicijski izrek:
\[\cos[m(\varphi - \varphi_0)] = \cos(m\varphi)\cos(m\varphi_0) + \sin(m\varphi)\sin(m\varphi_0)\]
Da zadostimo robnemu pogoju, je
\[c_0 = -\frac{\ln R^2}{4\pi}\]
Nato pa primerjamo vrsti in preberemo koeficiente:
\[c_m = \frac{1}{R^m}\left(\frac{r_0}{R}\right)^m \frac{1}{2\pi m}\cos(m\varphi_0)\]
\[d_m = \frac{1}{R^m}\left(\frac{r_0}{R}\right)^m \frac{1}{2\pi m}\sin(m\varphi_0)\]
Vse skupaj lahko malo kraj\v se zapi\v semo kot vrsto za \(g\):
\[g(\vct{r}, \vct{r_0}) = \frac{1}{4\pi}\left[-\ln R^2 + \sum_{m = 1}^{\infty}\frac{2}{m}\left(\frac{r_0 r}{R^2}\right)^m\cos(m(\varphi - \varphi_0))\right]\]
Uporabimo kar znano vsoto od prej in zapi\v semo
\[-\frac{1}{4\pi}\ln\left[R^2 + \frac{(r_0r)^2}{R^2} - 2rr_0\cos(\varphi - \varphi_0)\right]\]
To se\v stejemo z \(G_\infty\). Ozna\v cimo \v se \(|\vct{r_*}| = R^2/r_0\), \(\vct{r_*} \parallel \vct{r_0}\) (prezrcalimo \v cez rob kroga).
\[G(\vct{r}, \vct{r_0}) = \frac{1}{4\pi}\ln\frac{|\vct{r} - \vct{r_0}|^2}{\left(\frac{r_0}{R}\right)^2|\vct{r} - \vct{r_*}|^2}\]
\subsubsection{Lepljenje}
Krog razdelimo na zunanje in notranje obmo\v cje. Notranje obmo\v cje je krog s polmerom \(r_0\), zunanje pa kolobar z zunanjim polmerom \(R\) (imejmo kot enoto \(R = 1\)) in notranjim polmerom \(r_0\). Koordinatni sistem obrnemo tako, da je \(\varphi_0 = 1\)
Re\v sujemo Laplaceovi ena\v cbi
\[\nabla^2G_n = \frac{1}{r}\delta(r - r_0)\delta(\varphi)\]
\[\nabla^2G_z = \frac{1}{r}\delta(r - r_0)\delta(\varphi)\]
Pomagamo si s splo\v sno re\v sitvijo Laplaceove ena\v cbe:
\[u = (A_0 + B_0\ln r)(a_0 + b_0\varphi) + \sum_{m \in \Z \setminus \{0\}}\left(A_m r^{|m|} + B_m r^{-|m|}\right)\,e^{im\varphi}\]
Zapi\v semo:
\[\delta(\varphi) = \frac{1}{2\pi}\Sum[-\infty]{m}\cos(m\varphi)\]
Ena\v cimo koeficiente in vidimo, da nam v splo\v sni re\v sitvi Laplaceove ena\v cbe zado\v s\v ca vzeti le \v clene s \(\cos m\varphi\):
\[G_n = A_0 + \Sum[1]{m} A_m r^m \cos(m\varphi)\]
\[G_z = (A_0' + B_0'\ln r) + \Sum[1]{m}\left(A_m'r^m + B_m'r^{-m}\right)\cos m\varphi\]
Hkrati \v zelimo, da je \(G\) v \(r = r_0\) zvezna v \(r = r_0\).
\paragraph{Opomba.} \v Ce si zamislimo, da imamo pri \(G_n\) razvoj
\[G_n = \sum_m f_m(r)\cos(m\varphi),\]
Vidimo, da ima odvod \(f_m'\) okoli \(r_0\) nezvezen skok:
\[f_m'' + \frac{f_m'}{r} - \frac{m^2}{r^2}f_m = \frac{\delta(r - r_0)}{r\pi}\]
\[(rf_m')' - \frac{m^2}{r} = \frac{\delta(r - r_0)}{\pi}\]
Na obeh straneh integriramo od \(r_0 - \varepsilon\) do \(r_0 + \varepsilon\):
\[rf_m'\Big|^{r_0 + \varepsilon}_{r_0 - \varepsilon} = \frac{1}{\pi}\]
Se pravi:
\[\Delta(r_0f_m') = \frac{1}{\pi}\]
\[\Delta(r_0f_0') = \frac{1}{2\pi}\]
Iz zahteve po zveznosti v \(r_0\) dobimo
\[A_0 = B_0'\ln r_0,\qquad A_m = B_m'\left(\frac{1}{r_0^{2m}} - 1\right)\]
Iz zahteve, da na robu \(r = R\) velja \(G_z = 0\), dobimo:
\[A_0' = 0,\qquad A_m' + B_m' = 0\]
Iz skoka v odvodu okoli \(r_0\) dobimo
\[B_0' = \frac{1}{2\pi},\qquad B_m' = -\frac{r_0^m}{2m\pi}\]
Dobili smo
\[G_n = \frac{1}{2\pi}\ln r_0 + \Sum[1]{m}\frac{1}{m}r^m\left(r_0^m - \frac{1}{r_0^m}\right)\cos(m\varphi)\]
\[G_z = \frac{1}{2\pi}\ln r + \Sum[1]{m}\frac{1}{m}r_0^m\left(r^m - \frac{1}{r^m}\right)\cos(m\varphi)\]
Opazimo, da je edina razlika med funkcijama zamenjava spremenljivk \(r\) in \(r_0\), torej gre za nekak\v sno zrcaljenje. Spomnimo, da smo to izra\v cunali za \(R = 1\) in \(\varphi_0 = 0\).
Za poljubem \(\varphi_0\) in poljuben polmer bi morali zamenjati
\[\varphi \to \varphi - \varphi_0\]
\[r \to r/R\]
\[r_0 \to r_0/R\]
\subsubsection{Razvoj po lastnih funkcijah}
Recimo, da poznamo lastne funkcije, za katere je
\[\mathcal{L}u_n = \lambda_n u_n\]
in lahko razvijemo \(G\) in \(\delta(\vct{r} - \vct{r_0})\). Razvoj za \(\delta\) delta funkcijo je enostaven:
\[\delta(\vct{r} - \vct{r_0}) = \sum_n c_n(\vct{r_0})u_n(\vct{r})\]
\[c_n(\vct{r_0}) = \avg{u_n|\delta} = u_n^*(\vct{r_0})\]
Razvoj Greenove funkcije je torej:
\[G = \sum_{\vct{n}}\frac{u_n(\vct{r})u_n^*(\vct{r_0})}{\lambda_n}\]
\paragraph{Opomba.} Opazimo nekaj lastnosti:
\begin{itemize}
    \item \(G(\vct{r}, \vct{r_0}) = G^*(\vct{r_0}, \vct{r})\)
    \item \v Ce je \(\lambda_n = 0\), te \v clene preprosto izpustimo. V tem primeru za na\v so Greenofo funkcijo velja \[\mathcal{L}G = \delta(\vct{r} - \vct{r_0}) - \frac{1}{V},\] kjer je \(V = \sum_n u_n^*(\vct{r_0})u_n(\vct{r})\)
\end{itemize}
Spet vzamemo \(R = 1,~\varphi_0 = 0\).
\v Ce je na primer za Laplaceov operator \(\mathcal{L} = \nabla^2\) lastna funkcija \[u_{mk} \propto J_m(\xi_{mk}r)\cos(m\varphi)\]
Potrebujemo \v se normalizacijsko konstanto, ki je
\[N_0 = \frac{1}{\sqrt{\pi}J_1(\xi_{0k})}\quad\text{ za }m = 0\]
\[N_{m>0} = \sqrt{\frac{2}{\pi}}\frac{1}{J_{m+1}(\xi_{m, k})}\]
Lastne vrednosti so \(-\xi_{m, k}^2\)
\[G = -\Sum[1]{k}\frac{J_0(\xi_{0, k}r)J_0(\xi_{0, k}r_0)}{\pi\xi_{0, k}^2 J_1^2(\xi_{0, k})} - \Sum[1]{m}\Sum[1]{k}\frac{2J_m(\xi_{m, k}r)J_m(\xi_{m, k}r)}{\pi \xi_{m, k}^2J_{m+1}^2(\xi_{m, k})}\cos(m\varphi)\]
Stvar lahko malo lep\v se napi\v semo z upo\v stevanjem lastnosti \(J_{-m} = -J_m\):
\[G = \Sum[-\infty]{m}\Sum[1]{k}-\frac{J_m(\xi_{m, k}r)J_m(\xi_{m, k}r_0)}{\pi \xi_{m, k}^2J_{m+1}^2(\xi_{m, k})}\cos(m\varphi)\]
Vidimo, da je
\[\frac{\delta(r - r_0)}{r} = -2\Sum[1]{k}\frac{J_m(\xi_{m, k}r)J_m(\xi_{m, k}r_0)}{J_{m+1}^2(\xi_{m, k})}\,\qquad \forall m\]
\subsection{Povzetek}
za kon\v cen \(\mathcal{D}\) lahko Greenovo funkcijo poi\v s\v cemo na tri na\v cine:
\begin{itemize}
    \item Z nastavkom \(G = G_\infty + g\).
    \item Z razvojem po lastnih funkcijah operatorja.
    \item Z lepljenjem, pri katerem upo\v stevamo skok v odvodu koeficientov \(f_m'(r)\).
\end{itemize}
\end{document}