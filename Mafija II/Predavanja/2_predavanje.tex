\documentclass[a4paper]{article}
\usepackage{amsmath, amssymb, amsfonts}
\usepackage[margin=1in]{geometry}
\usepackage{graphicx}
\usepackage{tikz}
\usepackage{esint}
\setlength{\parindent}{0em}
\setlength{\parskip}{1ex}

\newcommand{\vct}[1]{\overrightarrow{#1}}
\newcommand{\dif}{\,\mathrm{d}}
\newcommand{\pd}[2]{\frac{\partial {#1}}{\partial {#2}}}
\newcommand{\dd}[2]{\frac{\mathrm{d} {#1}}{\mathrm{d} {#2}}}
\newcommand{\C}{\mathbb{C}}
\newcommand{\R}{\mathbb{R}}
\newcommand{\Q}{\mathbb{Q}}
\newcommand{\Z}{\mathbb{Z}}
\newcommand{\N}{\mathbb{N}}
\newcommand{\fn}[3]{{#1}\colon {#2} \rightarrow {#3}}
\newcommand{\avg}[1]{\langle {#1} \rangle}
\newcommand{\Sum}[2][0]{\sum_{{#2} = {#1}}^{\infty}}
\newcommand{\Lim}[1]{\lim_{{#1} \rightarrow \infty}}
\newcommand{\Binom}[2]{\begin{pmatrix} {#1} \cr {#2} \end{pmatrix}}
\newcommand{\duline}[1]{\underline{\underline{#1}}}
\newcommand{\bra}[1]{\langle {#1} |}
\newcommand{\ket}[1]{| {#1} \rangle}
\renewcommand{\figurename}{Slika}

\begin{document}
\paragraph{Diskretna sila na homogeno sredstvo}
Denimo, da sila deluje na to\v cko \(x_0\) v smeri pravokotno na vrv. Valovna ena\v cba:
\[u_{tt} = c^2 u_{xx} + \frac{F_y}{\mu}\delta(x-x_0)\]
Vzamemo limito integrala \(\int_{x_0 - \varepsilon}^{x_0 + \varepsilon}\dif x\), ko gre \(\varepsilon\) proti 0. Zaradi lastnosti integrala bo ta limita seveda 0.
\[0 = c^2 u_x \Big|^{x_0^+}_{x_0^-} + \frac{F_x}{\mu}\]
Sledi \(F u_x\big|_{x \to x_0} = -F_y\)
\paragraph{Diskretna masa} Izra\v cun je podoben, za maso uporabimo drugi Newtonov zakon.
\[mu_{tt}(x_0) = Fu_x\Big|_{x_0^-}^{x_0^+}\]
\paragraph{Struna iz dveh delov}
Imamo dva dela struna, v katerih ima valovanje razli\v cno hitrost \v sirjenja (ozna\v cimo \(c_1\) in \(c_2\)).
Ozna\v cimo tudi, da se \(c\) spremeni pri \(x=0\).
\[u_1 = u_(x - c_1t) + u_r(x + c_1t)\]
\[u_2 = u_t(x - c_2t)\]
Z indeksom \(i\) tu ozna\v cimo za\v cetni val, z indeksom \(r\) odbiti val, z indeksom \(t\) pa prepu\v s\v ceni val (ne \v casovnega odvoda).
Robni pogoj pri \(x=0\):
\[u_(0, t) = u_t(0, t)~~\text{(zveznost)}\]
\[u_{x}(0, t) = u_{tx}(0, t)~~\text{(zvezna odvedljivost)}\]
V \(x=0\) z integriranjem robnega pogoja o zvezni odvedljivosti dobimo:
\[-\frac{1}{c_1}u_(-c_1t) + \frac{1}{c_1}u_r(c_1t) = -\frac{1}{c_2}u_t(-c_2t)\]
Iz tega lahko spet izrazimo
\[u_r(s) = \frac{c_2-c_1}{c_2 + c_1}u_i(-s)\]
\[u_t(s) = \frac{2c_2}{c_2 + c_1}u_i(s\frac{c_1}{c_2})\]
Ozna\v cimo \(\displaystyle{R = \frac{c_2 - c_1}{c_2 + c_1}}\) (odbojnost) in \(\displaystyle{T = \frac{2c_2}{c_1 + c_2}}\) (prepustnost).
Zakon o ohranitvi energije je oblike \(R^2 + T^2\frac{c_1}{c_2} = 1\)
\end{document}