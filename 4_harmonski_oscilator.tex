\documentclass[a4paper]{article}
\usepackage{amsmath, amssymb, amsfonts}
\usepackage[margin=1in]{geometry}
\usepackage{graphicx}
\usepackage{tikz}
\usepackage{esint}
\setlength{\parindent}{0em}
\setlength{\parskip}{1ex}

\newcommand{\vct}[1]{\overrightarrow{#1}}
\newcommand{\dif}{\,\mathrm{d}}
\newcommand{\pd}[2]{\frac{\partial {#1}}{\partial {#2}}}
\newcommand{\dd}[2]{\frac{\mathrm{d} {#1}}{\mathrm{d} {#2}}}
\newcommand{\C}{\mathbb{C}}
\newcommand{\R}{\mathbb{R}}
\newcommand{\Q}{\mathbb{Q}}
\newcommand{\Z}{\mathbb{Z}}
\newcommand{\N}{\mathbb{N}}
\newcommand{\fn}[3]{{#1}\colon {#2} \rightarrow {#3}}
\newcommand{\avg}[1]{\langle {#1} \rangle}
\newcommand{\Sum}[2][0]{\sum_{{#2} = {#1}}^{\infty}}
\newcommand{\Lim}[1]{\lim_{{#1} \rightarrow \infty}}
\newcommand{\Binom}[2]{\begin{pmatrix} {#1} \cr {#2} \end{pmatrix}}
\newcommand{\duline}[1]{\underline{\underline{#1}}}
\newcommand{\bra}[1]{\langle {#1} |}
\newcommand{\ket}[1]{| {#1} \rangle}
\renewcommand{\figurename}{Slika}

\begin{document}
\paragraph{Kompleten sistem med seboj komutirajo\v cih operatorjev.} Naj za operatorja \(A\) in \(B\) velja \[[A, B] = 0\]
Naj bo \(a\) lastna vrednost \(A\) in \(\ket{a}\) pripadajo\v ci lastni vektor.
\[Ba\ket{a} = BA\ket{a} = AB\ket{a}\]
Oziroma \(AB\ket{a} = aB\ket{a}\). Imamo dve mo\v znosti: \v ce je \(a\) nedegeneriran, je \(B\ket{a} \propto \ket{a}\) in je potemtakem \(\ket{a}\) hkrati lastno stanje operatorja \(B\).
\v Ce je \(a\) degeneriran (torej je lastna vrednost za dve lastni stanji \(\ket{a_1}\) in \(\ket{a_2}\)), pa lahko lastni stanji operatorja \(B\) zapi\v semo kot linearno kombinacijo lastnih stanj \(A\):
\[\ket{b_{1,2}} = c_1\ket{a_1} + c_2\ket{a_2}\]
Lahko imamo tudi cele sisteme operatorjev, ki med seboj komutirajo. Primer so \(H, \vct{L}\) in \(\vct{S}\) (energija, vrtilna koli\v cina in spin).
\paragraph{Postulati (aksiomi) kvantne mehanike} (Kopenhagenska interpretacija):
\begin{enumerate}
    \item Kvantni pojavi na klasi\v cni skali niso zaznavni. Svet je lo\v cen na klasi\v cnega in kvantnega. Kvantni sistem je dolo\v cen s kvantnim stanjem \(\ket{\psi}\), ki je element Hilbertovega prostora stanj.
    \item Vsaki merljivi koli\v cini (opazljivki) ustreza hermitski operator, npr. \(A = A^\dag\).
    \item Pri\v cakovana vrednost operatorja za neko stanje \(\ket{\psi}\) je skalarni produkt \(\avg{\psi|A\psi}\). \v Ce poskus izvedemo velikokrat, bomo torej v povpre\v cju dobili to vrednost.
    \item \v Casovni razvoj je podan z unitarnim razvojem \(\ket{\psi(t)} = U(t)\ket{\psi(0)}\), pri \v cemer je generator Hamiltonov operator \(\displaystyle{U(t) = \exp\left(-\frac{iHt}{\hbar}\right)}\). Od tod sledi tudi Schr\" odingerjeva ena\v cba: \[\dd{\ket{\psi}}{t} = -\frac{iH}{\hbar}\ket{\psi}\]
    \item Kvantna meritev: \begin{itemize}
        \item \v Ce lahko neko stanje razvijemo kot \(\displaystyle \ket{\psi} = \sum_n c_n\ket{a_n}\) po lastnih stanjih operatorja \(A\) (velja torej \(A\ket{a_n} = a\ket{a_n}\)), je verjetnost, da bomo pri merjenju izmerili delec v lastnem stanju \(\ket{a_n}\) enaka \(P_n = |c_n|^2\). Vsota verjetnosti mora biti seveda \(1\), toraj morajo biti \(\{c_n\}_n\) ortonormirana baza.
        \item Po tak\v sni meritvi je sistem v stanju \(\ket{a_n}\), torej velja \(\ket{\psi} = \ket{a_n}\). Temu pravimo kolaps kvantnega stanja. Tedaj je torej \(c_m = 0\) za vsak \(n \neq m\). Sistem se \v casovno razvija naprej od novega za\v cetnega stanja.
    \end{itemize}
\end{enumerate}
Opazimo, da peti aksiom kr\v si \v cetrtega. Ima \v se eno zanimivo implikacijo: Pred meritvijo, je delec v linearni kombinaciji lastnih stanj in lastnosti, ki jih \v zelimo izmeriti, pravzaprav nima. Dobi jih \v sele, ko jih mi izmerimo.
\paragraph{Harmonski oscilator.} Klasi\v cni enodimenzionalni linearni harmonski oscilator v eni dimenziji opisujejo ena\v cbe
\[H = \frac{p^2}{2m} + \frac{1}{2}kx^2 = E\]
\[\ddot{x} + \omega^2x = 0,~\omega = \frac{k}{m}\]
\[x(t) = A\cos(\omega t) + B\sin(\omega t)\]
V kvantni mehaniki to izrazimo kot:
\[H = \frac{p^2}{2m} + \frac{1}{2}kx^2 = -\frac{\hbar^2}{2m}\dd{^2}{x^2} + \frac{1}{2}kx^2 = ...\]
Definiramo konstanto \(\displaystyle \xi^2 = \frac{\hbar}{\omega m}\), kjer je \(\omega\) definirana kot prej.
\[... = \hbar \omega \left(\frac{1}{2}\frac{x^2}{\xi^2} - \frac{1}{2}\xi^2\dd{^2}{x^2}\right)\]
Zdaj bomo naredili nekaj pikantnega: Matematika namre\v c pravi, da je \(a^2 - b^2 = (a + b)(a - b) = (a - b)(a + b)\). Ker sta v na\v sem primeru \(a\) in \(b\) operatorja, ki ne komutirata, moramo to narediti malo druga\v ce, in sicer je:
\[... = \frac{\hbar \omega}{4}\left[\left(\frac{x}{\xi} + \xi\dd{}{x}\right)\left(\frac{x}{\xi} - \xi\dd{}{x}\right) + \left(\frac{x}{\xi} - \xi\dd{}{x}\right)\left(\frac{x}{\xi} + \xi\dd{}{x}\right)\right]\]
Definiramo operator \(\displaystyle a = \frac{1}{\sqrt{2}}\left(\frac{x}{\xi} + \xi\dd{}{x}\right)\). Ker je \(displaystyle \left(\dd{}{x}\right)^\dag = -\dd{}{x}\), je \(\displaystyle a^\dag = \frac{1}{\sqrt{2}}\left(\frac{x}{\xi} + \xi\dd{}{x}\right)\). Operatorju \(a^\dag\) re\v cemo kreacijski, operatorju \(a\) pa anihilacijski. \\
Na\v sa ena\v cba torej dobi obliko
\[H = \frac{\hbar\omega}{2}\left(aa^\dag + a^\dag a\right)\]
Izra\v cunamo \([a, a^\dag]\):
\[aa^\dag - a^\dag a = ... = \frac{1}{2}\left(\frac{x^2}{\xi^2} + \xi\dd{}{x}\frac{1}{\xi}x - \xi^2\dd{^2}{x^2} - \frac{x}{\xi}\xi\dd{}{x}\right) - \frac{1}{2}\left(\frac{x^2}{\xi^2} - \xi\dd{}{x}\frac{1}{\xi}x + \frac{x}{\xi}\xi\dd{}{x} - \xi^2\dd{^2}{x^2}\right) =\]
\[= \dd{}{x}x - x\dd{}{x}\]
Ta operator uporabimo na "testni" funkciji \(\psi\):
\[[a, a^\dag]\psi = \dd{}{x}x\psi - x\dd{}{x}\psi = \psi + x\dd{\psi}{x} - x\dd{\psi}{x} = \psi\]
\[[a, a^\dag] = 1\]
\[aa^\dag = 1 + a^\dag a\]
To da na\v si ena\v cbi da kon\v cno obliko:
\[H = \hbar\omega\left(a^\dag a + \frac{1}{2}\right)\]
Definiramo operator \(\widehat{n} = a^\dag a\). Izka\v ze se, da je sebi adjungiran: \(\widehat{n}^\dag = (a^\dag a)^\dag = a^\dag a = \widehat{n}\). Zado\v s\v ca torej poiskati lastne vrednosti in lastne vektorje \(\widehat{n}\).
\[\widehat{n}\hat{\varphi_\lambda} - \lambda\ket{\varphi_\lambda}\]
Na obeh straneh skalarno pomno\v cimo z \(\bra{\varphi_\lambda}\).
\[\bra{\varphi_\lambda}\widehat{n}\ket{\varphi_\lambda} = \lambda\avg{\varphi_\lambda|\varphi_\lambda}\]
Vemo, da je \(\avg{\varphi_\lambda|\varphi_\lambda} \geq 0\). Iz lastnosti sebi adjungiranih operatorjev lasko sodimo, da je \(\lambda \geq 0\). \\
Preverimo lahko, ali je \(\lambda\) lahko enaka 0. Dobimo diferencialno ena\v cbo \[\left(\frac{x}{\xi} + \xi\dd{}{x}\right)\varphi_0(x) = 0\]
Diferencialna ena\v cba ima re\v sitev
\[\varphi_0 = \frac{1}{\sqrt{\sqrt{\pi}\xi}}\,e^{-\frac{1}{2}\frac{x^2}{\xi^2}}\]
Torej je \(\lambda = 0\) lahko lastna vrednost. Poi\v s\v cimo \v se ostale lastne vrednosti.
\[[\widehat{n}, a^\dag] = [a^\dag a, a^\dag] = a^\dag [a, a^\dag] + [a^\dag, a^\dag]a = a^\dag = \widehat{n}a^\dag - a^\dag\widehat{n}\]
\v Ce je \(\varphi_\lambda\) re\v sitev z lastno vrednostjo \(\lambda\), ra\v cunamo:
\[\widehat{n}\ket{\varphi_\lambda} = \lambda\ket{\varphi_\lambda}\]
\[\widehat{n}a^\dag\ket{\varphi_\lambda} = (a^\dag\widehat{n} + a^\dag)\ket{\varphi_\lambda}\]
\[= (\lambda + 1)\ket{\varphi_\lambda}\]
Hkrati vemo, da je \[|c_\lambda|^2\avg{\varphi_{\lambda + 1}|\varphi_{\lambda + 1}} = \avg{a^\dag\varphi_\lambda|a^\dag\varphi_\lambda} =\]
\[\bra{\varphi_\lambda}aa^\dag\ket{\varphi_\lambda} = \bra{\varphi_\lambda}(1 + a^\dag a)\ket{\varphi_\lambda} = (\lambda + 1)\avg{\varphi_\lambda|\varphi_\lambda}\]
Se pravi dobimo rekurzivno zvezo za lastna stanja \(\varphi_\lambda\):
\[\ket{\varphi_{n+1}} = \frac{a^\dag}{\sqrt{n+1}}\ket{\varphi_{n}}\]
Ali druga\v ce:
\[\ket{\varphi_{n}} = \frac{a^\dag}{\sqrt{n}}\ket{\varphi_{n-1}} = \frac{a^\dag a^\dag}{\sqrt{n(n-1)}}\ket{\varphi_{n-2}} = ...\]
\[= \frac{(a^\dag)^n}{\sqrt{n!}}\ket{\varphi_0}\]
Podobno:
\[[\widehat{n}, a] = ... = -a\]
\[\widehat{n}a\ket{\varphi_n} = ... (n-1)a\ket{\varphi_n} = (n-1)\tilde{c_n}\ket{\varphi_{n-1}}\]
\[\ket{\varphi_n} = \frac{a}{\sqrt{n+1}}\ket{\varphi_{n+1}} = ... = \frac{a^n}{\sqrt{(n+1)!}}\ket{\varphi_n} = \ket{\varphi_0}\]
Se pravi, da gre za naravna \v stevila. Dobimo znani rezultat:
\[H = \hbar\omega\left(\widehat{n} + \frac{1}{2}\right)\]
\[H\ket{\varphi_n} = E_n\ket{\varphi_n}\]
\[E_n = \hbar\omega\left(n + \frac{1}{2}\right)\]
Ali so mo\v zna \v se kak\v sna druga lastna stanja? Vemo, da negativnih lastnih vrednosti ne moremo imeti. Med drugim velja:
\[a\ket{\varphi_0} = 0\]
Zaraadi rekurzivne zveze bi ne-cela lastna vrednosti zahtevala obstoj lastne vrednosti med \(-1\) in \(0\), kar pa je nemogo\v ce.
\end{document}