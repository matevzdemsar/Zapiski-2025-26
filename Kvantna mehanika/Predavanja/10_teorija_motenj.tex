\documentclass[a4paper]{article}
\usepackage{amsmath, amssymb, amsfonts}
\usepackage[margin=1in]{geometry}
\usepackage{graphicx}
\usepackage{tikz}
\usepackage{esint}
\setlength{\parindent}{0em}
\setlength{\parskip}{1ex}

\newcommand{\vct}[1]{\overrightarrow{#1}}
\newcommand{\dif}{\,\mathrm{d}}
\newcommand{\pd}[2]{\frac{\partial {#1}}{\partial {#2}}}
\newcommand{\dd}[2]{\frac{\mathrm{d} {#1}}{\mathrm{d} {#2}}}
\newcommand{\C}{\mathbb{C}}
\newcommand{\R}{\mathbb{R}}
\newcommand{\Q}{\mathbb{Q}}
\newcommand{\Z}{\mathbb{Z}}
\newcommand{\N}{\mathbb{N}}
\newcommand{\fn}[3]{{#1}\colon {#2} \rightarrow {#3}}
\newcommand{\avg}[1]{\left\langle {#1} \right\rangle}
\newcommand{\Sum}[2][0]{\sum_{{#2} = {#1}}^{\infty}}
\newcommand{\Lim}[1]{\lim_{{#1} \rightarrow \infty}}
\newcommand{\Binom}[2]{\begin{pmatrix} {#1} \cr {#2} \end{pmatrix}}
\newcommand{\duline}[1]{\underline{\underline{#1}}}
\newcommand{\bra}[1]{\left\langle {#1} \right|}
\newcommand{\ket}[1]{\left| {#1} \right\rangle}
\renewcommand{\figurename}{Slika}

\begin{document}
\paragraph{Teorija motenj (perturbacij).} Od zadnji\v c: Perturbirano Hamiltonovo funkcijo zapi\v semo kot:
\[H = H_0 + H_1\]
\v Clen \(H_0\) je neperturbiran Hamiltonian z razvojem po lastnih funkcijah
\[H_0\ket{n^0} = E_n^{(0)}\ket{n^0}\]
\(H_1\) zapi\v semo kot \(\lambda\widehat{V}\) in pogledamo limito \(\lambda \to 0\). Tako, kot pri majhnih nihanjih v klasi\v cni mehaniki. \\
Naredimo \v se eno predpostavko, in sicer, da je motnja neodvisna od \v casa (temve\v c samo od kraja). Tedaj govorimo o Rayleigh-Schr\"odingerjevi perturbaciji. I\v s\v cemo lastne vrednosti \(\ket{n}\). Razvijemo:
\[E_n = E_n^{(0)} + \lambda E_n^{(1)} + \lambda^2 E_n^{(2)} + ...\]
Obi\v cajno ne vzamemo veliko \v clenov, in sicer iz dveh razlogov. Prvi\v c, ker nam olaj\v sa ra\v cunanje. Drugi\v c, ker je elektri\v cno polje znotraj atoma veliko mo\v cnej\v se od perturbacij, ki jim lahko izpostavimo doti\v cni atom. \\
Predpostavka:
\[\avg{n^0|n} = \avg{n^0|n^0} + \lambda\avg{n^0|n^1} + \lambda^2\avg{n^0|n^2} + ...\]
\v Ce zahtevamo, da je \(\avg{n^0|n} \equiv 1\), dobimo zahtevo \(\avg{n^0|n^i} = 0\) za \(i \neq 0\).
Zdaj si oglejmo \((H_0 + \lambda V)\ket{n}\):
\[\left(H_0 + \lambda V\right)\left(\ket{n} + \lambda\ket{n^1} + \lambda^2\ket{n^2} + ...\right)\]
\begin{align*}
    \lambda^0: & \quad H_0\ket{n^0} = E_n^{(0)}\ket{n^0} \\
    \lambda^1: & \quad H_0\ket{n^1} + V\ket{n^0} = E_n^{(0)}\ket{n^1} + E_n{(1)}\ket{n^0} \\
    \lambda^2: & \quad H_1\ket{n^2} + V\ket{n^1} = E_n^{(0)}\ket{n^2} + E_n{(1)}\ket{n^1} + E_n^{(2)}\ket{n^0} \\
    \vdots & \\
    \lambda^j: & \quad H_0\ket{n^j} + V\ket{n^{j-1}} = E_n^{(0)}\ket{n^j} + E_n^{(1)}\ket{n^{j-1}} + ... + E_n^{(j)}\ket{n^0}
\end{align*}
Re\v sujemo sistem ena\v cb. Vsako ena\v cbo na obeh straneh skalarno mno\v zimo z \(\bra{n^0}\). Na primer za \(lambda^1\):
\[\bra{n^0}H_0\ket{n^1} + \bra{n^0}V\ket{n^1} = E_n^{(0)}\avg{n^0\vert n^1} + E_n^{(1)}\avg{n^0|n^0}\]
Vemo, da tvorijo \(\ket{n}\) ortonormirano bazo, torej je \(\avg{n^0|n^1} = 0\) in \(\avg{n^0|n^0} = 0\). Dobimo torej:
\[E_n^{(1)} = \bra{n^0}V\ket{n^0} = V_{nn}\]
Zdaj namesto z \(\bra{n^0}\) obeh straneh pomno\v zimo z \(\bra{m^0}\), pri \v cemer \(n \neq m\), obe pa sta lastni funkciji. Spomnimo se, da velja:
\[\sum_m \ket{m^0}\bra{m^0} = I \quad \Rightarrow \quad \ket{n^1} = \sum_{m \neq n}\ket{m^0}\avg{m^0\vert n^1}\]
Sledi:
\[\bra{m^0}H_0\ket{n^1} + \bra{m^0}V\ket{n^1} \left(= E_n^{(0)}\avg{m^0|n^1} + V_{mn}\right) = E_n^{(0)}\avg{m^0\vert n^1} + E_{n}^{(1)}\avg{m^0\vert n^{0}}\]
Vemo, da je \(\avg{m^{0}|n^{1}} = 0\), za \(\avg{m^0\vert n^1}\) pa to ni nujno. Ko iz zgornje ena\v cbe izpustimo ni\v celne \v clene:
\[V_{mn} = \left(E_n^{(0)} - E_m^{(0)}\right)\avg{m^0 \vert n^1}\]
Tako mora biti
\[\ket{n^1} = \sum_{m \neq n} \frac{V_{mn}}{E_n^{(0)} - E_{m}^{(0)}}\avg{m^0}\]
To je super, ker smo iz lastne funkcije \(\ket{m^0}\) in matri\v cnega potenciala \([V]_{mn}\) dobili lastno funkcijo perturbiranega potenciala \(\ket{n^1}\).
Nadaljnje funkcije lahko ra\v cunamo na enak na\v cin:
\[E_n^{(2)} = \bra{n_0}V\sum_{m \neq n}\frac{V_{mn}}{E_n^{(1)} - E_m^{(1)}}\ket{m^{0}} = \sum_{m \neq n}\frac{\bra{n^0}V\ket{m_0}\bra{m^0}V\ket{n_0}}{E_n^{(1)} - E_m^{(1)}} = \sum_{m \neq n}\frac{|V_{mn|^2}}{E_{n}^{(0)} - E_{m}^{(0)}}\]
in tako naprej. Vendar tega ne po\v cnemo pogosto, saj smo predpostavili majhne odmike \(\lambda\), torej nas visoki \v cleni pogosto ne zanimajo. Izjemo dobimo, ko je \(V(x) = -V(-x)\), torej ko je potencial liha funkcija; tedaj je namre\v c \(V_{mn} = 0\) in potrebujemo drugi \v clen. Kadar predpostavimo \(\lambda = 1\) (kar je sicer pogosto), predpostavimo majhne \(V\).
\paragraph{Degeneriran spekter.} Lahko se primeri, da ima ve\v c baznih funkcij isto energijo:
\[H_0\ket{n^0_\alpha} = E_n^{(0)}\ket{n^0_\alpha}\quad \alpha = 1, \dots N\]
Tako je na primer, za \(N = 2\):
\[\ket{n} = c_1\ket{n_1^0} + c_2\ket{n_2^0} + \lambda\ket{n^1} + \dots\]
Naredimo spet isto kot prej:
\begin{align*}
    \lambda^0: & \quad H_0\ket{n^0_1} = E_n^{(0)}\ket{n^0_1} \\
    & \quad H_0\ket{n^0_2} = E_n^{(0)}\ket{n^0_2} \\
    \lambda^1: & \quad
    H_0\ket{n^1} + c_1V\ket{n^0_1} + c_2V\ket{n^0_2} = E_n^{(0)}\ket{n^1} + E_n{(1)}(c_1\ket{n^0_1} + c_2\ket{n^0_2}) \\
\end{align*}
Na obeh straneh izmeni\v cno mno\v cimo z \(\bra{n_1^0}\) in \(\bra{n_2^0}\). Dobimo podoben rezultat kot prej (po enakem postopku):
\[V_{11}c_1 + V_{12}c_2 = E_n^{(1)}c_1\]
\[V_{21}c_1 + V_{22}c_2 = E_n^{(1)}c_2\]
V matri\v cni obliki:
\[\begin{bmatrix}
    V_{11} & V_{12} \\
    V_{21} & V_{22} \\
\end{bmatrix}\begin{bmatrix}
    c_1 \\ c_2
\end{bmatrix} = E_n^{(1)}\begin{bmatrix}
    c_1 \\ c_2
\end{bmatrix}\]
Tako lahko \(c_1, c_2, E_{n,1}^{(1)}, E_{n,2}^{(1)}\) izrazimo tako, da diagonaliziramo matriko potenciala.
\end{document}