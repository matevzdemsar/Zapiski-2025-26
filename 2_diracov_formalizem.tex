\documentclass[a4paper]{article}
\usepackage{amsmath, amssymb, amsfonts}
\usepackage[margin=1in]{geometry}
\usepackage{graphicx}
\usepackage{tikz}
\usepackage{esint}
\setlength{\parindent}{0em}
\setlength{\parskip}{1ex}

\newcommand{\vct}[1]{\overrightarrow{#1}}
\newcommand{\dif}{\,\mathrm{d}}
\newcommand{\pd}[2]{\frac{\partial {#1}}{\partial {#2}}}
\newcommand{\dd}[2]{\frac{\mathrm{d} {#1}}{\mathrm{d} {#2}}}
\newcommand{\C}{\mathbb{C}}
\newcommand{\R}{\mathbb{R}}
\newcommand{\Q}{\mathbb{Q}}
\newcommand{\Z}{\mathbb{Z}}
\newcommand{\N}{\mathbb{N}}
\newcommand{\fn}[3]{{#1}\colon {#2} \rightarrow {#3}}
\newcommand{\avg}[1]{\langle {#1} \rangle}
\newcommand{\Sum}[2][0]{\sum_{{#2} = {#1}}^{\infty}}
\newcommand{\Lim}[1]{\lim_{{#1} \rightarrow \infty}}
\newcommand{\Binom}[2]{\begin{pmatrix} {#1} \cr {#2} \end{pmatrix}}
\newcommand{\duline}[1]{\underline{\underline{#1}}}
\renewcommand{\figurename}{Slika}

\begin{document}
Ehrenfestov teorem pravi: \[\dd{\avg{A}}{t} = \avg{\dd{A}{t}} + \frac{1}{i\hbar}\avg{\left[A, H\right]}\]
kjer je \(A\) poljubna koli\v cina. Najpomembnej\v sa primera sta \(A=x\) in \(A=p\). \\
V primeru \(A = x\):
\[\dd{\avg{x}}{t} = \frac{1}{i\hbar} \avg{\left[x, H\right]} = \frac{1}{i\hbar}\,i\hbar\frac{\avg{p}}{m}\]
Kajti \([x, H] = [x, p^2/2m + V(x, t)] = [x, p^2]/2m = ([x, p]p + p[x, p])/2m = i\hbar p/m\). \\
V primeru \(A = p\):
\[\dd{\avg{p}}{t} = \frac{1}{i\hbar}\avg{\left[p, H\right]} = \frac{1}{i\hbar}\avg{\left[p, \frac{p^2}{2m} + V(x, t)\right]}\]
Izra\v cunamo \([p, V]\):
\[[p, V(x, t)]\psi = (pV - Vp)\psi = -i\hbar\pd{}{x}(V\psi) + i\hbar V\pd{}{x}\psi =\]
\[-i\hbar\pd{V}{x}\psi - i\hbar V\pd{\psi}{x} + i\hbar V\pd{}{x}\psi = \left(-i\hbar\pd{V}{x}\right)\psi\]
Dobili smo torej \[\pd{\avg{p}}{t} = m\pd{\avg{x}^2}{t^2} = -\pd{V}{x}(x, t) = F(x, t)\]
kar je v bistvu samo 2. Newtonov zakon.
\paragraph{Nedolo\v cenost.} V verjetnosti imamo definiran pojem standardne deviacije:
\[\Delta x^2 = \sigma_x^2 = \avg{x^2} - \avg{x}^2 \geq 0\]
V kvantni mehaniki pojem raz\v sirimo na vse operatorje. Zdaj pa recimo, da imamo operatorja \(A\) in \(B\). Zanju velja:
\[\left|\Delta A \Delta B\right| \geq \frac{1}{2}\avg{[A, B]}\]
Dokaz lahko naredimo npr. preko Schwarzeve neenakosti:
\[\left|\int \varphi^*\psi\dif x\right| \leq \int|\varphi^2|\dif x\int\|\psi^2|\dif x\]
Podrobnej\v si dokaz bomo naredili na vajah, za nas je najbolj relevanten primer \(A = x\), \(B = p\). Tedaj je
\[\Delta x \Delta p \geq \frac{1}{2}\hbar\]
\paragraph{Formalizem kvantne mehanike.} Uporabili bomo Diracov formalizem (h kateremu je precej prispeval tudi von Neumann - le-ta je matemati\v cno pokazal, da v kvantni mehaniki lahko predpostavimo dolo\v cene stvari, kot na primer, da valovne funkcije tvorijo vektorski prostor. Upamo, da se ni zmotil, ker se je lem redkim dalo njegove dokaze dejansko preveriti): \\
1. Imamo vektorski prostor, ki je hilbertov oblike \(L^2\). Obstaja baza, ki jo lahko definiramo na razli\v cne na\v cine. Je tudi Banachov prostor, torej lahko element prostora namesto z vsoto baznih vektorjev opi\v semo kot integral nekih vektorjev. \\
2. V tem prostoru imamo skalarni produkt, definiran kot:
\[\avg{\varphi|\psi} = \int \varphi^*\psi\dif x\]
Mimogrede: to se razlikuje od skalarnega produkta, kakr\v snega obi\v cajno definirajo matematiki: prvi\v c po oznaki, drugi\v c po dejstvu, da bi matematiki konjugirali funkcijo \(\psi\). To lahko vodi do nekaterih manj\v sih te\v zav z matemati\v cnega stali\v s\v ca, vendar zaupamo, da je Neumannu to uspelo dobro utemeljiti. \\
Nekaj lastnosti tega skalarnega produkta:
\begin{itemize}
    \item \(\avg{\varphi|\psi} = \avg{\psi|\varphi}^*\)
    \item \(\avg{\psi|\psi} \geq 0\), in je enako 0 le v primeru \(\psi = 0\)
    \item \(|\avg{\varphi|\psi}|^2 \leq \avg{\varphi|\varphi}\avg{\psi|\psi}\)
\end{itemize}
Mimogrede: tej Diracovi notaciji re\v cemo "braket" notacija. "Ket" ali \(|\psi\rangle\) ozna\v cuje opazovano stanje ali vektor.
3. V tem prostoru imamo linearne operatorje: operatorje, za katere velja:
\[A\left(\lambda|\varphi\rangle + \eta|\psi\rangle\right) = \lambda A|\varphi\rangle = \eta A|\psi\rangle\]
Imamo tudi antilinearne operatorje, za katere velja:
\[A\lambda|\psi\rangle = \lambda^*A|\psi\rangle\]
Tak operator je na primer \(t \mapsto -t\).
4. Velja RIeszov izrek: lahko definiramo linearni funkcional \(f|\psi\rangle = z \in \C\), da je
\[\int f^*(x)\psi(x)\dif x = f\psi = \langle f|\psi \rangle\]
Dirac je na podlagi tega izreka uvedel bra; funkcional \(\langle\varphi| = f\), definiran s predpisom
\[\langle\varphi|... = \int \varphi^*(x) ... \dif x\]
V "braket" zapisu je torej "bra" funkcional, "ket" pa opis stanja.
\paragraph{Razvoj stanja po bazi.} V preteklosti smo dolo\v cili koeficiente \(c_n\), da je
\[\psi(x) = \sum_n c_n\varphi_n(x)\]
Tu so \(\varphi_n\) bazni vektorji, za katere naj velja, da so ortonormirani. \\
Ugotovimo \[\psi(x) = \sum_n\varphi_n\int \varphi_n^*(x)\psi(x)\dif x\]
Dirac je uporabljal malo poseben zapis, in sicer je namesto \(|\varphi_n\rangle\) pisal \(|n\rangle\).
\[|\psi\rangle = \sum_n\langle n|\psi\rangle|n\rangle = \sum_n|n\rangle\langle n|\psi\rangle\]
\[=\left(\sum_n|n\rangle\langle n|\right)|\psi\rangle = I\psi\]
Dobili smo torej, da je operacija \(\displaystyle{\left(\sum_n|n\rangle\langle n|\right)}\) enaka identiteti. \\
\paragraph{Razvoj operatorja.} Zdaj si oglejmo funkcijo \(A\psi\), kjer je A poljubni operator.
\[A\psi = IAI|\psi\rangle\]
Dvakrat smo vmes vrinili identiteto, s \v cimer v principu ni ni\v c narobe, nam bo pa po prej\v snjem razmisleku omogo\v cilo, da namesto ene identitete vstavimo
\(\displaystyle{\left(\sum_m|m\rangle\langle m|\right)}\), namesto druge pa \(\displaystyle{\left(\sum_n|n\rangle\langle n|\right)}\).
\[= \sum_{m} |m\rangle\langle m|A|n\rangle\langle n |\psi\rangle\]
\[= |m\rangle\int \varphi^*_m A \varphi_n \dif x \langle n |\psi\rangle\]
Definiramo matriko \(\{A_{mn}\}\), kjer je \(A_{mn} = \langle m|A|n \rangle\)
Dobili smo torej matriko, ki nam v dani bazi opi\v se operator \(A\). Matrika je sicer neskon\v cno dimenzionalna, toda govoto so kak\v sni primeri, ko to ne bo te\v zava. Zapi\v semo lahko:
\[|\psi\rangle = \sum_n c_n|n\rangle\]
\[A|\psi\rangle = \sum_m\left(\sum_n A_{mn} c_n\right)|m\rangle\]
Ozna\v cimo \(d_n = \displaystyle{\sum_n A_{mn} c_n}\) in dobili smo nov vektor: \(A|\psi\rangle = |\psi_1\rangle = \displaystyle{\sum_m d_m|m\rangle}\)
\paragraph{Hermitski simetri\v cni operatorji.} I\v s\v cemo operatorje, za katere velja
\[\langle\varphi|A|\psi\rangle = \langle A\varphi|\psi\rangle,~~\forall \varphi, \psi\]
Nekaj primerov:
\[\langle\psi|A|\psi\rangle = \langle A\psi|\psi\rangle = \langle \psi|A\psi\rangle^* = \langle\psi|A|\psi\rangle^* \in \R\]
\[A|a\rangle = a|a\rangle \Rightarrow \avg{a|A|a} = a\avg{a|a} \in R \Rightarrow a \in \R\]
\[A|a\rangle = a|a\rangle \text{~in~}A|b\rangle = b|b\rangle \Rightarrow \avg{b|A|a} = a\avg{b|a} = b\avg{b|a}\]
Iz zadnjega sledi, da je \((b-a)\avg{b|a} = 0\), torej za neki lastni vrednosti in pridru\v zena lastna vektorja operatorja \(A\) velja \(b = a\) ali pa \(\avg{b|a} = 0\)
\end{document}