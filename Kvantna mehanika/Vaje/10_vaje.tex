\documentclass[a4paper]{article}
\usepackage{amsmath, amssymb, amsfonts}
\usepackage[margin=1in]{geometry}
\usepackage{graphicx}
\usepackage{tikz}
\usepackage{esint}
\setlength{\parindent}{0em}
\setlength{\parskip}{1ex}

\newcommand{\vct}[1]{\overrightarrow{#1}}
\newcommand{\dif}{\,\mathrm{d}}
\newcommand{\pd}[2]{\frac{\partial {#1}}{\partial {#2}}}
\newcommand{\dd}[2]{\frac{\mathrm{d} {#1}}{\mathrm{d} {#2}}}
\newcommand{\C}{\mathbb{C}}
\newcommand{\R}{\mathbb{R}}
\newcommand{\Q}{\mathbb{Q}}
\newcommand{\Z}{\mathbb{Z}}
\newcommand{\N}{\mathbb{N}}
\newcommand{\fn}[3]{{#1}\colon {#2} \rightarrow {#3}}
\newcommand{\avg}[1]{\left\langle {#1} \right\rangle}
\newcommand{\Sum}[2][0]{\sum_{{#2} = {#1}}^{\infty}}
\newcommand{\Lim}[1]{\lim_{{#1} \rightarrow \infty}}
\newcommand{\Binom}[2]{\begin{pmatrix} {#1} \cr {#2} \end{pmatrix}}
\newcommand{\duline}[1]{\underline{\underline{#1}}}
\newcommand{\bra}[1]{\left\langle {#1} \right|}
\newcommand{\ket}[1]{\left| {#1} \right\rangle}
\newcommand{\rot}{\vct{\nabla}\times}
\newcommand{\dvg}{\vct{\nabla}\cdot}
\renewcommand{\figurename}{Slika}

\begin{document}
\section{Spin}
V dveh dimenzijah imamo delec s spinom \(S = 1/2\). Zapi\v semo Hamiltonian
\[H = \frac{p^2}{2m} + \lambda\left(p_x S_y - p_y S_x\right),\quad \vct{p} = (p_x, p_y)\]
Re\v sujemo Schr\"odingerjevo ena\v cbo
\[H\psi = E\psi\]
V nadaljnjih izra\v cunih nam bo koristil komutator \([H, \vct{p}]\). Izra\v cunamo ga po komponentah.
\[[H, p_x] = \left[\frac{p_x^2}{2m} + \frac{p_y^2}{2m} + \lambda\left(p_x S_y - p_y S_x\right),~p_x\right] = 0\]
Koli\v cini \(p_x\) in \(p_x S_y\) sta v razli\v cnih podprostorih, torej gotovo komutirata. Z ostalimi nimamo te\v zav.
Ker komutirata, lahko lastna stanje \(H\) izrazimo z lastnimi stanji \(\vct{p}\). Za krajevni del vemo:
\[\vct{p}\ket{\vct{k}} = \hbar\vct{k}\ket{\vct{k}}\]
\[\vct{p} = -i\hbar\nabla\]
Tak\v sna ena\v cba nam da re\v sitev
\[\psi_{\vct{k}}(\vct{r}) \propto e^{i\vct{k}\cdot\vct{r}}\]
Ker lahko re\v simo krajevni del in ker operatorja \(H\) in \(\vct{p}\) komutirata, lahko uporabimo nastavek
\[\ket{\psi} = \ket{\vct{k}}\ket{\chi}\]
Kjer \(\ket{\vct{k}}\) pomeni krajevni del, \(\ket{\chi}\) spinski del.
\[\left[\frac{p_x^2}{2m} + \frac{p_y^2}{2m} + \lambda\left(p_x S_y - p_y S_x\right)\right]\ket{\vct{k}}\ket{\chi} = E\ket{\vct{k}}\ket{\chi}\]
\[\left(\frac{p^2}{2m}\ket{\vct{k}}\right)\ket{\chi} + \lambda\left(p_x\ket{\vct{k}}\right)\left(S_y\ket{\chi}\right) - \lambda\left(p_y\ket{\vct{k}}\right)\left(S_x\ket{\chi}\right)\]
Vstavimo lastne vrednosti krajevnega dela:
\[\frac{\hbar^2 k^2}{2m}\ket{\vct{k}}\ket{\chi} + \lambda\hbar k_x\ket{\vct{k}} S_y\ket{\chi} - \lambda\hbar k_y\ket{\vct{k}} S_x\ket{\chi} = E\ket{\vct{k}}\ket{\chi}\]
Pokraj\v samo \(\vct{k}\):
\[\frac{\hbar^2 k^2}{2m}\ket{\chi} + \lambda\hbar k_x S_y\ket{\chi} - \lambda\hbar k_y S_x\ket{\chi} = E\ket{\chi}\]
Dobili smo dvidimenzionalen problem v spinskem prostoru. Za \(S = 1/2\) imamo dve mo\v znosti. Ozna\v cimo:
\[S = \frac{1}{2}\qquad\begin{matrix}
    \ket{\frac{1}{2},\,+\frac{1}{2}} = \ket{\uparrow} \\
    \ket{\frac{1}{2},\,-\frac{1}{2}} = \ket{\downarrow} \\
\end{matrix} \qquad \begin{matrix}
    S_x = \frac{S_+ + S_-}{2} \\
    S_y = \frac{S_+ - S_-}{2i}
\end{matrix}\]
Veljajo slede\v ce zveze:
\begin{align*}
    S_+\ket{\uparrow} & = 0 \\
    S_+\ket{\downarrow} & = \hbar\ket{\uparrow} \\
    S_-\ket{\uparrow} & = \hbar\ket{\downarrow} \\
    S_-\ket{\downarrow} & = 0
\end{align*}
Tako izpeljemo zveze za \(S_x\) in \(S_y\):
\begin{align*}
    S_x\ket{\uparrow} & = \frac{\hbar}{2}\ket{\downarrow} \\
    S_x\ket{\downarrow} & = \frac{\hbar}{2}\ket{\uparrow} \\
    S_y\ket{\uparrow} & = \frac{\hbar}{2i}\ket{\downarrow} \\
    S_y\ket{\downarrow} & = \frac{\hbar}{2}\ket{\uparrow} \\
\end{align*}
Za nastavek \(\ket{\chi}\) uporabimo linearno kombinacijo:
\[\ket{\chi} = \alpha\ket{\uparrow} +\beta\ket{\downarrow}\]
\[\frac{\hbar^2k^2}{2m}\left(\alpha\ket{\uparrow} + \beta\ket{\downarrow}\right) + \lambda k_x\frac{\hbar^2}{2i}\left(-\alpha\ket{\downarrow} + \beta\ket{\uparrow}\right) - \lambda k_y \frac{\hbar}{2i}\left(\alpha\ket{\downarrow} + \beta\ket{\uparrow}\right) = E\left(\alpha\ket{\uparrow} + \beta\ket{\downarrow}\right)\]
Iz ujemanja koeficientov dobimo dve linearni ena\v cbi:
\begin{equation}
    \frac{\hbar^2k^2}{2m}\alpha + \lambda\hbar\beta\left(k_x\frac{\hbar}{2i} - k_y\frac{\hbar}{2}\right) = E\alpha
\end{equation}
\begin{equation}
    \frac{\hbar^2k^2}{2m}\beta + \lambda\hbar\alpha\left(-k_x\frac{\hbar}{2i} - k_y\frac{\hbar}{2}\right) = E\beta
\end{equation}
Sistem prepi\v semo v problem lastnih vrednosti:
\[\begin{bmatrix}
    \frac{\hbar^2k^2}{2m} & \frac{\lambda\hbar^2}{2}\left(\frac{k_x}{i} - k_y\right) \\
    \frac{\lambda\hbar^2}{2}\left(-\frac{k_x}{i} - k_y\right) & \frac{\hbar^2k^2}{2m}
\end{bmatrix}\begin{bmatrix}
    \alpha \\ \beta
\end{bmatrix} = E\begin{bmatrix}
    \alpha \\ \beta
\end{bmatrix}\]
\[\begin{vmatrix}
    \frac{\hbar^2k^2}{2m} - E & \frac{\lambda\hbar^2}{2}\left(\frac{k_x}{i} - k_y\right) \\
    \frac{\lambda\hbar^2}{2}\left(-\frac{k_x}{i} - k_y\right) & \frac{\hbar^2k^2}{2m} - E   
\end{vmatrix} = 0 = \left(\frac{\hbar^2k^2}{2m} - E\right) + \frac{\lambda^2\hbar^4}{4}\left(-k^2\right)\]
\[= \left(\frac{\hbar^2k^2}{2m} - E - \frac{\lambda\hbar^2k}{2}\right)\left(\frac{\hbar^2k^2}{2m} - E + \frac{\lambda\hbar^2k}{2}\right)\]
Dobili smo dve lastni vrednosti \(E_1\) in \(E_2\):
\[E_1 = \frac{\hbar^2k^2}{2m} - \frac{\lambda\hbar^2k}{2}\]
\[E_2 = \frac{\hbar^2k^2}{2m} + \frac{\lambda\hbar^2k}{2}\]
Da izra\v cunamo lastni funkciji, nam bo koristila menjava koordinatnega sistema:
\[k_x + ik_y = ke^{i\varphi}\]
\[\frac{k_x}{i} - k_y = \frac{k}{i}\,e^{-i\varphi}\]
\[-\frac{k_x}{i} - k_y = -\frac{k}{i}\,e^{i\varphi}\]
Tako za \(E_1\) dobimo:
\[\lambda\hbar^2\left(\frac{\hbar}{2}\alpha + \frac{\hbar}{2i}\,e^{-i\varphi}\beta\right) = 0 \quad\rightarrow \beta = -i\alpha e^{-i\varphi}\]
Nato dobimo \(\alpha\) iz normalizacije \(|\alpha|^2 + |\beta|^2 = 1\): dobimo \(2|\alpha|^2 = 1\)
\[\chi_1 = \frac{1}{\sqrt{2}}\left(\ket{\uparrow} -ie^{i\varphi}\ket{\downarrow}\right)\]
Podobno:
\[\chi_2 = \frac{1}{\sqrt{2}}\left(\ket{\uparrow} +ie^{i\varphi}\ket{\downarrow}\right)\]
Bolj splo\v sno, za sfero:
\[\alpha\ket{\uparrow} + \beta\ket{\downarrow} = \cos\frac{\vartheta}{2} \pm \sin\frac{\vartheta}{2}e^{i\varphi}\ket{\downarrow}\]
V na\v sem primeru je \(\vartheta = \pi/2\)
\end{document}