\documentclass[a4paper]{article}
\usepackage{amsmath, amssymb, amsfonts}
\usepackage[margin=1in]{geometry}
\usepackage{graphicx}
\usepackage{tikz}
\usepackage{esint}
\setlength{\parindent}{0em}
\setlength{\parskip}{1ex}

\newcommand{\vct}[1]{\overrightarrow{#1}}
\newcommand{\dif}{\,\mathrm{d}}
\newcommand{\pd}[2]{\frac{\partial {#1}}{\partial {#2}}}
\newcommand{\dd}[2]{\frac{\mathrm{d} {#1}}{\mathrm{d} {#2}}}
\newcommand{\C}{\mathbb{C}}
\newcommand{\R}{\mathbb{R}}
\newcommand{\Q}{\mathbb{Q}}
\newcommand{\Z}{\mathbb{Z}}
\newcommand{\N}{\mathbb{N}}
\newcommand{\fn}[3]{{#1}\colon {#2} \rightarrow {#3}}
\newcommand{\avg}[1]{\left\langle {#1} \right\rangle}
\newcommand{\Sum}[2][0]{\sum_{{#2} = {#1}}^{\infty}}
\newcommand{\Lim}[1]{\lim_{{#1} \rightarrow \infty}}
\newcommand{\Binom}[2]{\begin{pmatrix} {#1} \cr {#2} \end{pmatrix}}
\newcommand{\duline}[1]{\underline{\underline{#1}}}
\newcommand{\bra}[1]{\left\langle {#1} \right|}
\newcommand{\ket}[1]{\left| {#1} \right\rangle}
\newcommand{\rot}{\vct{\nabla}\times}
\newcommand{\dvg}{\vct{\nabla}\cdot}
\renewcommand{\figurename}{Slika}

\begin{document}
\paragraph{Feynmanova formulacija.}
Imamo interferen\v cni poskus z dvema re\v zama. Zanima nas verjetnost, da bo delec kon\v cal na neki to\v cki v steni - recimo ji \(x_b\), za\v cetni to\v cki pa \(x_a\).
To naredimo tako, da si zamislimo, po katerih poteh lahko pride do stene. Za vsako pot izra\v cunamo
\[S[x(t)] = \int L \dif t = \int m\left(\frac{x_b - x_a}{t_b - t_a}\right)^2 \dif t\]
in se\v stejemo po vseh mo\v znih poteh. Takemu postopku pravimo Feynmanov integral. Tako je
\[\psi(x_b, t_b) = C \sum e^{i\frac{S[x(t)]}{\hbar}}\]
Za primer si vzamemo vse paraboli\v cne poti med to\v ckama \(x_a\) in \(x_b\). Za tak primer je
\[L = \frac{1}{2}m\left(v_0^2 - \frac{4\delta}{\varepsilon^2} + \frac{4\delta^2}{\varepsilon^4}t^2\right)\]
\[S = \int_{-\varepsilon}^{\varepsilon}L \dif t\]
Zdaj bomo to integrirali po \(\delta\), kar nam da vse mo\v zne poti.Izka\v ze se, da bodo re\v sitve, ki grejo po zelo nenavadni poti,
prispevale zelo malo in se celo izni\v cile med sabo. Najve\c v bodo prispevale poti, ki gredo po najkraj\v si (ali skoraj najkraj\v si) poti.
\paragraph{Feynmanov integral.}
Za\v cnemo z osnovno definicijo Riemannovega integrala: prostorski in \v casovni interval razdelimo na kratke intervale \(\varepsilon\) (ki jih bodi skupaj \(N\)),
nato pa re\v cemo, da se delec od ene do druge to\v cke giblje po premici.
Nato ra\v cunamo
\[\lim_{N \to \infty} \int_{-\infty}^{\infty}\int_{-\infty}^{\infty} \dots \int_{-\infty}^{\infty} e^{i\frac{S[x(t)]}{\hbar}}\dif^N x\]
Rezultat pa razvijemo do prvega \v clena, kar nam da Schr\"odingerjevo ena\v cbo. Te\v zava je edino v tem, da funkcija \(e^{ix}\) v limiti \(x \to \infty\)
ne konvergira. Obi\v cajno s tem nimamo te\v zav, ni pa matemati\v cno rigorozno.
V statisti\v cni termodinamiki imamo podobne postopke pri npr. ra\v cunanju verjetnosti, da se bo polimer postavil v dolo\v ceno obliko.
Verjetnost za posamezno porazdelitev dolo\v cenega zaporedje vozli\v s\v c je \[P = \frac{1}{Z}e^{-\beta H[{\vct{r}_i}]}\]
To je matemati\v cno bolj utemeljeno.
\paragraph{Bohmova interpretacija.}
Za\v cnemo z diferencialno ena\v cbo iz klasi\v cne mehanike:
\[\pd{S}{t} + \frac{(\nabla S)^2}{2m} - VS = 0\]
Vstavimo \(\psi = |\psi|e^{iS/\hbar}\) in dobimo Schr\"odingerjevo ena\v cbo. Iz realnega dela ena\v cbe dobimo
\[\pd{S}{t} + \frac{(\nabla S)^2}{2m} - \frac{\hbar^2}{2m}\frac{\nabla^2|\psi|}{|\psi|} - V = 0\]
Definiramo kvantni potencial
\[Q = \frac{\hbar^2}{2m}\frac{\nabla^2|\psi|}{|\psi|}\]
Na podlagi tega lahko definiramo tudi kvantno silo. Bohmova interpretacija kvantne mehanike pa je, da so pozicije in hitrosti
delcev dolo\v cene vnaprej, vendar jih nismo sposobni izmeriti, in zato pride do nedolo\v cenosti. Tega ne moremo ne potrditi ne ovre\v ci,
nam pa lepo razlo\v zi npr. \(\alpha\) razpad.
\end{document}