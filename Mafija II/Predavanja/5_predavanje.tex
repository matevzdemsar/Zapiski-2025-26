\documentclass[a4paper]{article}
\usepackage{amsmath, amssymb, amsfonts}
\usepackage[margin=1in]{geometry}
\usepackage{graphicx}
\usepackage{tikz}
\usepackage{esint}
\setlength{\parindent}{0em}
\setlength{\parskip}{1ex}

\newcommand{\vct}[1]{\overrightarrow{#1}}
\newcommand{\dif}{\,\mathrm{d}}
\newcommand{\pd}[2]{\frac{\partial {#1}}{\partial {#2}}}
\newcommand{\dd}[2]{\frac{\mathrm{d} {#1}}{\mathrm{d} {#2}}}
\newcommand{\C}{\mathbb{C}}
\newcommand{\R}{\mathbb{R}}
\newcommand{\Q}{\mathbb{Q}}
\newcommand{\Z}{\mathbb{Z}}
\newcommand{\N}{\mathbb{N}}
\newcommand{\fn}[3]{{#1}\colon {#2} \rightarrow {#3}}
\newcommand{\avg}[1]{\langle {#1} \rangle}
\newcommand{\Sum}[2][0]{\sum_{{#2} = {#1}}^{\infty}}
\newcommand{\Lim}[1]{\lim_{{#1} \rightarrow \infty}}
\newcommand{\Binom}[2]{\begin{pmatrix} {#1} \cr {#2} \end{pmatrix}}
\newcommand{\duline}[1]{\underline{\underline{#1}}}
\newcommand{\bra}[1]{\langle {#1} |}
\newcommand{\ket}[1]{| {#1} \rangle}
\renewcommand{\figurename}{Slika}

\begin{document}
\paragraph{Difuzijska ena\v cba.} Imamo ena\v cbo oblike
\[\pd{T}{t} = D \pd{^2T}{x^2}\]
Naredimo slede\v co transformacijo:
\[x \mapsto qx\]
\[t \mapsto q^2t\]
Iskana funkcija \(T\) postane
\[T(x, t) = T\left(\frac{x}{\sqrt{t}}, 1\right) = f(s)\]
Ozna\v cili smo \(s = x/\sqrt{t}\). Velja \(\dif s/\dif x = 1/\sqrt{t}\) in \(dif s / \dif t = x/(2t\sqrt{t})\). Na\v sa ena\v cba postane
\[f'\cdot\left(-\frac{s}{2t}\right) = f''\cdot\frac{D}{t}\]
Pokraj\v samo \(t\) in naredimo separacijo:
\[\frac{f''}{f'} = -\frac{s}{2D}\qquad /\scriptstyle{\int\dif s}\]
\[\ln(f') = -\frac{s^2}{4D} + C\]
\[f(s) = C\int_0^{s} e^{-\zeta^2/4D}\dif \zeta + C'\]
Izbira za\v cetnih pogojev nam omogo\v ci izra\v cun koeficientov \(C\) in \(C'\). \\[2mm]
Npr. za \(T_1(x, t=0) = sgn(x)\) (signum funkcija):
\[T_1(x, t) = \frac{1}{\pi D}\int_0^{x/\sqrt{t}}e^{-s^2/4D}\dif s = \mathrm{erf}\left(\frac{x}{\sqrt{4Dt}}\right)\]
Za \(T_2(x, t=0) = \delta(x)\) (Diracova delta funkcija):
\[T_2(x, t) = \frac{1}{4\pi Dt}\,e^{-x^2/4Dt}\]
Gre ravno za odvod \(\mathrm{erf}\) funkcije:
\[T_2(x, t) = \frac{1}{2}\pd{}{x}T_1(x, t)\]
Tudi superpozicija \(T_2\) je re\v sitev:
\[T(x, t) = \int_{-\infty}^{\infty} g(x_0)T_2(x-x_0, t) \dif x_0\]
Gre pravzaprav za konvolucijo. To pomeni tudi, da je Greenova funkcija problema (fundamentalna re\v sitev):
\[G(x, x_0, t) = \frac{1}{\sqrt{4\pi Dt}}e^{-(x-x_0)^2/4Dt}\Theta(t)\]
\(\Theta(t)\) je Heavisideova funkcija. \v Ce to vstavimo v originalno ena\v cbo, dobimo:
\[\left(\pd{}{t} - D\pd{^2}{x^2}\right)\,G(x, x_0, t) = \delta(x-x_0)\,\delta(t)\]
\paragraph{Robni pogoji.} Mislimo si, da se nahajamo na poltraku. Recimo, da imamo za\v cetni pogoj
\[T(x, t=0) = h(x)\]
in robni pogoj
\[T(x=0, t) = 0\]
Uporabimo t. i. Duhamelov princip:
Zamislimo si, da je na\v s robni pogoj preprost: \(h(x) = 1\). Zapi\v semo \(T(x, t) = [1 - T_1(x, t)]\Theta(t) = \omega(x, t)\). (Gre za \(\mathrm{erf}\), ki smo jo "premaknili" na pravo mesto.) \\
Za splo\v sni robnega pogoja dobimo ena\v cbo
\[h(t) = h(0) + \int h(\tau)\Theta(\tau)\dif \tau\]
Gre za konvolucijo med \(h\) in \(\omega\):
\[T(x, t) = \int_0^t h(\tau)\,\pd{}{t}\omega(x, t-\tau)\dif \tau\]
\paragraph{Separacija.} Re\v sujemo difuzijsko diferencialno ena\v cbo v dveh dimenzijah:
\[\pd{T}{t} = \nabla^2T\]
Z za\v cetnim pogojem \(T(\vct{r}, 0) = f(\vct{r})\). In robnim pogojem \(T=0\) na kvadratu s stranico \(L\). \\
\v Ce je ena\v cba homogena, uporabimo separacijo:
\[T(x, y, t) = \varphi(x, y)u(t) = X(x)Y(y)u(t)\]
\[\frac{\dot{u}}{u} = \frac{\nabla^2\varphi}{\varphi} = \frac{X''}{X} = \frac{Y''}{Y} = -\lambda\]
Oglejmo si, kako izbira \(\lambda\) vpliva na re\v sitev. \\
\begin{itemize}
    \item \(\lambda = 0\): \(u(t) = \text{konst.}\) in \(\varphi(x, y) = 0\). Primerna re\v sitev za \(f(\vct{r}) = 0\).
    \item \(\lambda < 0\): \(u(t) = Ae^{|\lambda|t}\) in \(A = \alpha\cosh(x) + \beta\sinh(x)\). Ta funkcija ima najve\v c eno ni\v clo, torej ne more zadostiti robnemu pogoju.
    \item \(\lambda > 0\): \(u(t) = Ae^{-\lambda t}\) in
    \[X_n = \sqrt{\frac{2}{L}}\sin\left(\frac{n\pi x}{L}\right),\qquad n = 1, 2,\dots\]
    \[Y_n = \sqrt{\frac{2}{L}}\sin\left(\frac{(n+1/2)\pi y}{L}\right)\]
    \[\lambda_{n, m} = \left(\frac{\pi}{L}\right)^2\left[n^2 + \left(n+\frac{1}{2}\right)^2\right]\]
    \[\varphi_{n, m} = X_nY_m\]
\end{itemize}
Splo\v sna re\v sitev je superpozicija re\v sitev pri \(\lambda > 0\):
\[T(x, t) = \sum_{n, m} c_{nm}\varphi_{nm}e^{-\lambda_{nm}t}\]
Koeficiente \(c_{nm}\) dobimo s skalarnim produktom iz za\v cetnih pogojev.
\end{document}