\documentclass[a4paper]{article}
\usepackage{amsmath, amssymb, amsfonts}
\usepackage[margin=1in]{geometry}
\usepackage{graphicx}
\usepackage{tikz}
\usepackage{esint}
\setlength{\parindent}{0em}
\setlength{\parskip}{1ex}

\newcommand{\vct}[1]{\overrightarrow{#1}}
\newcommand{\dif}{\,\mathrm{d}}
\newcommand{\pd}[2]{\frac{\partial {#1}}{\partial {#2}}}
\newcommand{\dd}[2]{\frac{\mathrm{d} {#1}}{\mathrm{d} {#2}}}
\newcommand{\C}{\mathbb{C}}
\newcommand{\R}{\mathbb{R}}
\newcommand{\Q}{\mathbb{Q}}
\newcommand{\Z}{\mathbb{Z}}
\newcommand{\N}{\mathbb{N}}
\newcommand{\fn}[3]{{#1}\colon {#2} \rightarrow {#3}}
\newcommand{\avg}[1]{\langle {#1} \rangle}
\newcommand{\Sum}[2][0]{\sum_{{#2} = {#1}}^{\infty}}
\newcommand{\Lim}[1]{\lim_{{#1} \rightarrow \infty}}
\newcommand{\Binom}[2]{\begin{pmatrix} {#1} \cr {#2} \end{pmatrix}}
\newcommand{\duline}[1]{\underline{\underline{#1}}}
\newcommand{\bra}[1]{\langle {#1} |}
\newcommand{\ket}[1]{| {#1} \rangle}
\renewcommand{\figurename}{Slika}

\begin{document}
\paragraph{Tok po kolobarju.} Upornost prevodne plo\v s\v cice. Plo\v s\v cica je oblike polovice kolobarja z notranjim premerom \(r_1\) in zunanjim premerom \(r_2\). Na kratkih robovih naredimo elektrodi (tako, da je napetost na celotnem posameznem robu enaka) in ju priklju\v cimo na vir napetosti. Zanima nas upornost.
\begin{figure}[h!]
    \centering
    \begin{tikzpicture}[scale=0.8]
    \draw (5, 0) arc (0:180:5);
    \draw (3, 0) arc (0:180:3);
    \draw (-5, 0) -- (-3, 0);
    \draw (5, 0) -- (3, 0);

    \draw[dashed] (0, 0) -- (3, 0);
    \draw[dashed] (0, 0) -- (4.2, 2.5);
    \node (1) at (1.5, -0.5) {\Large \(r_1\)};
    \node (2) at (2.5, 1) {\Large \(r_2\)};
    \node (s) at (0, 4) {\Large \(\sigma\)};
    \draw[<->] (-3, -0.2) -- (-5, -0.2);
    \node (h) at (-4, -0.5) {\Large \(h\)};
    \end{tikzpicture}
\end{figure}
Stacionarni tok opisuje ena\v cba
\[\vct{j} = \sigma\vct{E} = -\sigma\nabla U\]
Zakon o ohranitvi naboja:
\[\pd{e}{t} + \nabla\cdot\vct{j} = 0\]
V na\v sem primeru je prvi \v clen enak 0, torej nam ostane
\[\nabla\cdot\vct{j} = -\sigma \nabla^2 U = 0\]
Dobili smo Laplaceovo ena\v cbo. V cilindri\v cnih koordinatah ima ta re\v sitev
\[U(r, \varphi) = \Sum[1]{n}(A_n\cos n\varphi + B_n\sin n\varphi)(C_n r^m + D_nr^{-n}) + U_0\]
\(U_0\) izra\v cunamo z uporabo separacije:
\[U(r, \varphi) = R(r)\phi(\varphi)\]
To vstavimo v Laplaceovo ena\v cbo in dobimo:
\[r^2R'' + rR' = 0 \Rightarrow R = c\ln r + d\]
\[\phi'' = 0 \Rightarrow \phi(\varphi) = a\varphi + b\]
\[U(r, \varphi) = \Sum[1]{n}(A_n\cos n\varphi + B_n\sin n\varphi)(C_n r^m + D_nr^{-n}) + (a\varphi + b)(c\ln r + d)\]
Robni pogoji: Problem je nastavljen tako, da je potencial na elektrodah konstanten. Torej:
\[U(r, 0) = 0,\quad U(r, \pi) = - U_0\]
Sledi, da morata biti tako \(C_n\) kot \(D_n\) enaka 0, saj bi sicer pri kotu \(\varphi = 0\) dobili radialno odvisnost. Z istim argumentom dobimo \(c = 0\). Na\v sa vrsta torej postane:
\[U(r, \varphi) = \Sum[1]{n}(A_n\cos n\varphi + B_n\sin n\varphi) \cdot 0 + (a \varphi + b) \cdot d = A_0\varphi + B_0\]
Iz robnega pogoja \(U(r, 0) = 0\) dobimo \(B_0 = 0\), iz robnega pogoja \(U(r, \pi)\) pa izra\v cunamo \(A_0\):
\[U(r, \varphi) = -\frac{U_0}{\pi}\varphi\]
Izra\v cunajmo elektri\v cni tok.
\[I = \int\vct{j}\cdot\dif\vct{S} = \int j\cdot h\dif r = \int\sigma \pd{E}{\varphi} h \dif r =\]
Uporabimo \(\displaystyle \pd{E}{\varphi} = -\frac{1}{r}\pd{U}{\varphi}\)
\[= -\int_{r_1}^{r_2} \sigma \frac{1}{r}\pd{U}{\varphi} h\dif r = \frac{\sigma U_0 h}{\pi}\int_{r_1}^{r_2}\frac{\dif r}{r} = \frac{\sigma U_0 h}{\pi}\ln\frac{r_2}{r_1}\]
Iz Ohmovega zakona dobimo:
\[R = \frac{U_0}{I} = \frac{\pi}{\sigma h \ln(r_2/r_1)}\]
\paragraph{Cabrerov eksperiment.} Gre za eksperiment, s katerim so posku\v sali zaznati magnetne monopole. Imamo obro\v c z radijem \(A\) in induktivnostjo \(L\). Proti njemu izstrelimo magnetni monopol. Zanima nas, magnetni pretok \(\phi:\) tako \(\phi(d)\) kot \(\phi(t)\) (z \(d\) smo ozna\v cili razdaljo med monopolom in obro\v cem), nato pa \v se potek \(I(t)\). \\[2mm]
Magnetno polje monopola:
\[\vct{B}_M = \frac{\mu_0}{4\pi}\frac{g}{r^2}\frac{\vct{r}}{r}\]
Kjer \(g\) ozna\v cuje magnetni naboj z enoto \(\mathrm{Am}\). Izra\v cunati moramo integral
\[\phi = \int\vct{B}\cdot\dif\vct{S}\]
\[\dif S = 2\pi\rho \dif \rho\]
\[B = \frac{\mu_0}{4\pi}\frac{g}{(\rho^2 + d^2)}\]
Zanima nas \(x\) komponenta polja, saj je vzporedna z osjo obro\v ca.
\[B_x = B\,\frac{d}{\sqrt{\rho^2 + d^2}} = \frac{\mu_0 g}{4\pi}\frac{d}{(\rho^2 + d^2)^{3/2}}\]
To integriramo po \(\rho\) in s substitucijo \(u = \rho^2 + d^2\) dobimo:
\[\phi(d) = \frac{\mu_0 g d}{4}\,(-2)\left(\frac{1}{\sqrt{a^2 + d^2} - \frac{1}{d}}\right) = \frac{\mu_0g}{2}\left(1 - \frac{a}{d^2 + a^2}\right)\]
Ozna\v cimo s \(t = 0\) \v cas, ko pride monopol v sredino obro\v ca. Tedaj je \(d = \pm vt\), kjer predznak izberemo tako, da \(d\) ni negativen.
\[\phi(t) = \begin{cases}
    \frac{\mu_0 g}{2}\left(1 + \frac{vt}{\sqrt{(vt)^2 + a^2}}\right) & t < 0 \\
    - \frac{\mu_0 g}{2}\left(1 - \frac{vt}{\sqrt{(vt)^2 + a^2}}\right) & t > 0 \\
\end{cases}\]
Zadnji del naloge: kak\v sna je odvisnost \(I(t)\)? Uporabimo indukcijski zakon, pri katerem upo\v stevamo morebitni obstoj monopolov:
\[\nabla\times\vct{E} = -\pd{\vct{B}}{t} - \mu_0\vct{j_m}\]
Obe strani integriramo po \(S\) in uporabimo Stokesov zakon:
\[\oint\vct{E}\cdot\dif\vct{l} = -\pd{}{t}\int\vct{B}\dif\vct{S} - \mu_0\int\vct{j_m}\cdot\dif\vct{S}\]
V vseh treh integralih prepoznamo znane fizikalne spremenljivke:
\[U_i = \pd{\phi}{t} - \mu_0 I_m\]
Ker tok te\v ce skozi zanko samo v \v casu \(t = 0\), pi\v semo \(I_m = g\delta(t)\)
\[U_i = -\dot{\phi} - \mu_0g \delta(t) = L\dot{I}\]
Integriramo po \v casu od \(-\infty\) do \(t\), da dobimo \(I\).
\[\left[\phi(t) - \phi(-\infty)\right] - \mu_0g\delta(t) = L\left[I(t) - I(-\infty)\right]\]
S \(\theta\) smo ozna\v cili Heavyside funkcijo. \v Clene v neskon\v cnosti ozna\v cimo z \(0\) in dobimo
\[I = -\frac{1}{L}\left(\phi(t) + \mu_0g\theta(t)\right)\]
Cabrero se je nato spravil meriti, ali v zanki kdaj pride tdo takega toka, in dejansko dobil nekak\v sen pulz, kar bi kazalo na obstoj magnetnih monopolov, vendar eksperimenta nikomur ni uspelo uspe\v sno ponoviti.
\paragraph{Magnetna indukcija v kvadratnem okvirju.} Imamo dolga vodnika na medsebojni razdalji \(d\), ki sta nekje dale\v stran povezana - tvorita torej zunanjo zanko. Znotraj te zanke je \v se ena kvadratna zanka. Zanima nas medsebojna induktivnost \(L_{12}\).
\begin{figure}[h!]
    \centering
    \begin{tikzpicture}[scale=0.8]
    \draw (-5, 2) -- (5, 2);
    \draw[dashed] (5, 2) -- (5, 0);
    \draw[dashed] (-5, 2) -- (-5, 0);
    \draw (-5, 0) -- (5, 0);
    \draw (0, 0) -- (1, 1) -- (0, 2) -- (-1, 1) -- (0, 0);
    \end{tikzpicture}
\end{figure}
\[\phi_1 = L_{11} I_1\]
\[\phi_2 = L_{21} I_1\]
Stvar formalno definiramo kot:
\[L_{ij} = \frac{\mu_0}{4\pi}\oiint \frac{\dif\vct{l_1}\cdot\dif\vct{l_2}}{|\vct{r_i} - \vct{r_j}|}\]
Opazimo, da lahko indeksa \(i\) in \(j\) brez posledic zamenjamo, se pravi je \(L_{ij} = L_{ji}\). Te naloge ne bomo re\v sevali po definiciji, saj nam ni treba. Vemo, da lahko izra\v cunamo magnetno polje \(B\), ki ga ustvarja ena zanka, in nato primerjamo s pretokom, ki gre zavoljo tega polja skozi drugo zanko.
\[\phi_2 = \int \vct{B} \dif \vct{S}\]
\[B(y) = \frac{\mu_0 I}{2\pi y} + \frac{\mu_0 I}{2\pi(d - y)}\]
\[\dif S = 2 y\dif y\]
\[\phi_2 = \frac{\mu_0I}{\pi}\int_0^{d/2}\left(\frac{1}{y} + \frac{1}{d - y}\right)\,2y\dif y = ... = 2\ln 2 \frac{\mu_0 d I}{\pi}\]
Sledi \(L_{12} = (2\ln 2) \mu_0 d / \pi\).
\end{document}