\documentclass[a4paper]{article}
\usepackage{amsmath, amssymb, amsfonts}
\usepackage[margin=1in]{geometry}
\usepackage{graphicx}
\usepackage{tikz}
\usepackage{esint}
\setlength{\parindent}{0em}
\setlength{\parskip}{1ex}

\newcommand{\vct}[1]{\overrightarrow{#1}}
\newcommand{\dif}{\,\mathrm{d}}
\newcommand{\pd}[2]{\frac{\partial {#1}}{\partial {#2}}}
\newcommand{\dd}[2]{\frac{\mathrm{d} {#1}}{\mathrm{d} {#2}}}
\newcommand{\C}{\mathbb{C}}
\newcommand{\R}{\mathbb{R}}
\newcommand{\Q}{\mathbb{Q}}
\newcommand{\Z}{\mathbb{Z}}
\newcommand{\N}{\mathbb{N}}
\newcommand{\fn}[3]{{#1}\colon {#2} \rightarrow {#3}}
\newcommand{\avg}[1]{\langle {#1} \rangle}
\newcommand{\Sum}[2][0]{\sum_{{#2} = {#1}}^{\infty}}
\newcommand{\Lim}[1]{\lim_{{#1} \rightarrow \infty}}
\newcommand{\Binom}[2]{\begin{pmatrix} {#1} \cr {#2} \end{pmatrix}}
\newcommand{\duline}[1]{\underline{\underline{#1}}}
\newcommand{\bra}[1]{\langle {#1} |}
\newcommand{\ket}[1]{| {#1} \rangle}
\renewcommand{\figurename}{Slika}

\begin{document}
\paragraph{Naloga:} Poi\v s\v cimo vezana stanja delca v kon\v cni, neskon\v cno ozki potencialni jami. Velja:
\[V(x) = -\lambda \delta(x)\]
Pri\v cakujemo, da bomo (vsaj v limiti) dobili rezultat \[\psi_0(x) = \sqrt{\kappa_0}e^{\kappa_0|x|},~~\kappa_0 = \frac{m\lambda}{\hbar^2}\]
Vemo: \(V(-x) = V(x)\). Vemo, da bo valovna funkcija bodisi soda, bodisi liha, zato jo v obeh primerih izra\v cunajmo.
\[\widehat{H} = -\frac{\hbar^2}{2m}\dd{^2}{x^2} - \lambda\delta(x)\]

\[-\frac{\hbar^2}{2m}\dd{^2\psi(x)}{x^2}\lambda\delta(x)\psi(x) = E\psi(x)\]
Na obeh straneh integriramo preko nekega intervala \([-a, a]\) in nato limitiramo \(a \to 0\)
\[-\frac{\hbar^2}{2m}\dd{}{x}\psi(x)\Big|^{a}_{-a} - \lambda\psi(x)\Big|_{x=0} = E\int_{-a}^{a}\psi(x) = 0\]
Stvar je enaka 0, dokler je funkcija omejena.
Sledi: \[\lim_{a \to 0} -\frac{\hbar^2}{2m}\left[\psi'(a)-\psi'(-a)\right] = \lambda\psi(0)\]
Torej odvod valovne funkcije v \(x=0\) ni zvezen. Za opis prvega odvoda valovne funkcije potrebujemo Heavyside funkcijo, kar pomeni, da funkcija sama ne more biti liha. Sledi, da je na\v s nastavek za valovno funkcijo:
\[\psi(x) = \begin{cases}
    A e^{-\kappa x} & x < 0 \\
    A e^{\kappa x} & \text{sicer.}
\end{cases}\]
Zdaj preverimo \v se pogoj \(\lim_{x \to 0} \left[\psi'(x)-\psi'(-x)\right] = -2\kappa_0\psi(0)\)
\[\lim_{x \to 0}\psi' = -A\kappa e^{-\kappa x} = -A\kappa\]
\[\lim_{x \to 0}\psi' = A\kappa e^{\kappa x} = A\kappa\]
\[\lim_{x \to 0} \left[\psi'(x)-\psi'(-x)\right] = -2\kappa = -2\kappa_0\]
Sledi \(\kappa = \kappa_0\)
\paragraph{Naloga:} Za neki \v stevili \(a, b \in \R,~b>a\) imamo potencial oblike \[V(x) = \begin{cases}
    V_1 & x < a \\
    V_2 & x > b \\
    V_3(x) & \text{sicer}
\end{cases}\]
Vemo, da je lahko \(V_3(x) > E\), toprej obstaja verjetnost, da delec uide iz vezanega stanja. Zanima nas, kak\v sna ta verjetnost je. \\[2mm]
Gostota verjetnostnega toka: \[j(x) = \frac{\hbar}{2mi}\left[\psi^*(z)\psi'(x) - \psi(x)\psi^{*}\text{}'(x)\right]\]
Vemo, da je na obmo\v cjih \(1\) in \(2\) valovna funkcija enaka:
\[\psi_1(x) = \frac{1}{\sqrt{V_1}} \left[A_1 e^{ik_1x}+B_1e^{-ik_1x}\right]\]
\[\psi_2(x) = \frac{1}{\sqrt{V_2}} \left[A_2 e^{ik_1x}+B_2e^{-ik_2x}\right]\]
Pri \v cemer je \(\displaystyle{k = \sqrt{\frac{2m(E-V)}{\hbar^2}}}\).
Ko to vstavimo v ena\v cbo za verjetnostni tok, dobimo:
\[j_1(x) = |A_1|^2 - |B_1|^2\]
\[j_2(x) = |B_2|^2 - |A_2|^2\]
Kaj lahko povemo o obmo\v cju 3? Potenciala ne poznamo, pri\v cakujemo pa, da lahko zanj zapi\v semo valovno funkcijo, ki bo
linearna kombinacija dveh neodvisnih re\v sitev (matematiki pravijo, da to smemo).
\[\psi_3(x) = C\varphi(x) + D\chi(x)\]
Imamo tudi robne pogoje (kajti pri\v cakujemo, da bo na\v sa valovna funkcija zvezna in zvezno odvedljiva).
\begin{align*}
    \psi_1(a) & = \psi_3(a) \\
    \psi_1'(a) & = \psi_3'(a) \\
    \psi_2(b) & = \psi_3(b) \\
    \psi_2'(b) & = \psi_3'(b) \\
\end{align*}
V igri imamo \v set konstant (\(A_1, B_1, A_2, B_2, C, D\)), pri \v cemer konstanti \(A_1\) in \(A_2\) obravnavamo kot \v ze vnaprej znani. Vemo, da lahko tako izrazimo \(C\) in \(D\), da dobimo le ena\v cbi za \(B_1\) in \(B_2\). Dobili bomo sistem
\[\begin{bmatrix}
    B_1 \\ B_2
\end{bmatrix} = \begin{bmatrix}
    r & t' \\
    t & r'
\end{bmatrix}\begin{bmatrix}
    A_1 \\ A_2
\end{bmatrix} \equiv S\begin{bmatrix}
    A_1 \\ A_2
\end{bmatrix}\]
Matriki \(S\) re\v cemo sipalna matrika. Je kompleksna in njenih koeficientov zaenkrat ne bomo ra\v cunali (ne zdaj, temve\v c naslednji\v c). Vemo pa:
\begin{align*}
    j_1(x) & = 1 - |r|^2 \\
    j_2(x) & = |t|^2
\end{align*}
Koli\v cini \(R = |r|^2\) pravimo represivnost (ali odbojnost), \(T = |t|^2\) pa transmitivnost (ali prepustnost). Pri\v cakujemo, da bo njuna vsota enaka 1 (v relativisti\v cni kvantni mehaniki to ni nujno res - delci se lahko anihilirajo ali zlijejo ali kaj podobnega, s takimi primeri se ne bomo ukvarjali).
Se\v stejmo celoten tok, ki potuje prosti sipalcu in stran od sipalca: \\
\begin{tabular}{r l}
    Proti sipalcu: & \(~|A_1|^2 + |A_2|^2\) \\
    Stran od sipalca: & \(~|B_1|^2 + |B_2|^2\)
\end{tabular}
\[|A_1|^2 + |A_2|^2 = (A_1^*, A_2^*)\begin{pmatrix}
    A_1 \\ A_2
\end{pmatrix} = A^HA\]
\v Ce se verjetnost nikjer ne izgublja (kar bi veljalo v relativisti\v cnih primerih, mi pa to ignoriramo), velja:
\[A^HA = B^HB\]
\[A^HA = (SA)^H(SA)\]
\[A^HA = A^H(S^HS)A\]
Sledi, da je \(S^HS = I\) oziroma je \(S\) unitarna. \\[2mm]
\v Ce je Hamiltonian realen:
\[\psi_1^*(x) = \frac{1}{\sqrt{V_1}}\left[A_1^*e^{-ik_1x} + B_1^* e^{ik_1x}\right]\]
\[\psi_2^*(x) = \frac{1}{\sqrt{V_2}}\left[B_2^*e^{-ik_2x} + A_2^* e^{ik_2x}\right]\]
Spet je \(A^* = SB^*\) oziroma \(S^TA^* = B^*\).
Sledi: \(S^T = S\), torej \(t' = t\). Ponuja se vpra\v sanje, ali Hamiltonian kdaj ni realen? Da, in sicer v primeru magnetnega polja. Iz klasi\v cne mehanike se spomnimo:
\[H = \frac{(\vct{p} - e\vct{A})^2}{2m} + V(\vct{r})\]
Pri prehodu v kvantnon mehaniko vektor \(\vct{A}\) ostane realen, vektor \(\vct{p}\) pa postane \(i\hbar\nabla\). Tako da realnosti Hamiltoniana ne moremo kar tako predpostaviti. \\
Mi bomo obravnavali primer, ko je realen, in hkrati poseben primer, ko je potencial soda funkcija. To pomeni \(k_1 = k_2 = k\) in \(V_1 = V_2 = V\). Izrazimo \(\psi(-x)\):
\[\psi_1(-x) = \frac{1}{\sqrt{V}} = B_2e^{-ikx} + A_2e^{ikx}\]
\[\psi_2(-x) = \frac{1}{\sqrt{V}} = B_1e^{-ikx} + A_1e^{ikx}\]
Zapi\v simo ena\v cbo v matri\v cni obliki:
\[\begin{pmatrix}
    0 & 1 \\
    1 & 0
\end{pmatrix}\begin{pmatrix}
    B_1 \\ B_2
\end{pmatrix} = \begin{pmatrix}
    B_2 \\ B_1
\end{pmatrix} = S \begin{pmatrix}
    A_2 \\ A_1
\end{pmatrix} = S \begin{pmatrix}
    0 & 1 \\
    1 & 0
\end{pmatrix}\begin{pmatrix}
    A_1 \\ A_2
\end{pmatrix}\]
Novo matriko ozna\v cimo s \(\sigma_x\).
\[B = \sigma_x S \sigma_x A\]
\[\sigma_x S \sigma_x = \begin{pmatrix}
    r' & t \\
    t' & r
\end{pmatrix}\]
Sledi, da je \(r' = r\) in \(t' = t\).
\end{document}