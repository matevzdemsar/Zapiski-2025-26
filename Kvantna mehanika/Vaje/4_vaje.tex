\documentclass[a4paper]{article}
\usepackage{amsmath, amssymb, amsfonts}
\usepackage[margin=1in]{geometry}
\usepackage{graphicx}
\usepackage{tikz}
\usepackage{esint}
\setlength{\parindent}{0em}
\setlength{\parskip}{1ex}

\newcommand{\vct}[1]{\overrightarrow{#1}}
\newcommand{\dif}{\,\mathrm{d}}
\newcommand{\pd}[2]{\frac{\partial {#1}}{\partial {#2}}}
\newcommand{\dd}[2]{\frac{\mathrm{d} {#1}}{\mathrm{d} {#2}}}
\newcommand{\C}{\mathbb{C}}
\newcommand{\R}{\mathbb{R}}
\newcommand{\Q}{\mathbb{Q}}
\newcommand{\Z}{\mathbb{Z}}
\newcommand{\N}{\mathbb{N}}
\newcommand{\fn}[3]{{#1}\colon {#2} \rightarrow {#3}}
\newcommand{\avg}[1]{\langle {#1} \rangle}
\newcommand{\Sum}[2][0]{\sum_{{#2} = {#1}}^{\infty}}
\newcommand{\Lim}[1]{\lim_{{#1} \rightarrow \infty}}
\newcommand{\Binom}[2]{\begin{pmatrix} {#1} \cr {#2} \end{pmatrix}}
\newcommand{\duline}[1]{\underline{\underline{#1}}}
\newcommand{\bra}[1]{\langle {#1} |}
\newcommand{\ket}[1]{| {#1} \rangle}
\renewcommand{\figurename}{Slika}

\begin{document}
\paragraph{Naloga.} Imamo Gaussov valovni paket:
\[\psi(x, 0) = \frac{1}{\sqrt[4]{2\pi\sigma^2}}e^{-\frac{(x-\avg{x})^2}{4\sigma^2}}e^{i\frac{\avg{p}}{\hbar}x}\]
Zanima nas \v casovni razvoj \(\avg{x, t}\) pri konstantnem potencialu. Spomnimo se:
\[\avg{x, t} = \int\psi^*(x, t) x \psi(x, t)\dif x\]
O za\v cetnem stanju vemo:
\[\avg{x, 0} = \avg{x} \qquad \avg{p, 0} = \avg{p}\]
\[\delta x(0) = \sigma\]
\[\delta p(0) = \frac{\hbar}{2\sigma}\]
Funkcijo \(\psi\) lahko razvijemo po lastnih funkcijah. Dobimo:
\[\psi(0) = \int c(k)e^{ikx}\dif k\]
V tem prepoznamo Fourierovo transformacijo:
\[c(k) = \frac{1}{2\pi} \int \psi(x, 0) e^{-ikx} \dif x\]
In spet nazaj:
\[\psi(x, t) = \int c(\kappa) e^{-i\frac{E_kt}{\hbar}} e^{i\kappa x} \dif \kappa\]
Iz spodnjih dveh izrazov bi lahko dobili re\v sitev, vendar bi bila to le dolgotrajna vaja iz integriranja. Obstaja elegantnej\v si na\v cin.
\[\ket{\psi, t} = e^{-iHt\hbar}\ket{\psi t}\]
\[\bra{\psi, t} = \ket{\psi, t}^\dag = \bra{\psi, 0}e^{iHt/\hbar}\]
Za poljuben operator \(A\) torej velja:
\[\bra{\psi, t} A \ket{\psi, t} = \bra{\psi, 0} e^{iHt/\hbar} A e^{-iHt/\hbar}\ket{\psi, 0} = \bra{\psi, 0} A(t) \ket{\psi, 0}\]
Nekaj koristnih posledic tovrstnega zapisa:
\[(\alpha A + \beta B)(t) = \alpha A(t) + \beta B(t)\]
\[(AB)(t) = A(t)B(t)\]
\[A^\dag(t) = \left(A(t)\right)^\dag\]
\[\dd{}{t}A(t) = e^{iHt/\hbar} \frac{iH}{\hbar} A e^{-iHt/\hbar} + e^{iHt/\hbar} A e^{-iHt/\hbar}\left(-\frac{iH}{\hbar}\right)\]
Ker operatorja \(f(A)\) in \(A\) komutirata (kar bi lahko dokazovali z razvojem v funkcijsko vrsto), izpostavimo:
\[e^{-Ht/\hbar} \left(\frac{iH}{\hbar}A - A\frac{iH}{\hbar}\right)e^{iHt/\hbar}\]
V izrazu na sredini prepoznamo komutator \([H, A]\).
\[\dd{}{t}A(t) = \frac{i}{\hbar}[H, A](t)\]
To je v resnici diferencialna ena\v cba za operator \(A\) z za\v cetnim pogojem \(A(0) = A\).
Nas posebej zanimajo operatorji \(x\), \(p\), \(x^2\), \(p^2\).
\[\dd{}{t}x(t) = \frac{i}{\hbar}[H, x](t)\]
Izra\v cunamo komutator \([H, x]\):
\[[H, x] = [\frac{1}{2m}p^2, x]\]
Za la\v zje ra\v cunanje tu omenimo nekaj lastnosti komutatorja:
\[AB, C = A[B, C] + [A, C]B\]
\[[\alpha A + \beta B, C] = \alpha[A, C] + \beta[B, C]\]
\[[p, x] = -i\hbar\]
Sledi:
\[[H, x] = \frac{1}{2m}\left(p(-i\hbar) - (i\hbar)p\right) = -\frac{i\hbar p}{m}\]
\[\dd{}{t}x(t) = \frac{p(t)}{m}\]
Poglejmo si zdaj operator \(p\):
\[\dd{}{t}p(t) = \frac{i}{\hbar}[H, p](t) = \frac{1}{2m}[p^2, p] = 0\]
Dobili smo sistem linearnih diferencialnih ena\v cb. Re\v sitev je dokaj enostavna:
\[p(t) = p(0) = p = \text{konst.}\]
\[\dd{x}{t} = \frac{p}{m}\]
\[x(t) = \frac{p}{m}t + x(0)\]
Dobili smo premo enakomerno gibanje (kar je smiselno, saj smo predpostavili konstanten potencial, torej ni sil). Izrazimo \v se \v casovni razvoj:
\[\avg{x, t} = \frac{t}{m}\avg{p} + \avg{x}\]
\[\avg{p, t} = \avg{p}\]
\end{document}